Сценарии использования (см. Приложение \No 1 к настоящему документу) реализуются только в рамках требований, изложенных в п. \ref{funcreqs} \nameref{funcreqs}.

\section{Авторизация пользователя}
\ifcand
\subsection{Согласованные на 1 релиз}
\fi

\UCsubsubsection{Клиент хочет получать информацию о неудавшемся входе на сайт и причину неудачи}{HBR.TS.001.02}
{

\sect{Исходные данные c Wiki}

\begin{wiki}
Если клиент ввел не корректный логин/пароль, ему должна отобразиться страница с информацией, example (все поля можно  оставить как есть в примере)
\end{wiki}

\begin{teamidea}
Нужно определиться, будет ли это всплывающее окно или отдельная страница на сайте. Если отдельная - над или под ней будет форма.  В идеале нужен прототип. Но требование простое, можно просто ответить словами (ответ TN: <<Это будет всплывающее окно>>)
\end{teamidea}


\sect{Итоговая формулировка User Story}


\begin{itogo}
Пользователь вводит некорректный логин или пароль; При отправке формы пользователь получает сообщение "Попробуйте ввести реквизиты еще раз или воспользуйтесь сервисом восстановления пароля";
\end{itogo}

\sect{Пункты спецификации, имеющие отношение к требованию}

\clonerequirement{ID.50}
\clonerequirement{ID.240}

}


%\UCsubsubsection{Клиент хочет видеть правила ввода данных при авторизации или регистрации}{HBR.TS.001.05}
%{
%
%\sect{Исходные данные c Wiki}
%
%\begin{wiki}
%Если система определила некорректно введенные данные, или определила, что пароль, выбранный при регистрации не является надежным, она должна отобразить эту информацию, и автоматически сбросить некорректные параметры при формировании учетной записи, подсветив поле с некорректной записью красным окном или отобразить эту информацию красным шрифтом над регистрационной формой
%\end{wiki} 
%
%\sect{Пояснения}
%
%\begin{teamidea}
%На первом этапе нет формы регистрации. Для формы авторизации оба поля обязательны. В случае ошибочно заполненной формы указывать на поле, в котором была сделана ошибка, нельзя по соображениям безопасности. Поэтому этот User Story для первого этапа целиком повторяет \textbf{HBR.TS.001.02} 
%\end{teamidea}
%
%\sect{Итоговая формулировка User Story}
%
%\begin{itogo}
%Пользователь вводит некорректный логин или пароль;
%При отправке формы пользователь получает сообщение "Указанные вами реквизиты доступа к системе некорректные. Воспользуйтесь сервисом восстановления пароля";
%\end{itogo}
%
%}


%\UCsubsubsection{Клиент хочет видеть подсказки по ходу процесса регистрации}{HBR.TS.001.06}
%{
%\sect{Исходные данные c Wiki}
%
%\begin{wiki}
%Напротив полей, предназначенных для ввода показывать знаки (например знак вопроса) и при нажатии на него выводить всплывающую подсказку, в grainger информация отображается в отдельном окне браузера, происходит переход на другое окно , но клиент не покидает форму регистрации.
%\end{wiki}
%
%\sect{Итоговая формулировка User Story}
%
%\begin{itogo}
%При авторизации пользователь может получить подсказку, что означает каждое поле, нажав на иконку рядом (если это предусмотрено дизайном);
%\end{itogo}
%
%
%
%}

\UCsubsubsection{Клиент хочет online связаться с call-центром и получить помощь в регистрации}{HBR.TS.001.07}
{

\sect{Исходные данные c Wiki}

\begin{wiki}
При прохождении регистрации, клиенту необходимо видеть сервис "связаться с оператором" или телефон номера 8-800, для оперативной поддержки, а так же необходима информация клиенту, призывающая связаться с оператором, если регистрация на сайте не доступна/не получается и т.д., что бы клиент при многоразовых попытках ввода своих данных, не ушел со страницы регистрации. Либо эта подсказка может появляться как всплывающее окно, после трех-кратно неудачных попыток регистрации. Здесь стоит понимать, что в хайбрис при многократных попытках происходить ничего не будет, данный модуль будет доработкой. вопрос-принимаем или нет. 
\end{wiki}

\begin{teamidea}
Требование сформулированно некорректно. "Клиент хочет online связаться с call-центром" - это что-то про онлайн-чат и онлайн-поддержку. В описании же указано совсем иное - это просто информационное окно.
Логика вида "После N неудачных попыток показывать информацию", думаю, избыточна. Достаточно показывать эту строчку просто всегда.
(ответ Технониколь: стоит обсудить потребность в данном функционале /Сорокин Андрей Романовский Виктор/)
\end{teamidea}

\sect{Итоговая формулировка User Story}

\begin{itogo}
 При прохождении регистрации, клиенту необходимо видеть ссылку на сервис <<связаться с оператором>>\footnote{Сервис не входит в объем проекта} или телефон номера 8-800, для оперативной поддержки.
\end{itogo}
}

\UCsubsubsection{Клиент хочет разлогиниться и выполнить вход под другими учетными данными}{HBR.TS.001.09}
{

\sect{Исходные данные c Wiki}

\begin{wiki}
Клиент просто выходит из системы при помощи кнопки "выйти" и при входе вводит необходимый для него логин/пароль. Это не должно быть сложно. Да же если клиент запомнил свои данные, и при новом входе на сайт, вверху сразу отобразилась последняя используемая учетная запись, он прjсто рядом с этим же меню выходит из аккаунта, и при помощи кнопки "войти" может войти под другим аккаунтом. 
\end{wiki}

\sect{Итоговая формулировка User Story}

\begin{itemize}
\item Клиент выходит из системы при помощи кнопки <<выйти>>. При входе вводит необходимый для него логин/пароль.  
\end{itemize}

\sect{Пункты спецификации, имеющие отношение к требованию}

\clonerequirement{ID.590}

}



\UCsubsubsection{Клиент хочет разместить заказ по телефону}{HBR.TS.001.10}
{
\sect{Исходные данные c Wiki}

\begin{wiki}
Клиент звонит по телефону сам, или оставляет заявку, что бы с ним связались -оставив заявку по телефону, если у клиента уже создан аккаунт, этот заказ так же должен отобразиться в его личном кабинете в разделе заказов, оставленных по телефону. см. табличку п. 2 статус заказа.
\end{wiki}

\begin{teamidea}
Потребуется доработка "Заявка на обратный звонок" (1-2 человеко-дня)
(ответ ТехноНиколь: здесь ключевое требование не столько заявка на обратный звонок, сколько отображения заказов, сделанных через телефон, в личном кабинете-предотвратить потерю истории Алиев Рауф)
\end{teamidea}

\sect{Итоговая формулировка User Story}

\begin{itogo}
Клиент может оставтиь заявку по телефону, если у клиента уже создан аккаунт. Этот заказ также должен отобразиться в его личном кабинете в разделе заказов, оставленных по телефону.
\end{itogo}


\sect{Пункты спецификации, имеющие отношение к требованию}

\clonerequirement{ID.1460}
}

\UCsubsubsection{Клиент хочет зарегистрироваться в процессе оформления заказа}{HBR.TS.001.11}
{

\sect{Исходные данные c Wiki}

\begin{wiki}
После того, как клиент положил товар в корзину и нажал checkout, ему будет предложено или войти как авторизованному пользователю или зарегистрироваться. Если же он решил пройти регистрацию до момента checkout, он должен пройти регистрацию, при этом товары, накиданные в корзину должны остаться в ней. 
\end{wiki}

\begin{teamidea}
Это в определенной степени не согласуется с желанием таскать с собой корзину между устройствами. Типа положил что-то на мобильном (или дома), а на десктопе (или на работе) после авторизации это что-то увидел уже положенным в корзину. Тем не менее, требуемый порядок может реализовать на первом этапе
(Ответ ТехноНИКОЛЬ: Рауф, ты говоришь про кроссдевайсность, это хорошая вещь, но вряд ли на 1 этапе)
\end{teamidea}

\begin{tn}
<<Добавить сообщение о необходимости связаться с call-центром>> 
(Сорокин Андрей)
\end{tn}

\sect{Итоговая формулировка User Story}

\begin{itogo}
После того, как клиент положил товар в корзину и нажал checkout, ему будет предложено или войти как авторизованному пользователю или зарегистрироваться. Если же он решил пройти регистрацию до момента checkout, товары, положенные в корзину, должны остаться в ней. Также должно выводиться сообщение о необходимости связаться с колл-центром в случае любых проблем с регистрацией. 
\end{itogo}

\sect{Пункты спецификации, имеющие отношение к требованию}

\clonerequirement{ID.1510.1}
\clonerequirement{ID.61}
}

\UCsubsubsection{Клиент хочет видеть информацию о том, что он не может завершить заказ, пока не авторизуется}{HBR.TS.001.13}
{
\sect{Исходные данные c Wiki}

\begin{wiki}
При совершении заказа, клиент должен не просто видеть вариант авторизации или регистрации, он должен понимать, что заказ в один клик, без авторизации/регистрации -невозможен. Это необходимо отобразить при помощи дополнительного сообщения.
\end{wiki}

\sect{Итоговая формулировка User Story}

\begin{itogo}
Необходимо отобразить информационное сообщение, что заказв один клик без авторизации невозможен.
\end{itogo}

Заказчик предоставляет информационное сообщение и дизайн-макеты для реализации этого требования до начала работ по этому блоку.

\sect{Пункты спецификации, имеющие отношение к требованию}

\clonerequirement{ID.1510.1}
}

\UCsubsubsection{Клиент хочет иметь возможность автоматической авторизации}{HBR.TS.001.14}
{
\sect{Исходные данные c Wiki}

\begin{wiki}
При входе в личный кабинет, необходимо предоставить клиенту запомнить его учетные данные, и при повторном входе на сайт, не запрашивать их., в том числе и при оформлении заказа. После сохранения данных, входа на сайт и добавления товара в корзину, после checkout система пропустила шаг с авторизацией, и сразу выдала шаг-выбор способа доставки и оплаты. 
\end{wiki}

\sect{Итоговая формулировка User Story}

\begin{itogo}
При входе в личный кабинет, необходимо предоставить клиенту запомнить его учетные данные, и при повторном входе на сайт, не запрашивать их, в том числе и при оформлении заказа (с учетом ограничений, накладываемых браузером). 
Для авторизованного пользователя заказ проходит в обход формы авторизации, так как авторизация уже совершена ранее.
\end{itogo}

}

\ifcand
\subsection{Кандидаты на последующие этапы}
\UCsubsubsection{Клиент, в зависимости от роли,хочет пройти авторизацию или регистрацию на сайте}{HBR.TS.001.01}{}
\UCsubsubsection{Клиент хочет изменить пароль или логин}{HBR.TS.001.03}{}
\UCsubsubsection{Клиент забыл пароль и хочет получить новый/старый пароль на свой e-mail}{HBR.TS.001.04}{}
\UCsubsubsection{Клиент хочет получить на свой e-mail ссылку для подтверждения регистрации}{HBR.TS.001.08}{}
\UCsubsubsection{Клиент хочет видеть статус своего запроса на регистрацию}{HBR.TS.001.12}{}
\fi

\section{Личный кабинет пользователя}

\ifcand
\subsection{Согласованные на 1 релиз}
\fi
\UCsubsubsection{Клиент зашел как покупатель}{HBR.TS.002.01}
{
\sect{Исходные данные c Wiki}

\begin{wiki}
Клиент в роли покупателя не может использовать функции управления аккаунтами и может смотреть историю только по своим \sout{покупкам} и заказам
\end{wiki}

\begin{hybris}
История по покупкам и заказам - это история заказа и его содержимого?
\end{hybris}

\sect{Итоговая формулировка User Story}

\begin{itogo}
Клиент в роли покупателя не может использовать функции управления аккаунтами и может смотреть историю только по своим заказам
\end{itogo}

\sect{Пункты спецификации, имеющие отношение к требованию}

\clonerequirement{ID.450}
\clonerequirement{ID.275}


}
\UCsubsubsection{Клиент зашел как утверждающий заказы}{HBR.TS.002.02}
{

\sect{Исходные данные c Wiki}

\begin{wiki}
Клиент в роли утверждающего заказы наследует права покупателя, и может видеть в своем ЛК уведомления о наличии заказов, которые необходимо утвердить
\end{wiki}

\begin{teamidea}
В логике станлартного акселлератора он не обязательно может наследовать.
\end{teamidea}

\begin{hybris}
Нет наследования, есть группы с соответствующим функционалом. B2BCustomer роль может покупать. B2B Approver может утверждать. Соответственно пользователь может получить обе роли
\end{hybris}

\sect{Итоговая формулировка User Story}

\begin{itogo}
Клиент в роли утверждающего заказы может иметь права покупателя. Может видеть в своем личном кабинете уведомления о наличии заказов, которые необходимо утвердить
\end{itogo}

\sect{Пункты спецификации, имеющие отношение к требованию}

\clonerequirement{ID.220.1}

}
\UCsubsubsection{Клиент зашел как менеджер}{HBR.TS.002.03}
{

\sect{Исходные данные c Wiki}

\begin{wiki}
Клиент в роли менеджера наследует роль покупателя, и может видеть все покупки и все заказы сотрудников своей организации
\end{wiki}

\begin{teamidea}
В логике стандартного акселлератора он не обязательно может наследовать.
\end{teamidea}

\sect{Итоговая формулировка User Story}

\begin{itogo}
Клиент в роли менеджера может иметь роль покупателя, и может видеть все покупки и все заказы сотрудников своей организации
\end{itogo}

\sect{Пункты спецификации, имеющие отношение к требованию}

\clonerequirement{ID.440}
\clonerequirement{ID.500}
\clonerequirement{ID.235}
}



\UCsubsubsection{Клиент зашел как администратор}{HBR.TS.002.04}
{
\sect{Исходные данные c Wiki}

\begin{wiki}
Клиент в роли администратора может использовать все функции ЛК
\end{wiki}

\sect{Итоговая формулировка User Story}

\begin{itogo}
Клиент в роли администратора может использовать все функции ЛК
\end{itogo}

\sect{Пункты спецификации, имеющие отношение к требованию}

\clonerequirement{ID.210.1}
\clonerequirement{ID.220.1}
\clonerequirement{ID.230.1}
\clonerequirement{ID.30}
}


\UCsubsubsection{Клиент хочет отредактировать свои персональные данные}{HBR.TS.002.08}
{
\sect{Исходные данные c Wiki}

\begin{wiki}
Клиент, в роли администратора, переходит на вкладку "Мои персональные данные" и может редактировать персональные данные:
\begin{itemize}
\item User ID
\item Страну
\item ФИО
\item Почтовые адреса
\item телефон
\item Так же клиент может добавлять:
	\begin{itemize}
	\item несколько вариантов телефонов
	\item несколько вариантов адресов
	\item несколько вариантов e-mail, без изменения текущего
	\end{itemize}
\end{itemize}
\end{wiki}

\begin{hybris}
То есть пользователь с другой ролью не может делать эти изменения?
\end{hybris}

\begin{tn}
<<Заменить UserID на Логин>> (Сорокин Андрей)
\end{tn}

\sect{Итоговая формулировка User Story}

\begin{itogo}
Клиент, в роли администратора, переходит на вкладку "Мои персональные данные" и может редактировать персональные данные:
\begin{itemize}
	\item Логин
	\item Страну
	\item ФИО
	\item Почтовые адреса
	\item телефон
	\item Так же клиент может добавлять:
		\begin{itemize}
		\item несколько вариантов телефонов
		\item несколько вариантов адресов
		\item несколько вариантов e-mail, без изменения текущего>>
		\end{itemize}
\end{itemize}
\end{itogo}

\sect{Пункты спецификации, имеющие отношение к требованию}

\clonerequirement{ID.260.1}
\clonerequirement{ID.380}
\clonerequirement{ID.390}


}

\UCsubsubsection{Клиент хочет посмотреть статус по заказам}{HBR.TS.002.11}{

\sect{Исходные данные c Wiki}

\begin{wiki}
Клиент хочет посмотреть статус по всем своим заказам, независимо от того, как был оформлен заказ (on-line, off-line, call)

Клиент переходит на вкладку "Мои заказы" и осуществляет поиск товара, Клиент может выбрать параметр по которому ищется заказ - \sout{по дате, по артикулу товара или} номеру заказа.
\end{wiki}

\begin{teamidea}
Новый функционал. 5-7 человеко-дней
\end{teamidea}

\begin{hybris}
Это не похоже на просмотр статуса по заказам, описан поиск товаров заказанных ранее? Зачем?
\end{hybris}

\begin{tn}
Вернуть в первый релиз
\end{tn}


\sect{Итоговая формулировка User Story}

\begin{itogo}
Клиент хочет посмотреть статус по всем своим заказам, независимо от того, как был оформлен заказ (on-line, off-line, call). Данные о заказах, выполненных вне системы hybris, поставляется в hybris через IMPEX-интеграцию (не входит в проект, в области ответственности заказчика). 

Клиент переходит на вкладку "Мои заказы". Клиент может выбрать параметр по которому ищется заказ - \sout{по дате, по артикулу товара или} номеру заказа.
\end{itogo}


}


\UCsubsubsection{Клиент, как администратор, хочет создавать пользователей, назначать им права, настраивать права пользователей, которых он привязывает к учетной записи}{HBR.TS.002.15}
{

\sect{Исходные данные c Wiki}

\begin{wiki}
HBR.TS.002.15: Клиент переходит на вкладку "Управление аккаунтом" и просматривает права всех пользователей, привязанных к этой учетной записи, а так же задает права созданным пользователям и задает условия привязки для новых пользователей.

HBR.TS.002.16: Клиент переходит на вкладку "Сопровождение пользователей" и создает новых пользователей, заполняет их данные и назначает права.
\end{wiki}

\sect{Пояснения}

\begin{teamidea}
HBR.TS.002.15: Непонятно, что такое <<задает условия привязки для новых пользователей>>

HBR.TS.002.16: Непонятно, почему не обойтись одной вкладкой? Зачем "Сопровождение..." ограничивать только созданием
\end{teamidea}

\sect{Итоговая формулировка User Story}

\begin{itogo}
Клиент переходит на вкладку <<Сопровождение пользователей>> и создает новых пользователей, заполняет их данные и назначает права.

Клиент переходит в раздел <<Управение пользователями>> личного кабинета и создает новых пользователей, заполняет их данные и назначает права. просматривает права всех пользователей, привязанных к этой учетной записи, а так же задает права созданным пользователям.
\end{itogo}

\sect{Пункты спецификации, имеющие отношение к требованию}

\clonerequirement{ID.210.1}
\clonerequirement{ID.560.1}
\clonerequirement{ID.570.1}
\clonerequirement{ID.580.1}
}




\ifcand
\subsection{Кандидаты на последующие этапы}
\UCsubsubsection{Клиент хочет отредактировать свои данные для входа на сайт}{HBR.TS.002.09}{}
\UCsubsubsection{Клиент хочет управлять своими маркетинговыми настройками}{HBR.TS.002.10}{}
\UCsubsubsection{Клиент, как администратор, хочет проводить персонализацию сайта для каждого пользователя}{HBR.TS.002.17}{}
\UCsubsubsection{Клиент хочет управлять настройками метода отгрузок}{HBR.TS.002.18}{}
\UCsubsubsection{Клиент хочет управлять ТК и экспедиторскими службами, с которыми работает.}{HBR.TS.002.19}{}
\UCsubsubsection{Клиент хочет настраивать для себя точки выдачи и отгрузки товара}{HBR.TS.002.20}{}
\UCsubsubsection{Клиент хочет скачать копии счет-фактур и других отгрузочных документов, привязанных к выбранному заказу}{HBR.TS.002.21}{}
\UCsubsubsection{Клиент хочет видеть счет-фактуры всех заказов своей организации.}{HBR.TS.002.22}{}
\UCsubsubsection{Клиент хочет видеть акты-сверки в своем личном кабинете}{HBR.TS.002.23}{}
\UCsubsubsection{Клиент хочет распечатывать список своих корзин, покупок и заказов}{HBR.TS.002.24}{}
\fi

\section{Поиск товара}
\ifcand
\subsection{Согласованные на 1 релиз}
\fi
\UCsubsubsection{Клиент воспользовался быстрым поиском на сайте}{HBR.TS.003.01}
{

\sect{Исходные данные c Wiki}

\begin{wikilong}
	Окно быстрого поиска товаров должно быть доступно на любой странице сайта.
	
	Поиск возможно осуществлять по параметрам:
	\begin{itemize}
		\item Артикул товара (ЕКН);
		\item Полное/частичное наименование товара;
		\item Ключевые слова;
		\item Категория>>
	\end{itemize}
	
	Предусмотрен поиск по нескольким значениям артикулов (указываются через запятую).
	
	В поисковой строке должна содержаться подсказка о том, что требуется ввести (Введите артикул или слово для поиска).
	
	Описание действий:
	\begin{itemize}
	
		\item Поиск по артикулу:
			\begin{enumerate} 
				\item Клиент вводит артикул в окно поиска;
				\item Нажимает Поиск (Search);
				\item Окрывается конкретная карточка товара.
			\end{enumerate}	
			
		\item Поиск по ключевым словам:
		 	\begin{enumerate} 
			 	\item Клиент дает запрос по слову/словосочетанию в окно поиска;
				\item Появляется всплывающий список, подходящий под параметры (категории, бренды) с возможностью выбора и перехода на любой пункт. При этом, если выбор сделан, производится автоматический переход к списку товаров категории либо бренда;
				\item Нажать на Поиск (Search);
				\item Отображаются товары, содержащие ключевые слова в полях, по которым условились выполнять поиск по ключевым словам.
			\end{enumerate}
			
		\item Поиск по названию категории (полное совпадение названия):
		 	\begin{enumerate} 
		 		\item Клиент дает запрос названия категории в окно поиска;
				\item	Появляется всплывающий список, подходящий под параметры с возможностью перехода. При этом, если выбор сделан, производится автоматический переход к списку товаров категории;
				\item	Нажать на Поиск (Search);
				\item	Ниже появляются товары, принадлежащие категории.
			\end{enumerate}	
				
		\item	Поиск по имени бренда (полное совпадение):
		 	\begin{enumerate} 
				\item	Клиент дает запрос имени бренда в окно поиска;
				\item	Появляется всплывающий список, подходящий под параметры с возможностью перехода. При этом, если выбор сделан, производится автоматический переход к списку товаров бренда;
				\item	Нажать на Поиск (Search);
				\item	Ниже появляются товары, принадлежащие бренду.
			\end{enumerate}	
	\end{itemize}
\end{wikilong}

\begin{teamidea}
"Появляется всплывающий список, подходящий под параметры (категории, бренды)" - стандартное поведение акселлератора предполагает отображение в всплывающем списке товаров, а не брендов или категорий. Да и пользоавтели привыкли видеть там дополнение их начатой фразы, а не совсем другие слова. Подтвердите, что верно понимаем задачу.

\textbf{Ответ заказчика: "Если в поиске мы введем Шинглас, то должна отобразиться категория Шинглас, куда клиент сможет "провалиться" и увидеть полный перечень товаров. На Грейнджере реализовано именно так."}

Поиск по категориям является дополнительным функционалом, которого нет в базовом акселлераторе hybris. В качестве результатов поиска будут выдаваться всегда только товары. Другое поведение - на второй этап. Для оценки нужно дополнительное время на изучение (около 1 недели)
\end{teamidea}

\sect{Итоговая формулировка User Story}

\begin{itogolong}
Окно быстрого поиска товаров должно быть доступно на любой странице сайта.

Поиск возможно осуществлять по параметрам:
\begin{itemize}
	\item Артикул товара (ЕКН);
    \item Код (артикул) производителя (ManufacturerCode);		
	\item Полное/частичное наименование товара;
	\item Ключевые слова;
	\item Категория>>
\end{itemize}

Предусмотрен поиск по нескольким значениям артикулов (указываются через запятую).

В поисковой строке должна содержаться подсказка о том, что требуется ввести (Введите артикул или слово для поиска).

Описание действий:
\begin{itemize}

\item Поиск по артикулу:
	\begin{enumerate} 
		\item Клиент вводит артикул в окно поиска;
		\item Нажимает Поиск (Search);
		\item Окрывается конкретная карточка товара.
	\end{enumerate}	
	
\item Поиск по ключевым словам:
 	\begin{enumerate} 
	 	\item Клиент дает запрос по слову/словосочетанию в окно поиска;
		\item Появляется всплывающий список, подходящий под параметры (категории, бренды) с возможностью выбора и перехода на любой пункт. При этом, если выбор сделан, производится автоматический переход к списку товаров категории либо бренда;
		\item Нажать на Поиск (Search);
		\item Отображаются товары, содержащие ключевые слова в полях, по которым условились выполнять поиск по ключевым словам.
	\end{enumerate}
	
\item Поиск по названию категории (полное совпадение названия):
 	\begin{enumerate} 
 		\item Клиент дает запрос названия категории в окно поиска;
		\item	Появляется всплывающий список, подходящий под параметры с возможностью перехода. При этом, если выбор сделан, производится автоматический переход к списку товаров категории;
		\item	Нажать на Поиск (Search);
		\item	Ниже появляются товары, принадлежащие категории.
	\end{enumerate}	
		
\item	Поиск по имени бренда (полное совпадение):
 	\begin{enumerate} 
		\item	Клиент дает запрос имени бренда в окно поиска;
		\item	Появляется всплывающий список, подходящий под параметры с возможностью перехода. При этом, если выбор сделан, производится автоматический переход к списку товаров бренда;
		\item	Нажать на Поиск (Search);
		\item	Ниже появляются товары, принадлежащие бренду.
	\end{enumerate}	
\end{itemize}
\end{itogolong}


\sect{Пункты спецификации, имеющие отношение к требованию}

\clonerequirement{ID.470}
\clonerequirement{ID.810}
\clonerequirement{ID.1000}
\clonerequirement{ID.1010}
 
}



\UCsubsubsection{Клиент хочет ввести ограничения к поиску}{HBR.TS.003.02}
{

\sect{Исходные данные c Wiki}

\begin{wikilong}
	Уточнение поиска возможно при "неконкретном" запросе товара (когда поиск осуществляется по ключевым словам, категориям, брендам). Либо при изначальном поиске (после выбора категории в каталоге товаров).
	
	Дополнительные фильтры появляются слева.
	
	Фильтры делятся по типам (категория, производитель, ценовой диапазон, технические характеристики, поиск по словам в найденном, вариант каталога).
	
	Выводятся только те значения, которые возможны при изначальной фильтрации. Т.е., если изначальные выбор сделан по категории Батареи, то категория 
	
	Абразивы для уточнения появиться не может.
	
	Рядом с каждым возможным выбором указано количество sku.
	
	Предусмотрена возможность просмотра полного списка вариантов по выбранному типу уточнения.
	
	Предусмотрена возможность скрыть (свернуть) тип уточнения.
	
	Описание действий:
	\begin{itemize}
		\item Поиск по словам в найденном:
			\begin{itemize}
				\item Клиент вводит слово/словосочетание в поле слева;
				\item Нажимает Перейти (Go);
				\item Выводится список товаров, которые подходят под изначальный этап фильтрации + введенное слово/словосочетание.
			\end{itemize}
		
		\item Уточнение по категории:
			\begin{itemize}
				\item Слева имеется возможность просмотра более полного перечня предлагаемых категорий;
				\item Производится выбор интересующей категории;
				\item Выводится список товаров;
				\item Происходит автоматическое уменьшение возможных фильтров исходя из отобранных товаров.
			\end{itemize}
		
		\item Уточнение по бренду:
		
			\begin{itemize}
				\item Слева имеется возможность просмотра более полного перечня предлагаемых брендов;
				\item Производится выбор интересующего бренда;
				\item Выводится список товаров;
				\item Происходит автоматическое уменьшение возможных фильтров исходя из отобранных товаров.
			\end{itemize}
		
		\item Уточнение по ценовому диапазону:
		
			\begin{itemize}
				\item Слева имеется возможность просмотра более полного перечня предлагаемых диапазонов;
				\item Производится выбор интересующего диапазона;
				\item Выводится список товаров;
				\item Происходит автоматическое уменьшение возможных фильтров исходя из отобранных товаров.
			\end{itemize}
	
	\end{itemize}
\end{wikilong}

\begin{teamidea}
"Слева имеется возможность просмотра более полного перечня предлагаемых категорий;" - уточнение по категории показывает категории, в которых находятся найденные товары, а не "более полный перечень". Следовательно, должны отображаться категории, сужающие поисковые результаты, список этих категорий является списком категорий, в которые входят найденные товары. Просим подтвердить, правильно ли понимаем задачу.

Ответ ТехноНиколь: Речь об опции View All Categories
\end{teamidea}


\sect{Итоговая формулировка User Story}

\begin{itogolong}
Уточнение поиска возможно при <<неконкретном>> запросе товара (когда поиск осуществляется по ключевым словам, категориям, брендам). Либо при изначальном поиске (после выбора категории в каталоге товаров).

Дополнительные фильтры появляются слева.

Фильтры делятся по типам (категория, производитель, ценовой диапазон, технические характеристики, поиск по словам в найденном, вариант каталога).

Выводятся только те значения, которые возможны при изначальной фильтрации. Т.е., если изначальные выбор сделан по категории Батареи, то категория Абразивы для уточнения появиться не может.

Рядом с каждым возможным элементом фильтра указано количество sku.

Предусмотрена возможность просмотра полного списка вариантов по выбранному типу уточнения.

Предусмотрена возможность скрыть (свернуть) тип уточнения.

Описание действий:
\begin{itemize}
\item Поиск по словам в найденном:
	\begin{itemize}
		\item Клиент вводит слово/словосочетание в поле слева;
		\item Нажимает Перейти (Go);
		\item Выводится список товаров, которые подходят под изначальный этап фильтрации + введенное слово/словосочетание.
	\end{itemize}

\item Уточнение по категории:
	\begin{itemize}
		\item Слева имеется возможность просмотра более полного перечня предлагаемых категорий;
		\item Производится выбор интересующей категории;
		\item Выводится список товаров;
		\item Происходит автоматическое уменьшение возможных фильтров исходя из отобранных товаров.
	\end{itemize}

\item Уточнение по бренду:

	\begin{itemize}
		\item Слева имеется возможность просмотра более полного перечня предлагаемых брендов;
		\item Производится выбор интересующего бренда;
		\item Выводится список товаров;
		\item Происходит автоматическое уменьшение возможных фильтров исходя из отобранных товаров.
	\end{itemize}

\item Уточнение по ценовому диапазону:

	\begin{itemize}
		\item Слева имеется возможность просмотра более полного перечня предлагаемых диапазонов;
		\item Производится выбор интересующего диапазона;
		\item Выводится список товаров;
		\item Происходит автоматическое уменьшение возможных фильтров исходя из отобранных товаров.
	\end{itemize}

\end{itemize}
\end{itogolong}
}



\UCsubsubsection{Клиент хочет сортировать результат поиска}{HBR.TS.003.03}
{
\sect{Исходные данные c Wiki}

\begin{wikilong}
	Возможность сортировки появляется только в том случае, если выведен список товаров.
	\begin{itemize}
		\item Варианты сортировки:
		\item Лидеры продаж;
		\item По бренду А-Я;
		\item По бренду Я-А;
		\item По артикулу (ЕКН): возрастание;
		\item По артикулу (ЕКН): убывание;
		\item По остаткам на складе: возрастание;
		\item По остаткам на складе: убывание;
		\item По цене: возрастание;
		\item По цене: убывание;
		\item По рейтингу (отзывам): возрастание;
		\item По рейтингу (отзывам): убывание.
	\end{itemize}
	Все варианты сортировки находятся в одном выпадающем списке.
	После осуществления сортировки по выбранному параметру возможность повторной сортировки должна оставаться.
	При выборе нового способа сортировки предыдущий выбор должен сбрасываться.
	Сортировка так же доступна в самом списке товаров - при нажатии на название соответствующего столбика в списке товаров.
	Необходимые действия:
	\begin{enumerate}
		\item Сортировка через окно:
		Необходимо сделать выбор типа сортировки, после чего будет выведен список товаров в соответствующем порядке.
		\item Сортировка в списке товаров:
		Нажать на название колонки, по которой должна быть сделана сортировка. При этом, колонки, по которым возможна сортировка, меняют тип курсора (со стрелки на руку). После - формируется список в выбранном порядке.
	\end{enumerate}
\end{wikilong}

\begin{teamidea}

"Возможность сортировки появляется только в том случае, если выведен список товаров." - Результатом поиска является всегда список товаров. Просим подтвердить, правильно ли понимаем задачу.

Ответ Технониколь: На Грейнджере не всегда список товаров является результатом поиска.

По остаткам на складе сортировка возможна только, если эти остатки будут индексироваться. В случае, если остатки изменяются часто извне, и по большому набору товаров сразу, (пере)индексирование большинства товаров может занять 1-2 часа (точные оценки после прототипирования). За это время может прийти обновленная информация по остаткам. Будет расхождение. Рекомендуется убрать из индекса эту сортировку.
С ценой замечания те же, но цены меняются реже наличия.
Вопрос - зачем нужна сортировка по артикулам?
"При выборе нового способа фильтрации предыдущий выбор должен сбрасываться." - непонятное требование

Ответ Технониколь: Речь о том, что сортировка должна выстраиваться заново. Вероятно, это логично. Но на всякий случай указала.

"Сортировка так же доступна в самом списке товаров - при нажатии на название соответствующего столбика в списке товаров." - непонятное требование

Ответ Технониколь: Ниже описано в Необходимые действия

В блоке "Необходимые действия" есть ощущение, что автор путает сортировку с фильтрацией. Фильтрация есть ограничение списка по какому-то критерию. Сортировка - изменение списка с сохранением его длины и состава элементов.

Ответ Технониколь: внесены коррективы в кейс
\end{teamidea}

\begin{hybris}
Согласен, большая тема и нужно на этапе анализа требований понять, что мы будем индексировать и хранить в индексе.

Вопрос Технониколь: Anton Gavazyuk - а какие данные по умолчанию индексируются
\end{hybris}

\sect{Итоговая формулировка User Story}

\begin{itogolong}
Возможность сортировки появляется только в том случае, если выведен список товаров.

Результатом поиска является всегда список товаров. 

Варианты сортировки:
\begin{itemize}
\item Лидеры продаж;
\item По бренду А-Я;
\item По бренду Я-А;
\item По артикулу (ЕКН): возрастание;
\item По артикулу (ЕКН): убывание;
\item По цене: возрастание;
\item По цене: убывание;
\item По рейтингу (отзывам): возрастание;
\item По рейтингу (отзывам): убывание.
\end{itemize}

Сортировка производится по цене, хранящейся в индексе (может отличаться от цены для конкретного покупателя и от цены, изменившейся после последней индексации).

Все варианты сортировки находятся в одном выпадающем списке.

После осуществления сортировки по выбранному параметру возможность повторной сортировки должна оставаться.

При выборе нового способа сортировки предыдущий выбор должен сбрасываться.

Сортировка так же доступна в самом списке товаров - при нажатии на название соответствующего столбика в списке товаров.

Необходимые действия:

Необходимо сделать выбор типа сортировки, после чего будет выведен список товаров в соответствующем порядке.

Сортировка в списке товаров:

Нажать на название колонки, по которой должна быть сделана сортировка. При этом, колонки, по которым возможна сортировка, меняют тип курсора (со стрелки на руку). После - формируется список в выбранном порядке.

\end{itogolong}

}




\UCsubsubsection{Клиент хочет видеть информацию о количестве найденных товаров}{HBR.TS.003.04}
{

\sect{Исходные данные c Wiki}

\begin{wiki}
Количество отобранных товаров всегда отображается над списком товаров.
Так же отображается количество товаров при потенциальном отборе (указано в описании кейса на уточнении поиска).
\end{wiki}

\begin{teamidea}
Количество товаров показывается поисковой системой SOLR, демонстрируем "как есть". Теоретически может отличаться от действительного количества товаров, удовлетворяющих критериям.
\end{teamidea}

\sect{Итоговая формулировка User Story}

\begin{itogo}
Количество отобранных товаров всегда отображается над списком товаров. Также отображается количество товаров при потенциальном отборе
\end{itogo}

\sect{Пункты спецификации, имеющие отношение к требованию}

\clonerequirement{ID.1700}
\clonerequirement{ID.1710}
\clonerequirement{ID.1720}
\clonerequirement{ID.1730}
\clonerequirement{ID.1585.1}

}



\UCsubsubsection{Клиент хочет иметь возможность накладывать дополнительные условия поиска на результат поиска}{HBR.TS.003.05}
{

\sect{Исходные данные c Wiki}

\begin{wiki}
Количество отобранных товаров всегда отображается над списком товаров.
Так же отображается количество товаров при потенциальном отборе (указано в описании кейса на уточнении поиска).
\end{wiki}

\begin{hybris}
Большая тема, если facet search должен учитывать цены завязанные для клиентов, так как по умолчанию в поисковом индексе хранится только одна цена, не специальная для клиента. То есть сейчас возможна ситуация применить фильтр 50-100 рублей, но цена товара для текущего клиента будет не в этом диапазоне.
\end{hybris}

\sect{Итоговая формулировка User Story}

\begin{itogolong}
Данная возможность появляется после того, как установлено ограничение на список товаров.

После того, как устанавливается каждое из дополнительных условий, список таковых должен корректироваться в зависимости от перечня товаров.

Рядом с каждым дополнительным условием отображается количество доступных sku.

Предусмотрена возможность накладывания нескольких дополнительных условий.

Типы дополнительных условий (такие же, как и ограничения):

\begin{itemize}
\item категория, 
\item производитель, 
\item ценовой диапазон\footnote{Будет использована базовая цена, актуальная на момент индексации}, 
\item технические характеристики, 
\item поиск по словам в найденном, 
\item вариант каталога
\end{itemize}

Возможные действия:

\begin{itemize}
\item Установка дополнительного условия
\item Очистить форму дополнительного условия
\item Просмотреть более широкий список условий по предпочитаемому типу
\item Скрыть тип дополнительного условия, при этом установленных выбор не <<слетает>>
\end{itemize}
\end{itogolong}


}



\UCsubsubsection{Клиент хочет выбирать товары из результата поиска и переходить в карточку товара}{HBR.TS.003.06}
{

\sect{Исходные данные c Wiki}

\begin{wiki}
Клиент кликает на название либо Код ТН товара из сформированного списка и "проваливается" в карточку товара
\end{wiki}

\sect{Итоговая формулировка User Story}

\begin{itogo}
Клиент кликает на на название либо артикул (ЕКН) товара из сформированного списка и "проваливается" в карточку товара
\end{itogo}

\sect{Пункты спецификации, имеющие отношение к требованию}

\clonerequirement{ID.740}
\clonerequirement{ID.730}
\clonerequirement{ID.470}



}


\UCsubsubsection{Клиент хочет позвонить менеджеру и проконсультироваться по товару}{HBR.TS.003.07}
{
\sect{Исходные данные c Wiki}

\begin{wiki}
Вариант 1: Клиент в футере сайта обращается к блоку "Есть вопрос? Позвоните:..." и совершает звонок по указанному номеру телефона.
 
Вариант 2: В футере сайта клиент обращается к блоку Поддержка/Заказ отложенного звонка.
 
Заполняет форму обращения и отправляет заявку: цель обращения, контактные данные, вопрос.
\end{wiki}

\begin{teamidea}
<<Форма обращения>> -- новый функционал. Имеет ли смысл делать эту US или лучше сконцетрироваться на том, чтоб у клиента не было вопросов при просмотре товара.
\end{teamidea}

\sect{Итоговая формулировка User Story}

\begin{itogo}
Клиент в футере сайта обращается к блоку "Есть вопрос? Позвоните:..." и совершает звонок по указанному номеру телефона.
\end{itogo}

}
\UCsubsubsection{Клиент хочет позвонить менеджеру, если поиск не выдал ни одного результата}{HBR.TS.003.08}
{
Дублирует HBR.TS.003.07.
}
\UCsubsubsection{Клиент хочет видеть полное или краткое описание товара в результате поиска}{HBR.TS.003.10}{

\sect{Исходные данные c Wiki}

\begin{wiki}
В отфильтрованном списке товаров отображается краткое описание:
 
Полное описание товара клиент может увидеть, перейдя в карточку товара (описано в Карточка товара).
\end{wiki}

\sect{Итоговая формулировка User Story}

\begin{itogo}
В отфильтрованном списке товаров отображается краткое описание товара.  
Полное описание товара клиент может увидеть, перейдя в карточку товара.
\end{itogo}

\sect{Пункты спецификации, имеющие отношение к требованию}

\clonerequirement{ID.470}
\clonerequirement{ID.170}


}

\UCsubsubsection{Клиент хочет увеличить изображение товара}{HBR.TS.003.11}{

\sect{Исходные данные c Wiki}

\begin{wiki}
Доступно только в Карточка товара


\end{wiki}

\sect{Итоговая формулировка User Story}

\begin{itogo}
Дублирует User Story для Карточки товара
\end{itogo}

\sect{Пункты спецификации, имеющие отношение к требованию}

\clonerequirement{ID.470}
\clonerequirement{ID.170}

}

\ifcand
\subsection{Кандидаты на последующие этапы}
\UCsubsubsection{Клиент хочет видеть основные ТХ товаров в результате поиска, чтобы удобно сравнивать их}{HBR.TS.003.09}{}
\fi


\section{Карточка товара}

\subsection{Согласованные на 1 релиз}

\UCsubsubsection{Клиент перешел к карточке товара}{HBR.TS.005.01}{

\sect{Исходные данные c Wiki}

\begin{wikilong}

	Карточка товара, должна отвечать определенным критериям и содержать в себе информацию о: 
	
	Перед информацией о товаре включается текст, с описанием продукта -по своей сути это  SEO текст, с основными ключевиками. 
	\begin{enumerate}
		\item наименование товара
		\item цена авторизованного пользователя. так же непосредственно в карточке должна содержаться информация, что если клиент не авторизовался, ему доступна, только розничная цена товара
		\item бренд товара
		\item идентификатор производителя
		\item ID товара на сайте 
		\item Минимальное количество, от которого производится отгрузка
		\item Минимальное количество, которое доступно для заказа
		\item отгрузочный вес
		\item количество доступное для заказа
		\item страница с товаром в печатном каталоге
		\item страна производитель
		\item количество, требуемое для заказа 
	\end{enumerate}
	
	Далее система должна предложить клиенту оформить заказ, как: 
	
	\begin{itemize}
		\item единичный заказ
		\item добавить товар, в список для системы авто-заказа
	\end{itemize}
	
	(продолжение списка)
	\begin{enumerate}
		\item добавить в корзину
		\item добавить в персональный или общий лист
		\item проверить наличие необходимого количества товара на складе
				\begin{itemize}
					\item   поле для ввода необходимого количества
					\item   поле для ввода идентификатора (индекса)
				\end{itemize}
		\item фото контент 
		\item  автоматически отобразить,доступные для данного товара услуги, и позволять добавлять их в заказ 
		\item  возможность работы с отзывами о товаре
				\begin{itemize}
					\item оставить отзыв
					\item прочитать отзывы других пользователей
					\item воспользоваться сервисом вопрос-ответ
				\end{itemize}
	\end{enumerate}
\end{wikilong}

\begin{teamidea}
Автозаказ возможен только на весь заказ целиком. Поэтому кнопку "Автозаказ" на карточке товара без изменения логики сделать нельзя.

Не предполагаем реализацию на первом этапе страницы товара в печатном каталоге (непонятно)

Решили не показывать количество - не учитываем

Решили не делать мультикорзинность на первом этапе - не учитывали на первом этапе эти требования

Не было требований по услугам - если услуги будут частью каталога и будут связанными товарами - то ок, если нет - то на второй этап

Непонятно, что такое сервис "вопрос-ответ" Не включен в оценку на первый этап

Со всеми дополнениями - на первый этап
\end{teamidea}

\begin{hybris}
Минимальное количество, которое доступно для заказа (Ответ ТехноНИКОЛЬ: это привязано к остатками или к продукту?)

количество доступное для заказа (Ответ ТехноНИКОЛЬ: решили же не показывать)

страница с товаром в печатном каталоге (Ответ ТехноНИКОЛЬ: это что?)

В первом релизе возможно только добавления в текущую корзину (Ответ ТехноНИКОЛЬ: верно?)
\end{hybris}

\begin{tn}
\begin{itemize}
\item <<добавить поле выбора единиц измерения для кол-ва, требуемого для заказа>> (Сорокин Андрей)
\item <<Убрать информацию о том, что цена -- розничная>> (Сорокин Андрей)
\item <<EKN и артикул -- разные поля>> (Сорокин Андрей)
\item <<"отгрузочный вес" заменить на "вес">> (Сорокин Андрей)
\item <<убрать отзывы>> (Сорокин Андрей)
\end{itemize}
\end{tn}

\sect{Итоговая формулировка User Story}

\begin{itogo}
Карточка товара, должна содержать в себе информацию о: 
\begin{itemize}
\item наименование товара,
\item описание товара,
\item цена,
\item статус доступности
\item поле выбора единицы измерения,
\item бренд товара,
\item идентификатор производителя,
\item ЕКН ,
\item Код (артикул) производителя,
\item Минимальное количество, от которого производится отгрузка,
\item Минимальное количество, которое доступно для заказа,
\item вес,
\item страна-производитель,
\item количество, требуемое для заказа.
\item добавить в корзину
\item фотография товара
\end{itemize}
\end{itogo}

\sect{Пункты спецификации, имеющие отношение к требованию}

\clonerequirement{ID.740}
\clonerequirement{ID.750}
\clonerequirement{ID.810}
\clonerequirement{ID.820}
\clonerequirement{ID.850}
\clonerequirement{ID.940}
\clonerequirement{ID.950}
\clonerequirement{ID.990}
\clonerequirement{ID.1000}


}


\UCsubsubsection{Клиент хочет увидеть увеличенное изображение товара}{HBR.TS.005.02}{

	\sect{Исходные данные c Wiki}
	
	\begin{wiki}
	В карточке товара есть фото-контент. Изображение можно увеличить, с помощью специальной кнопки "увеличить изображение"
	картинка, в увеличенном формате, всплывает в дополнительно окне. 
	\end{wiki}
	
	\sect{Итоговая формулировка User Story}
	
	\begin{itogo}
	Изображение товара можно увеличить, с помощью специальной кнопки "увеличить изображение". Картинка, в увеличенном формате, всплывает в дополнительном окне. 
	\end{itogo}

\sect{Пункты спецификации, имеющие отношение к требованию}

\clonerequirement{ID.740}


}

\UCsubsubsection{Клиент хочет увидеть другие изображения товара}{HBR.TS.005.03}{

\sect{Исходные данные c Wiki}

\begin{wiki}
Если изначально системой залито несколько фотографий товара, клиент может пролистать список, при помощи стрелки передвижения вперед-назад
Если товар имеет вариантные товары, клиент может менять изображение по вариантам.
\end{wiki}

\sect{Итоговая формулировка User Story}

\begin{itogo}
Если изначально загружено несколько фотографий товара, клиент может просматривать другие изображения товара. 
Если товар имеет вариантные товары, клиент может менять изображение по вариантам.
\end{itogo}

\sect{Пункты спецификации, имеющие отношение к требованию}

\clonerequirement{ID.980}
\clonerequirement{ID.990}
\clonerequirement{ID.1000}
\clonerequirement{ID.1010}
\clonerequirement{ID.1020}
\clonerequirement{ID.740}


}

\UCsubsubsection{Клиент хочет получить информацию о возможности заказать товар}{HBR.TS.005.04}{

\sect{Исходные данные c Wiki}

\begin{wiki}
Данная информация отображается в карточке товара,  в виде отдельной строки для проверки наличия товара на складе, привязанного к идентификатору-индексу.
\end{wiki}

\begin{teamidea}
Зачем необходима кнопка, если ее функции выполняет "Добавить в корзину"? Проще попробовать добавить и получить отказ (уже нет на складе), чем перед каждым нажатием на "Добавить в корзину" нажимать на "Проверить".

Поскольку интерфейс проверки все равно будет (для корзины), функционал даже в описанном виде можно на первый этап.
\end{teamidea}

\begin{hybris}
Функциональность проверки заказа на складе? Это в каком релизе и с помощью чего?

Ответ Технониколь: В первом через web service
\end{hybris}

\begin{tn}
Информация о возможности заказать товар должна отображаться еще в поиске и корзине.
\end{tn}

\sect{Итоговая формулировка User Story}

\begin{itogo}
Информация о возможности заказать товар отображается в карточке товара, поиске и корзине, в виде отдельной строки для проверки наличия товара на складе, привязанного к идентификатору-индексу.
\end{itogo}

\sect{Пункты спецификации, имеющие отношение к требованию}

\clonerequirement{ID.1720}
\clonerequirement{ID.1730}
\clonerequirement{ID.1585.1}

}

\UCsubsubsection{Клиент хочет получить подробное описание товара}{HBR.TS.005.05}{

\sect{Исходные данные c Wiki}

\begin{wikilong}
	Для того, чтобы получить более подробную информацию о продукте, после ключевых характеристик и данных,предусмотрены закладки с дополнительной информацией
	
	Закладки содержат информацию о:
	
	\begin{itemize}
		\item технические характеристики 
		\item дополнительная информация
		\item меры предосторожности
		\item паспорт безопасности( технический паспорт)
		\item необходимые аксессуары ( товары без которых использование основного товара не возможно-расходники)
		\item дополнительные аксессуары ( товары, которые рекомендуется использовать вместе с основным товаром) 
		\item альтернативные товары (товары, которыми можно заменить основной товар, без потери в характеристиках и качестве). 
		система предлагает сравнить выбранные товары в режиме on-line  по всем ключевым характеристикам, и предоставляет возможность самостоятельной настройки отображения таблицы
		\item запасные части ( см. отображение вложение №4) 
	\end{itemize}
	
	Всю необходимую информацию можно так же выводить списком, вслед за основной информацией, по стандарту 
	\begin{itemize}
		\item запасные части
		\item необходимые аксессуары
		\item дополнительные аксессуары
		\item альтернативные товары
	\end{itemize}
\end{wikilong}

\begin{teamidea}
Откуда меры предосторожности и паспорт безопасности( технический паспорт)?
Этого не было в структуре продукта. За исключением этого - на первый этап
\end{teamidea}

\begin{hybris}
Где будут администрироваться связи между продуктами - кто мастер?
Ответ Технониколь: hybris
\end{hybris}

\sect{Итоговая формулировка User Story}

\begin{itogolong}
Для того, чтобы получить более подробную информацию о продукте, после ключевых характеристик и данных, предусмотрены закладки с дополнительной информацией

Закладки содержат информацию о:
\begin{itemize}
\item технические характеристиках 
\item дополнительной информации
\item необходимых аксессуарах (товары без которых использование основного товара не возможно-расходники)
\item дополнительных аксессуарах ( товары, которые рекомендуется использовать вместе с основным товаром) 
\item альтернативных товарах (товары, которыми можно заменить основной товар, без потери в характеристиках и качестве). 
\item запасных частях
\end{itemize}

Всю необходимую информацию можно также выводить списком, вслед за основной информацией, по стандарту 
\begin{itemize}
\item запасные части
\item необходимые аксессуары
\item дополнительные аксессуары
\item альтернативные товары
\end{itemize}

\end{itogolong}

\sect{Пункты спецификации, имеющие отношение к требованию}

\clonerequirement{ID.760}
\clonerequirement{ID.830}
\clonerequirement{ID.860}


}

\UCsubsubsection{Клиент хочет увидеть похожие товары}{HBR.TS.005.06}{

\sect{Исходные данные c Wiki}

\begin{wiki}
Для реализации этого сервиса служит закладка в карточке товара, которая называется <<альтернативные товары>>, а также при наличии таковых товаров, в теле карточки отображается возможность подобрать аналоги.

Таких товаров может быть неограниченное множество.  (один аналог) или   (два и более аналогов) 

Товары могут быть сравнены в виде таблицы, по ключевым характеристикам.Отображение полей в сравнительной таблице настраивается самим пользователем.
так же доступен для сравнения выбор и с другим товаром, на усмотрение клиента
\end{wiki}

\begin{teamidea}
Без включения функционала сравнения
\end{teamidea}

\begin{hybris}
Где будут администрироваться связи между продуктами - кто мастер?

Ответ ТехноНиколь: На пилоте Hybris, на втором этапе рекомендательный сервис
\end{hybris}

\begin{tn}
Выводить не в закладке, а в отдельном окне под карточкой товара. Закладку переименовать в область.
\end{tn}

\sect{Итоговая формулировка User Story}

\begin{itogo}
Для реализации этого сервиса служит область в карточке товара, которая называется <<альтернативные товары>>, а также при наличии таковых товаров, в теле карточки отображается возможность подобрать аналоги.
\end{itogo}

\sect{Пункты спецификации, имеющие отношение к требованию}

\clonerequirement{ID.770}
\clonerequirement{ID.1000}
\clonerequirement{ID.1550.1}

}
\UCsubsubsection{Клиент хочет увидеть сопутствующие товары}{HBR.TS.005.07}{

\sect{Исходные данные c Wiki}

\begin{wikilong}
	\begin{enumerate}
		\item Аксессуары в карточке товара должны делиться на 
		обязательные (те, без которых использование продукта не возможно)
		желательные (те, использование которых улучшает эффект от использования основного продукта)
		\item  Аксессуары отображаются "закладками" в карточке товара 
		\item  Аксессуары выводятся списком внизу и могут быть добавлены в корзину, в необходимом количестве 
		\item Добавив необходимый или желательный аксессуар в коризну, клиент получает возможность увидеть подборку товаров, которые система так же рекомендует приобрести
		из корзины предоставлена возможность продолжить покупки или перейти к оформлению заказа. Если выбрать один из рекомендованных товаров, система оставляет тебя в корзине,добавив выбранный товар,система перенастраивает список рекомендованных продуктов и выводит аксессуары, к последнему добавленному в корзину товару. 
	\end{enumerate}
\end{wikilong}

\begin{hybris}
Где будут администрироваться связи между продуктами - кто мастер?

Ответ Технониколь: На пилоте Hybris, на втором этапе рекомендательный сервис
\end{hybris}

\begin{tn}
Добавить рекомендуемый товар в карточку
\end{tn}

\sect{Итоговая формулировка User Story}

\begin{itogo}
Необходимо иметь возможность выводить товары, связанные с данным.
Предусмотреть разные типы связей товаров (желательные аксессуары, обязательные аксессуары)

Добавив связанный товар в корзину, клиент получает возможность увидеть в корзине и карточке товара рекомендации, составленные на основе связей товаров из корзины.  
\end{itogo}

\sect{Пункты спецификации, имеющие отношение к требованию}

\clonerequirement{ID.770}
\clonerequirement{ID.1000}
\clonerequirement{ID.1550.1}



}
\UCsubsubsection{Клиент хочет увидеть специальные предложения по товару}{HBR.TS.005.08}{

\sect{Исходные данные c Wiki}

\begin{wiki}
После основной информации, клиенту должна быть отображена информация о доступных к этому товару акциях. 

После того, как клиент положил товар в корзину, и на товар распространяется акция, она должна отобразиться в корзине, как подсказка.

Например, услуга дополнительной гарантии. Клиент может добавить ее в корзину, выбрать ее стоимость (пакет), и ознакомиться с условиями предоставления.
\end{wiki}

\sect{Итоговая формулировка User Story}

\begin{itogo}
Если на товар распространяется акция, посетитель должен иметь возможность получить информацию об этом с карточки товара и из корзины.
\end{itogo}

\sect{Пункты спецификации, имеющие отношение к требованию}

\clonerequirement{ID.1540}
\clonerequirement{ID.1520}
\clonerequirement{ID.1080}


}
\UCsubsubsection{Клиент хочет добавить товар в корзину}{HBR.TS.005.09}{

\sect{Исходные данные c Wiki}

\begin{wiki}
Клиент может из карточки товара добавить товар в корзину.

Товар отображается как добавленный, а клиенту предлагается продолжить покупки или оформить заказ.

При этом клиенту отображается список, рекомендованных к покупке товаров, к тому, что добавлен в корзину. 

При переходе в корзину, клиент видит все товары, которые он добавлял ранее, а если он авторизуется, то увидит и товары прошлых сессий (неделю назад), которые не были доведены до checkout 
\end{wiki}

\begin{teamidea}
Описывается объединение товаров неавторизованного покупателя и корзины авторизованного после прохождения авторизации. hybris работает не так. Проговаривали вариант, когда в случае, если корзина заполнена, после авторизации она остается с тем же набором товаров. В оценку включено именно такое поведение системы на первый этап, так как это стандартное поведение акселлератора
\end{teamidea}

\begin{hybris}
Где будут администрироваться связи между продуктами - кто мастер?

Ответ ТехноНиколь: На пилоте Hybris, на втором этапе рекомендательный сервис
\end{hybris}

\sect{Итоговая формулировка User Story}

\begin{itogo}
Клиент может из карточки товара добавить товар в корзину.

Товар отображается как добавленный, а клиенту предлагается продолжить покупки или оформить заказ.

При этом клиенту в корзине отображается список, рекомендованных к покупке товаров, к тому, что добавлен в корзину. 

\end{itogo}

\sect{Пункты спецификации, имеющие отношение к требованию}

\clonerequirement{ID.1540}
\clonerequirement{ID.1520}
\clonerequirement{ID.1550.1}
\clonerequirement{ID.1530}
\clonerequirement{ID.1510.1}

}
\UCsubsubsection{Клиент хочет видеть стоимость товара в зависимости от своей роли и согласно с заключенным договором на поставку товаров}{HBR.TS.005.10}{

\sect{Исходные данные c Wiki}

\begin{wiki}
Что бы клиенту отображались цены, согласно его условиям, в каждой карточке отображается информация, с просьбой авторизоваться, что бы это стало возможно. 
\end{wiki}

\begin{tn}
Не выводить информацию о необходимости идентификации
\end{tn}

\sect{Итоговая формулировка User Story}

\begin{itogo}
Разные пользователи могут иметь разные цены.  %На карточке товара для неаутентифицированных пользователей должна отображаться информация о том, что нужно пройти аутентификацию. 
\end{itogo}

\sect{Пункты спецификации, имеющие отношение к требованию}

\clonerequirement{ID.1040}
\clonerequirement{ID.1050}
\clonerequirement{ID.1060}


}


\section{Корзина}
\ifcand
\subsection{Согласованные на 1 релиз}
\fi
\UCsubsubsection{Клиент хочет добавить в корзину выбранные им товары}{HBR.TS.006.01}
{

\sect{Исходные данные с Wiki}

\begin{wiki}
\begin{itemize}
\item Шаг 1. Находим нужный товар в каталоге Продуктов
\item Шаг 2. Вносим в окне "Кол-во", кол-во товара к заказу
\item Шаг 3. Нажимаем кнопку "В корзину"
\end{itemize}

Находясь в продуктовом каталоге клиент указывает количество товара для заказа и нажимает на кнопку "Добавить в корзину". 

Клиент попадает на форму в которой ему предлагается добавить в корзину товары рекомендуемые к закупке вместе с тем товаром который добавляется в корзину, а также он может выбрать дальнейшие действия и либо: продолжить выбор товаров (ссылка "Продолжить добавление" или кнопка "Оформить заказ")
\end{wiki}

\begin{hybris}
Процесс Отличается от стандартной процедура добавления товара в корзину, рекомендую исключить форму с рекомендованными товарами из процесса покупки - не думаете что будет раздражать пользователя?

Ответ Технониколь: Кейс работает в e-commerce.
\end{hybris}

\sect{Итоговая формулировка User Story}

\begin{itogo}
Шаг 1. Находим нужный товар в каталоге Продуктов
Шаг 2. Вносим в окне "Кол-во", кол-во товара к заказу
Шаг 3. Нажимаем кнопку "В корзину"
\end{itogo}

\sect{Пункты спецификации, имеющие отношение к требованию}

\clonerequirement{ID.1520}
\clonerequirement{ID.1530}

}
\UCsubsubsection{Заказ рекомендованного товара в корзине}{HBR.TS.006.02}
{
\sect{Исходные данные с Wiki}

\begin{wiki}
Шаг 1. Нажимаем ссылку "Моя корзина".
Шаг 2. Клиент вводит кол-во товара к заказу в окне "Кол-во" выбранного товара.
Шаг 3. Клиент нажимает кнопку "В корзину" и товар попадает в список товаров в корзине.

Находясь в корзине клиент видит в нижней ее части список товаров которые доступны к заказу в качестве Дополнительных к выбранным им товарам.
\end{wiki}

\begin{hybris}
Где мастер ассоциаций между товарами?

Ответ Технониколь: Excel. На пилоте hybris, на втором этапе рекомендательный сервис
\end{hybris}

\sect{Итоговая формулировка User Story}

\begin{itogo}
Находясь в корзине клиент видит в нижней ее части список товаров которые доступны к заказу в качестве рекомендованных к выбранным им товарам.
\end{itogo}

\sect{Пункты спецификации, имеющие отношение к требованию}

\clonerequirement{ID.1550.1}


}
\UCsubsubsection{Редактирование кол-ва товара добавленного в корзину}{HBR.TS.006.04}{
\sect{Исходные данные с Wiki}

\begin{wiki}
Шаг 1. Нажимаем ссылку "Моя корзина".
Шаг 2. Меняем количество товара в окне "Кол-во".
Шаг 3. Нажимаем кнопку "Редактировать".
\end{wiki}

На форме корзины клиент может изменить кол-во заказанного товара

\sect{Итоговая формулировка User Story}

\begin{itogo}
Шаг 1. Нажимаем ссылку "Моя корзина".
Шаг 2. Меняем количество товара в окне "Кол-во" напротив товара в корзине.
Шаг 3. Нажимаем кнопку "Редактировать".

На форме корзины клиент может изменить кол-во заказанного товара.

\end{itogo}

\sect{Пункты спецификации, имеющие отношение к требованию}

\clonerequirement{ID.1550.1}

}



\UCsubsubsection{Удаление товара добавленного в корзину}{HBR.TS.006.05}{

\sect{Исходные данные с Wiki}

\begin{wiki}
Шаг 1. Нажимаем ссылку "Моя корзина"
Шаг 2. Меняем количество товара в окне "Кол-во" на "0"
Шаг 3. Нажимаем кнопку "Редактировать"

В форме корзины клиент может удалить строку добавленного товара
\end{wiki}

\sect{Итоговая формулировка User Story}

\begin{itogo}
В форме корзины клиент может удалить строку добавленного товара через кнопку "Удалить" или установкой количества товара в ноль.
\end{itogo}

\sect{Пункты спецификации, имеющие отношение к требованию}

\clonerequirement{ID.1550.1}

}
\UCsubsubsection{Клиент хочет, чтобы неподтвержденная корзина сохранялась и при следующем входе на сайт}{HBR.TS.006.13}{

\sect{Исходные данные с Wiki}

\begin{wiki}
Шаг 1. Клиент входит на любую страницу сайта. 
Шаг 2. Клиент видит в правом верхнем углу ссылку "Моя корзина" с кол-вом строк в ней

при входе на сайт клиент должен видеть количество товаров в корзине в верхней правой части экрана.
\end{wiki}




\begin{teamidea}
Будет конфликт с требованием сохранять корзину при авторизации.

Если аноним положил 3 товара в корзину, потом авторизовался под Ивановым, который вчера бросил корзину с другими 2 товарами, то есть три варианта действий:
\begin{enumerate}
\item очистить корзину анонима (3) и заменить ее корзиной Иванова (2). 
\item оставить корзину анонима (3), про корзину Иванова забыть (Вариант "как в Хайбрисе")
\item объединить корзины (3+2). При разлогинивании оставить только анонимную (3). Сложный вариант.
\end{enumerate}
\end{teamidea}

\begin{hybris}
в стандартном B2B accelerator'e если у анонима есть товар в корзине, потом он входит на сайт под своим именем - корзина авторизированного пользователя заменяется корзиной анонима. Надо определиться с логикой этих замен корзин или их слиянием

Ответ Технониколь: Меня вот это тоже взволновало, есть best practice по слиянию? 
\end{hybris}

\sect{Итоговая формулировка User Story}

\begin{itogo}
Шаг 1. Клиент входит на любую страницу сайта. 
Шаг 2. Клиент видит в области "Корзина" ссылку "Моя корзина" с количеством товаров, отложенных в корзину.

При открытии любой страницы сайта клиент должен видеть количество товаров в корзине в соответствующей области.
\end{itogo}

}


\UCsubsubsection{Клиент хочет начать оформление заказа воспользовавшись функцией корзины Checkout }{HBR.TS.006.19}
{

\sect{Исходные данные с Wiki}

\begin{wikilong}
\begin{itemize}
	\item Шаг 1. после нажатия кнопки Checkout клиент попадает в новую форму в которой ему предлагается выбор "Типа доставки" и "Способа оплаты"
	\item Шаг 2. После выбора типа "Доставка" появляется окно для ввода дополнительной информации по доставке
	\item Шаг 3. у клиента есть возможность выставить галку и сделать такой метод доставки методом по умолчанию
	\item Шаг 4. у клиента есть возможность установить галкой необходимость ожидания полной комплектации заказа перед его доставкой
	\item Шаг 5. у клиента есть возможность выбрать один из сохраненных адресов или ввести новый
	\item Шаг 6. у клиента есть возможность внести (и или откоректировать персональные данные, страну, адрес, индекс, город, контактный телефон, контактный адрес электронной почты, и название компании в текстовых полях или оставить в них значения по умолчанию.
	\item Шаг 7. у клиента есть возможность сохранить введенную информацию как один из своих адресов доставки
	\item Шаг 8. клиент может выбрать из списка оператора по доставке
	\item Шаг 9. клиент может указать будет ли он платить за доставку отдельно оператору или хочет чтобы поставщик включил стоимость доставки в счет
	\item Шаг 10. клиент может ввести данные для наклейке к грузу (упаковочного листа) указав детальную информацию по доставке, свою должность, номер почтового ящика, номер проекта, дополнительные номера для идентификации груза
	\item Шаг 11. после ввода данных клиент может продолжить чекаут корзхины нажав кнопку Сохранить, данные по доставке сохраняются
	\item Шаг 12. клиент имеет возможность выбрать один из доступных способов оплаты (Master, Visa, Amex)
	\item Шаг 13. клиент может указать свои платежные реквизиты, имя фамилию, номер карты, дата ее окончания, комментарий, сохранить данные о карте в качестве одного из методов оплаты по умолчанию
	\item Шаг 14. клиент может выбрать опцию и указать свои координаты доставки как адрес доставки документов или ввести новые (Фамилию, Имя, Компанию, Адрес 1, Адрес 2, Город, Штат, Индекс, телефон и факс)
	\item Шаг 15. после нажатия на кнопку "Сохранить" информация о платежных реквизитах сохранится
	\item Шаг 16. клиент имеет возможность завершить оформление заказа нажав кнопку "Продолжить"
\end{itemize}

Клиент после завершения подборра товара в корзину может ввести адрес доставки, указать способ оплаты и подтвердить заказ к исполнению.
\end{wikilong}

\sect{Пояснения}

\begin{teamidea}
Описанный процесс отличается от стандартного B2B Checkout. 
Описанный процесс требует сертификации по PCI DSS.
Описанный процесс требует интеграции с процессинговым центром.
Многое в описанном непонятно.
Слишком много шагов. Настолько длинный механизм заказа не встречается ни на одном сайте - редко кто доберется до конца.
Оператор доставки = метод доставки? Другого способа не предусмотрено.
\end{teamidea}

\sect{Итоговая формулировка User Story}

\begin{itogolong}
 Из корзины неаутентифицированные пользователи по нажатию на "Оформление заказа" попадают на форму аутентификации (т.к. требуется быть аутентифицированным для совершения заказа). После аутентификации пользователь попадает на страницу оформления заказа.
\begin{itemize}
\item Выбор метода платежа (счет, кредит)
\item В случае выбора "счет" доступен блок "Выбор Cost Center".
\item В случае выбора подразделения (Cost Center) и выбора метода платежа доступен блок "Адрес доставки"
\item В случае отсутствия адреса доставки предлагается ввести адрес доставки и привязать в дальнейшем к пользователю
\item В случае наличия нескольких адресов доставки предлагается выбрать нужный
\item В случае выбора адреса доставки, предлагается выбрать метод доставки
\item После указания адреса доставки, метода доставки, подразделения (Cost Center) пользователь может
\begin{itemize}
\item указать, что заказ является заказом по расписанию, после подтвердить заказ
\item сделать запрос на предложение и подтвердить заказ
\item подтвердить заказ
\end{itemize}
%\item В случае выбора заказа по расписанию, пользователь может ввести
%\begin{itemize}
%\ item Дату начала заказов по расписанию - дату первого заказа
%\item Расписание регулярности заказов, а именно одно из
%\begin{enumerate}
%\item Через указанное число дней
%\item В указанный день недели через указанное число недель
%\item В указанный день месяца
%\end{enumerate}
%\end{itemize}
\end{itemize}
Страница оформления заказа. Включает:
В случае выбора запроса на предложение - ввести в свободной форме комментарий и выполнить запрос
\end{itogolong}

\sect{Пункты спецификации, имеющие отношение к требованию}

\clonerequirement{ID.1560}
\clonerequirement{ID.1585.1}

}
\ifcand
\subsection{Кандидаты на последующие этапы}
\UCsubsubsection{Быстрое добавление товара в корзину}{HBR.TS.006.06}{}
\UCsubsubsection{Проверка доступности товара на складе ближайшем к месту доставки}{HBR.TS.006.07}{}
\UCsubsubsection{Клиент может проверить доступность товара набранного в корзину на ближайшем складе после ввода индекса}{HBR.TS.006.09}{}
\fi

\section{Заказ}
\ifcand
\subsection{Согласованные на 1 релиз}
\fi
%\UCsubsubsection{Клиент хочет подтвердить корзину и перейти к оформлению заказа}{HBR.TS.007.01}{
%
%\sect{Исходные данные с Wiki}
%
%\begin{wiki}
%Клиент находясь в корзине может её подтвердить к заказу и перейти к оформлению заказа, нажав на кнопку "Оформить заказ".
%В интерфейсы корзины есть возможность вернуться к подбору покупок. Есть информационный блок с контактной информацией центра поддержки.
%\end{wiki}
%
%\begin{hybris}
%"Превращение корзины в заявку" - что это за процесс и чем он отличается от процесса заказа?
%\end{hybris}
%
%\sect{Итоговая формулировка User Story}
%
%\begin{itogo}
%Клиент, находясь в корзине, может её подтвердить к заказу и перейти к оформлению заказа, нажав на кнопку "Оформить заказ".
%В интерфейсе корзины есть возможность вернуться к подбору покупок. Есть информационный блок с контактной информацией центра поддержки.
%\end{itogo}
%
%\sect{Пункты спецификации, имеющие отношение к требованию}
%
%\clonerequirement{ID.1560}
%\clonerequirement{ID.1570}
%
%}
\UCsubsubsection{Клиент хочет видеть пошаговый механизм оформления заказа}{HBR.TS.007.03}{

\sect{Исходные данные с Wiki}

\begin{wiki}
Клиент находясь во всех интерфейсах оформления заказа начиная с страницы корзины видит все предыдущие и последующие этапы оформления, и понимает на каком из этапов он находится в настоящий момент.
 
Предлагаю рассмотреть альтернативы:
Hybris: аутентификация - выбор способа оплаты - выбор адреса - выбор метода доставки
Grainger: аутентификация - выбор адреса - выбор метода доставки - выбор способа оплаты - summary заказа
Home Depot: аутентификация - выбор адреса - выбор метода доставки - выбор способа оплаты - summary заказа
\end{wiki}

\begin{teamidea}
У hybris тоже есть summary. Пока в оценку положен стандартный механизм.
\end{teamidea}

\sect{Итоговая формулировка User Story}

\begin{itogo}
Клиент, находясь во всех интерфейсах оформления заказа, начиная с страницы корзины, видит все предыдущие и последующие этапы оформления, и понимает на каком из этапов он находится в настоящий момент.
\end{itogo}

\sect{Пункты спецификации, имеющие отношение к требованию}

\clonerequirement{ID.1600}

}
\UCsubsubsection{Клиент хочет получать подсказки о назначении элементов интерфейса в контекстном меню}{HBR.TS.007.04}{

\sect{Исходные данные с Wiki}

\begin{wiki}
Клиент в процессе оформления заказа хочет понимать, какой значок/термин, что значит, например:
\begin{enumerate}
\item Статус доступности
\item Внутренний номер заказа клиента
\end{enumerate}
\end{wiki}

\begin{hybris}
Подсказки нужны только в случае идентификации "непонятных" для пользователя элементов, старайтесь не перегружать UI ненужными элементами
\end{hybris}

\sect{Итоговая формулировка User Story}

\begin{itogo}
Клиент в процессе оформления заказа хочет понимать, какой значок/термин, что значит, например:
\begin{itemize}
\item Статус доступности
\item Внутренний номер заказа клиента
\end{itemize}

\end{itogo}
}
\UCsubsubsection{Клиент хочет выбирать тип оплаты}{HBR.TS.007.05}{

\sect{Исходные данные с Wiki}

\begin{wiki}
После подтверждения корзины клиент переходит в интерфейс выбора способа оплаты. Доступные способы оплаты:
	\begin{itemize}
		\item Предоплата
		\item Кредит, может подать заявку на кредитный лимит
		\item Оплата картой
	\end{itemize}
\end{wiki}

\begin{teamidea}
В оценку заложен payonline или Cybersource как шлюз для карточек.
Все прочее нужно прорабатывать и это точно займет больше времени.
Оплата картой, кстати, тоже предоплата. Первый пункт лучше переименовать в "Оплата по счету".
\end{teamidea}

\begin{hybris}
В первых релизах не будет оплаты картой, соответствующая интеграция не нужна. Совет: разделите этот US на несколько согласно кол-ву методов оплаты. Например: покупатель имеет возможность оплатить заказ с помощью платежной карты во время Checkout. Для этого он должен ввести такие данные и провести определенные действия.
\end{hybris}

\sect{Итоговая формулировка User Story}

\begin{itogo}
После подтверждения корзины клиент переходит в интерфейс выбора способа оплаты. Доступные способы оплаты:
	\begin{itemize}
		\item Оплата по счету
		\item Кредитный лимит
	\end{itemize} 
\end{itogo}

\sect{Пункты спецификации, имеющие отношение к требованию}

\clonerequirement{ID.430.1}

}
\UCsubsubsection{Клиент хочет выбирать юридическое лицо для конкретного заказа}{HBR.TS.007.06}{

\sect{Исходные данные с Wiki}

\begin{wiki}
После выбора желаемого способа оплаты клиент выбирает юридической лицо, от лица которого он хочет сделать заказ
\end{wiki}

\begin{teamidea}
Предполагается, что юрлица будут храниться в структуре Cost Center. Для пользоавтеля это будет незаметно.
\end{teamidea}

\begin{hybris}
Было определенно что Юрлица - B2BUnit не "хранятся" в Cost Centre а наоборот - к B2B Unit подключается набор (множество) Cost centres. Пожтому в стандартном функционале пользователь видит в списке список доступных CC в B2B Unit
\end{hybris}

\begin{tn}
<<Убрать payonline из ограничения способов оплаты>>, <<Поменять центр финансовых затрат на подразделение>>
\end{tn}

\sect{Итоговая формулировка User Story}

\begin{itogo}
После выбора способа оплаты клиент выбирает подразделение (Cost Center) 
\end{itogo}

\sect{Пункты спецификации, имеющие отношение к требованию}


}
\UCsubsubsection{Клиент хочет выбирать способ отгрузки из предлагаемого списка способов доставки}{HBR.TS.007.08}{

\sect{Исходные данные с Wiki}

\begin{wikilong}
	После подтверждения корзины (checkout) клиенту предлагается оформить доставку.
	
	Способ доставки/метод доставки определяется автоматически в зависимости от стратегии обеспечения данного SKU и от ВГХ SKU (например, если у нас тяжелые грузы, то мы предлагаем только самовывоз). Ответственный сотрудник за ведение номенклатуры (SKU) в ERP должен иметь возможность выбирать вид доставки и фиксировать его для этого SKU
	
	Возможные способы доставки: 
	\begin{itemize}
	\item доставка
	\item самовывоз с указанием доступных точек выдачи заказов в данном регионе
	\item экспорт
	\end{itemize}
	
	Данный список может расширяться при необходимости.
	
	Если клиент выбирает способ "доставка", то он должен точно идентифицировать адрес доставки из существующего списка адресов доставки аккаунта, либо добавить новый адрес доставки на шаге оформления доставки. При введении нового адреса доставки клиент заполняет карточку адреса доставки, где обязательно указывает поле " эл.адрес", который должен добавляться в список контактов в учетной записи для дальнейшей коммуникации по статусу заказа. Если клиент хочет сохранить новый адрес доставки, то он должен поставить соответствующий флаг, при наличии этого флага система автоматически формирует еще один контакт в аккаунте. Клиенту должна предоставляться возможность выбора Перевозчика ( shipping method). Список перевозчиков для клиента уже определен со стоимостью доставки. Клиент выбирает оптимального для себя по стоимости и по времени доставки перевозчика.
	
	Если клиент выбирает способ доставки "самовывоз", то ему предоставляется возможность по введенному индексу желаемого места вывоза товара или по адресу увидеть видеть список доступных пунктов выдачи заказов (ПВЗ). Клиент выбирает удобный для него адрес самовывоза. При этом он должен иметь возможность видеть на карте локацию всех ПВЗ ( с контактной информацией, графиком работы и прочее) и свою локацию по отношению к ПВЗ. Возможность выбора ПВЗ должна быть и с помощью выбора ПВЗ на карте.
	
	Если клиент выбирает способ доставки "доставка в другой регион/экспорт" (точно не на первом релизе), то ему необходимо заполнить адрес доставки по аналогии с "доставкой". Клиенту должна предоставляться возможность выбора Перевозчика ( shipping method). Список перевозчиков для клиента уже определен со стоимостью доставки. Клиент выбирает оптимального для себя по стоимости и по времени доставки перевозчика.
	
	Клиенту должна быть предоставлена возможность установки флага "отгрузка всего заказа целиком/отгрузка заказа частями". При установки данного флага - доставка/отгрузка заказа клиенту осуществляется всего заказа целиком.
\end{wikilong}

\begin{teamidea}
То есть, каждый из товаров будет иметь список допустимых видов доставки? Может, стоит ограничиться группами? При оценке предполагаем, что в товарах информация о допустимых доставках является опциональной - если ее нет, то допустимы все доставки. Если она есть, то она содержит перечень допустимых доставок. Вопрос - что делать, когда добавится еще одна доставка - изменять массово все товары?

Непонятно, что означает "Список перевозчиков для клиента уже определен со стоимостью доставки."

Интерфейс просмотра доступных пунктов выдачи заказов по индексу или адресу - плюс 1 день на индекс.
В оценку не включен способ доставки "в другой регион/экспорт"
\end{teamidea}

\begin{hybris}
Согласен, US не проработано с точки зрения "как будет администрироваться и кем" привязка способа доставки к SKU/группе SKU.
\end{hybris}

%\begin{tn}

%описать подробнее как будет оформляться доставка, сейчас из спецификации не понятно

%Бабаджанов: "Функция выбора способа/метода доставки, предоставление клиенту сценария оформления доставки в зависимости от выбранного способа доставки, предоставление клиенту возможности выбора перевозчика на доставку, возможность при оформлении доставки указать новый адрес доставки, интерактивная возможность в виде карты для клиента выбрать удобный пункт выдачи заказов"
%\end{tn}

\sect{Итоговая формулировка User Story}

\begin{itogo}
После подтверждения корзины (checkout) клиенту предлагается оформить доставку.
\end{itogo}

\sect{Пункты спецификации, имеющие отношение к требованию}

\clonerequirement{ID.1540}
\clonerequirement{ID.1550.1}

}
\UCsubsubsection{Клиент хочет изменить дату доставки на последнем шаге оформления заказа (финальная страница order review)}{HBR.TS.007.13}{

\sect{Исходные данные с Wiki}

\begin{wiki}
Визуально вся последовательность шагов должна быть пронумерована и выстроена в логическом порядке (мастер оформления), должна быть навигация по мастеру и возможность кликом возвращаться на нужный шаг, дублирование клика - стрелки "назад-вперед".
\end{wiki}

\begin{hybris}
Данные могут не сохраняться - например при оплате картой система не должна хранить информацию о введенной карте, так как это нарушение безопасности. Нужно специфицировать данные которые сохраняются
\end{hybris}

\sect{Итоговая формулировка User Story}

\begin{itogo}
Визуально вся последовательность шагов должна быть пронумерована и выстроена в логическом порядке (мастер оформления), должна быть навигация по мастеру и возможность кликом возвращаться на нужный шаг, дублирование клика - стрелки "назад-вперед".
\end{itogo}

}
\UCsubsubsection{Клиент хочет распечатать спецификацию на этапе корзины (текущей или сохранённой)}{HBR.TS.007.15}{

\sect{Исходные данные с Wiki}

\begin{wiki}
Клиент может распечатать текущую или сохранённую корзину. Нажав на кнопку печать заказа он видит интерфейс с печатной формой, которую он может отправить сразу на печать или выслать на e-mail
\end{wiki}

\sect{Итоговая формулировка User Story}

\begin{itogo}
Клиент может распечатать текущую или сохранённую корзину. Нажав на кнопку печать заказа он видит интерфейс с печатной формой, которую он может отправить сразу на печать или выслать на e-mail
\end{itogo}

}
\UCsubsubsection{Клиент хочет видеть подсказки по оформлению заказа}{HBR.TS.007.16}{

\sect{Исходные данные с Wiki}

\begin{wiki}
Клиент в процессе оформления заказа видит подсказки под каждым функциональным блоком страницы или отдельной функции. Подсказки ненавязчивые, но на них в любой момент можно посмотреть и сориентироваться
\end{wiki}

\begin{hybris}
Опять же подсказки нужны в интерфейсе который изначально не может быть понят пользователем - старайтесь структурировать Checkout процесс наиболее удобным для пользователя способом, чтоб у него не возникало вопросов.
\end{hybris}

\sect{Итоговая формулировка User Story}

\begin{itogo}
Клиент в процессе оформления заказа видит подсказки под каждым функциональным блоком страницы или отдельной функции. Подсказки ненавязчивые, но на них в любой момент можно посмотреть и сориентироваться
\end{itogo}

}
\UCsubsubsection{Клиент хочет видеть сообщения о неудачной обработке заказа и причину}{HBR.TS.007.17}{

\sect{Исходные данные с Wiki}

\begin{wiki}
На любом из этапов оформления заказа в случае сбоя (системная ошибка, ошибка соединения, неправильно заполненнное поле), отсутствие подтверждения со стороны ERP и пр. клиент видит сообщение, что что-то пошло не так, причину, и что следует предпринять
\end{wiki}

\begin{hybris}
Что следует предпринять - это всегда звонок в Call Centre
\end{hybris}

\sect{Итоговая формулировка User Story}

\begin{itogo}
На любом из этапов оформления заказа в случае сбоя (системная ошибка, ошибка соединения, неправильно заполненнное поле), отсутствие подтверждения со стороны ERP и пр. клиент видит сообщение, что что-то пошло не так, причину, и что следует предпринять
\end{itogo}

\sect{Пункты спецификации, имеющие отношение к требованию}

\clonerequirement{ID.1650}
\clonerequirement{ID.1660}

}
\UCsubsubsection{Клиент хочет связаться со специалистом call-центра для получения помощи}{HBR.TS.007.18}{

\sect{Исходные данные с Wiki}

\begin{wiki}
Клиент может связаться с службой поддержки из любого окна процесса оформления заказа:
\begin{itemize}
\item по чату
\item по телефону
\end{itemize}
\end{wiki}

\begin{teamidea}
С телефоном понятно, что сразу, с чатом - на будущие этапы. Или внешний сервис (тогда первый)
Настоятельно рекомендуется внешний сервис
\end{teamidea}

\sect{Итоговая формулировка User Story}

\begin{itogo}
Клиент может связаться с службой поддержки из любого окна процесса оформления заказа по телефону.
\end{itogo}


}
\UCsubsubsection{Клиент хочет поменять тип оплаты}{HBR.TS.007.19}{

\sect{Исходные данные с Wiki}

\begin{wiki}
Клиент имеет возможность вернуться к выбору типа оплаты на любом этапе оформления заказа до момента подтверждения заказа
\end{wiki}

\sect{Итоговая формулировка User Story}

\begin{itogo}
Клиент имеет возможность вернуться к выбору типа оплаты на любом этапе оформления заказа до момента подтверждения заказа
\end{itogo}

\sect{Пункты спецификации, имеющие отношение к требованию}

\clonerequirement{ID.430.1}
\clonerequirement{ID.1550.1}

}
\UCsubsubsection{Клиент хочет иметь возможность разместить свой заказ}{HBR.TS.007.22}{

\sect{Исходные данные с Wiki}

\begin{wiki}
После прохождения всех этапов оформления заказа клиент видит результат всех предыдущих этапов и "подтверждает заказ"
\end{wiki}

\sect{Итоговая формулировка User Story}

\begin{itogo}
После прохождения всех этапов оформления заказа клиент видит результат всех предыдущих этапов и "подтверждает заказ"
\end{itogo}

\sect{Пункты спецификации, имеющие отношение к требованию}

\clonerequirement{ID.1570}
\clonerequirement{ID.1585.1}
\clonerequirement{ID.1640}
\clonerequirement{ID.1650}
\clonerequirement{ID.1660}

}
\UCsubsubsection{Клиент хочет получать уведомление на свой e-mail о том, что он успешно разместил заказ}{HBR.TS.007.25}{

\sect{Исходные данные с Wiki}

\begin{wiki}
После размещения заказа клиент получает на почту уведомление с подтверждением заказа и основными данными по заказу
\end{wiki}


\sect{Итоговая формулировка User Story}

\begin{itogo}
После размещения заказа клиент получает на почту уведомление с подтверждением заказа и основными данными по заказу
\end{itogo}

}
\UCsubsubsection{Клиент хочет получить информацию о том, что заказ не удалось разместить и причину неудачи}{HBR.TS.007.26}{

\sect{Исходные данные с Wiki}

\begin{wiki}
Клиент видит результат операции: прошла успешно / не прошла: причина, действия
\end{wiki}

\sect{Итоговая формулировка User Story}

\begin{itogo}
Клиент видит результат заказа - <<заказ размещен>> или <<возникла проблема с размещением заказа>>. 
\end{itogo}

\sect{Пункты спецификации, имеющие отношение к требованию}

\clonerequirement{ID.1585.1}
\clonerequirement{ID.1640}
\clonerequirement{ID.1650}
\clonerequirement{ID.1660}

}
\UCsubsubsection{Клиент хочет, чтобы с ними связался менеджер, если заказ не удалось разместить}{HBR.TS.007.27}{

\sect{Исходные данные с Wiki}

\begin{wiki}
Если клиент видит сообщение об ошибке в размещении заказа клиент должен видеть указанный КРУПНЫМИ БУКВАМИ номер центра поддержки 
\end{wiki}

\begin{tn}
Исправить КРУПНЫМИ БУКВАМИ на КРУПНЫМИ ЦИФРАМИ.
\end{tn}

\sect{Итоговая формулировка User Story}

\begin{itogo}
Если клиент видит сообщение об ошибке в размещении заказа клиент должен видеть номер центра поддержки КРУПНЫМИ ЦИФРАМИ
\end{itogo}

}
\UCsubsubsection{Клиент хочет связаться с менеджером, если у него возникли вопросы по размещению заказа}{HBR.TS.007.28}{

\sect{Исходные данные с Wiki}

\begin{wiki}
Клиент может связаться с службой поддержки из любого окна процесса оформления заказа:
\begin{itemize}
\item по чату
\item по телефону
\end{itemize}
\end{wiki}

\begin{teamidea}
По чату такие же вопросы как для одной из предыдущих US - будет ли реализация на первом этапе и тд
\end{teamidea}

\sect{Итоговая формулировка User Story}

\begin{itogo}
Клиент может связаться с службой поддержки из любого окна процесса оформления заказа по телефону 
\end{itogo}
}
\UCsubsubsection{Клиент хочет регулярно получать одни и те же товары, без дополнительной процедуры оформления заказа}{HBR.TS.007.29}{

\sect{Исходные данные с Wiki}

\begin{wiki}
Клиент на этапе оформления заказа может поставить отметку автозаказ на весь заказ целиком
\end{wiki}

\begin{hybris}
Автозаказ - это заказ по расписанию/повторяющийся заказ. Стандартно заказ не имеет отметку автозаказа на каждый элемент. Не имеет смысла ставить эту отметку потому как вы даете возможность нагенерировать множество заказов с одной позицией.
\end{hybris}

\sect{Итоговая формулировка User Story}

\begin{itogo}
Клиент на этапе оформления заказа может запланировать заказ текущей корзины по расписанию 
\end{itogo}

\sect{Пункты спецификации, имеющие отношение к требованию}

\clonerequirement{ID.1600}
\clonerequirement{ID.1610}


}
\UCsubsubsection{Клиент хочет видеть, что товар в наличии на складе}{HBR.TS.007.30}{

\sect{Исходные данные с Wiki}

\begin{wiki}
Клиент в каталоге товаров, в карточке товара и в корзине видит статус «В наличии на складе»
\end{wiki}

\sect{Итоговая формулировка User Story}

\begin{itogo}
Клиент в каталоге товаров, в карточке товара и в корзине видит статус «В наличии на складе»
\end{itogo}

\sect{Пункты спецификации, имеющие отношение к требованию}

\clonerequirement{ID.1720}
\clonerequirement{ID.1060}
\clonerequirement{ID.790}

}

\UCsubsubsection{Клиент хочет видеть, когда на складе может появиться товар, который он хочет заказать}{HBR.TS.007.31}{

\sect{Исходные данные с Wiki}

\begin{wiki}
Клиент в каталоге товаров, в карточке товара и в корзине видит статус «Прибудет в течении X дней», если товар относится к категории C.
Клиент в каталоге товаров, в карточке товара и в корзине видит статус «Ожидается на складе с…по...», если товар относится к категории А и В.
\end{wiki}

\begin{hybris}
Статусы вида "Прибудет в течении X дней" - кто и где это администрирует и предоставляет в Hybris?
\end{hybris}

\sect{Пояснения}

\begin{teamidea}
Данная информация будет выводиться как ответ от веб-сервиса, предоставляемого заказчиком.
\end{teamidea}

\sect{Итоговая формулировка User Story}

\begin{itogo}
Клиент хочет видеть, когда на складе может появиться товар, который он хочет заказать. Данная информация приходит от веб-сервиса, вызываемого по запросу покупателя через нажатие на кнопку/ссылку.
\end{itogo}


}
\UCsubsubsection{Клиент хочет иметь возможность выбрать любой тип отгрузки не зависимо от доступности (статуса) товара}{HBR.TS.007.38}{

\sect{Исходные данные с Wiki}

\begin{wiki}
При оформлении заказа на шаге выбора способа отгрузки клиенту в выпадающем списке доступны все варианты отгрузки
\end{wiki}

\sect{Итоговая формулировка User Story}

\begin{itogo}
При оформлении заказа клиент может выбрать варианты доставки.
\end{itogo}

\sect{Пункты спецификации, имеющие отношение к требованию}

\clonerequirement{ID.1550.1}


}
%\UCsubsubsection{Клиенту доступен бесплатный способ доставки}{HBR.TS.007.39}{
%
%\sect{Исходные данные с Wiki}
%
%\begin{wiki}
%Мы хотим клиенту или группе клиентов дотировать доставку (делать бесплатной)	
%\end{wiki}
%
%\begin{hybris}
%Можно сделать через новый Promotion на Order.
%\end{hybris}
%
%\sect{Итоговая формулировка User Story}
%
%\begin{itogo}
%Клиент или группа клиентов может получить бесплатную доставку (через акцию, модуль Promotion)
%\end{itogo}
%
%\sect{Пункты спецификации, имеющие отношение к требованию}
%
%\clonerequirement{ID.1550.1}
%
%}
\UCsubsubsection{Клиент хочет видеть прогнозируемую дату доставки}{HBR.TS.007.10}{

\sect{Исходные данные с Wiki}

\begin{wiki}
После подтверждения корзины (checkout) клиенту предлагается оформить заказ. На этом этапе мы сообщаем клиенту прогнозируемую дату доставки заказа. (Прогнозируемая дата доставки предоставляется отдельно по каждой позиции/строчке в заказе, а прогнозируемая дата доставки всего заказа - это самая поздняя дата исполнения позиции/строчки заказа). Дата рассчитывается путем обращения в сервис 1С.
Нужна опция, когда мы не можем рассчитать прогнозируемую дату доставки и возвращаем в Hybris "0" значение, поле прогнозируемая дата доставки" будет пустым. В этом случает у клиента должна быть возможность увидеть информационное окно "уточнить дату доставки" обязательное для заполнения. Клиент проставляет флаг в этом окне. Это значит, что в ближайшее время с ним свяжутся для уточнения прогнозируемой даты доставки.

функция, которая дает возможность клиенту видеть прогнозируемую дату (период) дату доставки
функция, которая позволяет клиенту уточнить через звонок требования по прогнозируемой дате доставки

\end{wiki}

\begin{teamidea}
Оценка доработок - 14-16 человеко-дней.
\end{teamidea}

\sect{Итоговая формулировка User Story}

\begin{itogo}
После подтверждения корзины (checkout) клиенту предлагается оформить заказ. На этом этапе мы сообщаем клиенту прогнозируемую дату доставки заказа. 

Прогнозируемая дата доставки предоставляется отдельно по каждой позиции/строчке в заказе, а прогнозируемая дата доставки всего заказа - это самая поздняя дата исполнения позиции/строчки заказа). Дата рассчитывается путем обращения в сервис 1С.

\end{itogo}

\sect{Пункты спецификации, имеющие отношение к требованию}

\clonerequirement{ID.1550.1}


}


\ifcand
\subsection{Кандидаты на последующие этапы}
\UCsubsubsection{Клиент хочет сохранить корзину и перейти к списку сохранённых корзин}{HBR.TS.007.02}{}
\UCsubsubsection{Клиент хочет сделать запрос на коммерческое предложение}{HBR.TS.007.07}{}
\UCsubsubsection{Клиент хочет указать желаемую дату доставки при оформлении заказа}{HBR.TS.007.09}{}
\UCsubsubsection{Клиенту доступен бесплатный способ доставки}{HBR.TS.007.39}{}
\UCsubsubsection{Клиент хочет, чтобы стоимость доставки добавилась к стоимости заказа}{HBR.TS.007.11}{}
\UCsubsubsection{Клиент хочет оставить комментарий для курьера при оформлении заказа}{HBR.TS.007.12}{}
\UCsubsubsection{Клиент хочет иметь возможность оплатить заказ картой на этапе его формирования}{HBR.TS.007.20}{}
\UCsubsubsection{Клиент хочет иметь возможность использовать раннее введенные данные кредитной карты}{HBR.TS.007.21}{}
\UCsubsubsection{Клиент хочет видеть стоимость заказа с НДС/без НДС}{HBR.TS.007.23}{}
\UCsubsubsection{Клиент хочет видеть, что на складе имеется ограниченное кол-во товара }{HBR.TS.007.32}{}
\UCsubsubsection{Клиент хочет видеть, что некоторое кол-во заказанного товара не доступно к отгрузке }{HBR.TS.007.33}{}
\UCsubsubsection{Клиент хочет видеть доступность акционного товара}{HBR.TS.007.34}{}
\UCsubsubsection{Клиент хочет видеть, когда товар снимается с производства }{HBR.TS.007.35}{}
\UCsubsubsection{Клиент хочет видеть, когда товар временно недоступен}{HBR.TS.007.36}{}
\UCsubsubsection{Клиент хочет видеть возможные способы оплаты, в зависимости от выбранного способа отгрузки}{HBR.TS.007.37}{}
\fi

\section{Управление заказами}
\ifcand
\subsection{Согласованные на 1 релиз}
\fi
\UCsubsubsection{Клиент хочет отменить заказ}{HBR.TS.008.02}{

\sect{Исходные данные с Wiki}

\begin{wiki}
Клиент имеет возможность войти в своем личном кабинете войти в список заказов и отослать запрос на отмену заказа если он не находится в статусе старше "Отгружен"

Клиент нажимает кнопку "Отменить заказ", Hybris вызывает внешний сервис и передает измененный статус по заказу
Дополнение:

описать отдельно БП отмены заказа.
\end{wiki}

\begin{teamidea}
Плюс 7-10 рабочих дней. Запрос на отмену предполагает включение в процессы фулфилмента. Много неясных деталей по товародвижению. В оценку включено предположение, что будет некоторый веб-сервис со стороны интеграционной шины, в которую будет направляться эта заявка, а в списке заказов можно будет видеть ее статус или перепослать. Требование новое, на второй этап
\end{teamidea}

\begin{hybris}
Нужен процесс описывающий что и как происходит при отмене заказа. Ответ Технониколь: Отмена заказа разве не должна быть в 1ом релизе Безруков Дмитрий если описывал процесс вставь сюда ссылку
\end{hybris}
\sect{Итоговая формулировка User Story}

\begin{itogo}
Клиент может отправить запрос на отмену заказа. Запрос направляется в ERP веб-сервисом и полностью обслуживается на стороне ERP. Отмена заказа возможна до определенного статуса заказа.
\end{itogo}

}
\UCsubsubsection{Клиент хочет создать заказ на основании старой заказа}{HBR.TS.008.04}{

\sect{Исходные данные с Wiki}

\begin{wiki}
Клиент имеет имеет возможность войти в своем личном кабинете войти в список заказов и создать новую корзину на основании уже существующего заказа
\end{wiki}

\sect{Итоговая формулировка User Story}

\begin{itogo}
Клиент имеет имеет возможность в своем личном кабинете войти в список заказов и создать новую корзину на основании уже существующего заказа
\end{itogo}

\sect{Пункты спецификации, имеющие отношение к требованию}

\clonerequirement{ID.1600}
\clonerequirement{ID.1610}


}
\UCsubsubsection{Клиент хочет утвердить заказ}{HBR.TS.008.05}{

\sect{Исходные данные с Wiki}

\begin{wiki}
Клиенту приходит запрос на утверждение заказа, если он в пользовательской роли менеджера
\end{wiki}

\begin{teamidea}
... в роли утверждающего.
\end{teamidea}

\sect{Итоговая формулировка User Story}

\begin{itogo}
Клиенту приходит запрос на утверждение заказа, если он в пользовательской роли менеджера
\end{itogo}

\sect{Пункты спецификации, имеющие отношение к требованию}

\clonerequirement{ID.1640}
\clonerequirement{ID.1570}

}
\UCsubsubsection{Клиент хочет посмотреть статусы заказа построчно}{HBR.TS.008.06}{

\sect{Исходные данные с Wiki}

\begin{wiki}
Клиент имеет возможность зайти в историю заказов и посмотреть фактические отгрузки по размещенным заказам
(по опыту работы ЭТМ, одна из критических доработок с их слов)
\end{wiki}

\begin{hybris}
Откуда брать фактические отгрузки?
Мы готовы отображать рядом с товаром у заказа в личном кабинете иконку "отгружено", если эта информация будет приходить из внешней информационной системы. Интеграцию не оцениваем - предполагаем, что будет реалзована импексами через уже обсужденный механизм.
\end{hybris}

\sect{Итоговая формулировка User Story}

\begin{itogo}
Клиент имеет возможность зайти в историю заказов и посмотреть фактические отгрузки по размещенным заказам. Информация об отгрузках предоставляется Технониколь вместе с обновлением заказа через механизм IMPEX.
\end{itogo}

}
\ifcand
\subsection{Кандидаты на последующие этапы}
\UCsubsubsection{Клиент хочет изменить заказ}{HBR.TS.008.03}{}
\fi


\section{Каталог версии Staged}
\ifcand
\subsection{Кандидаты на последующие этапы}
\UCsubsubsection{Создание каталога}{HBR.TS.009.1}{

\sect{Исходные данные с Wiki}

\begin{wiki}
Каталог создается сотрудником Первой платформы. 
US не актуальна. 
\end{wiki}


\sect{Итоговая формулировка User Story}

User Story помечена как неактуальная со стороны Заказчика. 


}

\UCsubsubsection{Создание категорий каталога Master}{HBR.TS.009.2}{

\sect{Исходные данные с Wiki}

\begin{wiki}
Импортируются из 1С
\end{wiki}

\sect{Итоговая формулировка User Story}

User Story помечена как неактуальная со стороны Заказчика. 

}
\UCsubsubsection{Создание категорий каталога Web версии Staged}{HBR.TS.009.3}{

\sect{Исходные данные с Wiki}

\begin{wiki}
 US для сотрудников Первой платформы
\end{wiki}

\sect{Итоговая формулировка User Story}

User Story помечена как неактуальная со стороны Заказчика. 

}
\UCsubsubsection{Создание категорий в каталоге Web, версии поставщика}{HBR.TS.009.4}{

\sect{Исходные данные с Wiki}

\begin{wiki}
Создание категорий в каталоге Web, версии поставщика
\end{wiki}

\sect{Итоговая формулировка User Story}

User Story помечена как неактуальная со стороны Заказчика. 

}
\UCsubsubsection{Создание товаров поставщиком }{HBR.TS.009.5}{

\sect{Исходные данные с Wiki}

\begin{wiki}
Создание товаров поставщиком

\begin{enumerate}
\item Staged 1 (доступен для редактирования только сотрудникам Поставщика 1)
\item Staged 2 (доступен для редактирования только сотрудникам Поставщика 2)
\item Staged 3 (доступен для редактирования только сотрудникам Поставщика 3)
\item Staged 4 (доступен для редактирования только сотрудникам Поставщика 4)
\end{enumerate}

Создание категорий:
\begin{enumerate}
\item Предоставление доступа к соответствующей версии каталога поставщику;
\item Поставщик создает категории в собственной версии каталога;
\item Поставщик меняет статус у категории (допустим, "Изменено");
\item Сотрудник Первой платформы видит присвоенный статус и принимает решение об изменении в своей версии каталога;
\item После обработки полученных изменений, сотрудник Первой платформы меняет статус (допустим, "Обработано").
\end{enumerate}
\end{wiki}

\sect{Итоговая формулировка User Story}

User Story помечена как неактуальная со стороны Заказчика. 
}
\UCsubsubsection{Редактирование категорий в каталоге Web версии поставщика}{HBR.TS.009.6}{

\sect{Исходные данные с Wiki}

\begin{wiki}
\begin{enumerate}
\item Импорт sku из 1С
\item Поставщик указывает местоположение товаров в собственном каталоге;
\item Поставщик наполняет карточки товара контентом, медиа;
\item Поставщик меняет статус у карточки товара (допустим, "Изменено");
\item Сотрудник Первой платформы видит присвоенный статус и принимает решение об изменении в своей версии каталога;
\item После обработки полученных изменений, сотрудник Первой платформы меняет статус (допустим, "Обработано").
\end{enumerate}
\end{wiki}

\sect{Итоговая формулировка User Story}

User Story помечена как неактуальная со стороны Заказчика. 


}
\UCsubsubsection{Редактирование товаров в каталоге Web версии поставщика}{HBR.TS.009.7}{

\sect{Исходные данные с Wiki}

\begin{wiki}
\begin{enumerate}
\item Поставщик в собственной версии каталога находит интересующую категорию;
\item Поставщик меняет данные в категории;
\item Поставщик меняет статус у категории (допустим, "Изменено");
\item Сотрудник Первой платформы видит присвоенный статус и принимает решение об изменении в своей версии каталога;
\item После обработки полученных изменений, сотрудник Первой платформы меняет статус (допустим, "Обработано").
\end{enumerate}
\end{wiki}

\sect{Итоговая формулировка User Story}

User Story помечена как неактуальная со стороны Заказчика. 

}
\fi


\section{Конвертация посетителей в покупателей  (захват)}
\ifcand
\subsection{Кандидаты на последующие этапы}
\UCsubsubsection{Первый визит клиента на сайт}{HBR.TS.011.01}{

\sect{Исходные данные с Wiki}

\begin{wiki}
\newline Клиент ввел адрес в строку браузера(перешел из поисковой системы), на сайт компании.
\newline Клиенту в диалоговом окне предлагается ввести свой e-mail адрес,
в обмен на уникальное предложение(ex: первая доставка бесплатно)
\newline Клиенту не доступна работа с сайтом, пока он не 
- воспользуется уникальным предложением
- не откажется от уникального предложения, закрыв окно
\end{wiki}

\sect{Итоговая формулировка User Story}

User Story не согласовано. 

}
\UCsubsubsection{регистрация клиента на сайте}{HBR.TS.011.02}{

\sect{Исходные данные с Wiki}

\begin{wiki}
\newline Клиенту доступна работа сайта с/без предварительной регистрации
\newline Клиент не может произвести checkout без регистрации
\newline Клиент может добавить товар в корзину без регистрации, и товар будет
отображаться в корзине.
\newline Клиент может уйти с сайта без регистрации, система запомнит ID пользователя
и отобразит при следующем заходе корзину, с добавленными в нее ранее товарами.
\newline Клиент ушел с сайта без регистрации, доп.система, настроенная на сайте
запомнила ID пользователя и включила механизм таргетированного
воздействия ( при обращении клиента к любым ресурсам сети Internet
клиент будет видеть персонализированный медийно-контекстный баннер,
с последними добавленными им в корзину товарами.
\end{wiki}

\sect{Итоговая формулировка User Story}

User Story не согласовано. 

}
\UCsubsubsection{Клиент зарегистрировался, но не совершил покупки}{HBR.TS.011.03}{

\sect{Исходные данные с Wiki}

\begin{wiki}
\newline Клиент прошел форму регистрации, но не добавил товар в корзину/ не запросил КП
\newline Клиент сразу после регистрации, получает на почту приветственное письмо, содержащее элементы каталога
\newline Письмо содержит ключевое для первого контакта:
\newline - благодарность за регистрацию (в теме письма)
\newline - призыв к максимально эффективному использованию функционала сайта
\newline - призыв начать совершать покупки уже сейчас+лист товаров
(в подборке из товаров, просмотренных за сеанс на сайте)
\newline - призыв скачать мобильное приложение 
\newline - призыв присоединиться к социальным сетям, что бы быть в курсе акций, новинок и пр.
\newline Клиент, через несколько часов получает еще дополнительное письмо, что пользователь прошел регистрацию, но не положил ни одного товара
в корзину. Письмо напоминает о том, что же он смотрел на сайте
\newline Клиент положил товар в корзину, но не оформил заказ, на почту приходит
письмо, с предложением воспользоваться помощью кц компании, а также
обещанием сохранить товар, оставленный в корзине.
\end{wiki}

\sect{Итоговая формулировка User Story}

User Story не согласовано. 
}
\UCsubsubsection{Клиент зарегистрировался, и подключил к аккаунту других сотрудников компании}{HBR.TS.011.04}{

\sect{Исходные данные с Wiki}

\begin{wiki}
\newline Приглашенные к аккаунту пользователи, получают приветственное
письмо, с краткой инструкцией, как пройти аутентификацию.
\newline Приглашенные пользователи приглашаются к совершению покупок
через личный кабинет
\newline Приглашенным пользователям, отображается список товаров, просмотренных администратором аккаунта.
\end{wiki}

\sect{Итоговая формулировка User Story}

User Story не согласовано. 
}
\UCsubsubsection{Клиент зарегистрировался и совершил покупку}{HBR.TS.011.05}{

\sect{Исходные данные с Wiki}

\begin{wiki}
\newline клиент на почту получает письмо-благодарность, с призывом ответить на вопросы и улучшить сервис компании
\newline письмо содержит информацию о ключевых услугах, доступных клиентам компании
\newline письмо содержит ссылки-переходы на выборку товаров.
\end{wiki}

\sect{Итоговая формулировка User Story}

User Story не согласовано. 

}
\UCsubsubsection{Клиент зарегистрировался, настроил информацию о себе}{HBR.TS.011.06}{

\sect{Исходные данные с Wiki}

\begin{wiki}
\newline клиент прошел процедуру регистрации, и перешел в личный кабинет
\newline клиент ознакомился с функционалом
\newline клиент может ответить на маркетинговый опросник
\newline после ответа, система обрабатывает данные о клиенте, и формирует
целевые рассылки
\end{wiki}

\sect{Итоговая формулировка User Story}

User Story не согласовано. 
}
\UCsubsubsection{На сайте доступна подписка на e-mail рассылки}{HBR.TS.011.07}{

\sect{Исходные данные с Wiki}

\begin{wiki}
\newline на главной странице сайта в футере доступен призыв подписаться на 
e-mail рассылки
\newline пользователь проходит по кнопке "подписаться"
\newline пользователь заполняет информацию о 
\newline - базовая информация ( e-mail адрес, дополнительный e-mail, имя, фамилия, наименование компании, индекс)
\newline - пользователь отмечает на какого типа рассылку он хочет подписаться ( новости, акции, новинки) или может отписаться от всех рассылок
\newline - расширенная информация (о клиенте и его бизнесе)
\end{wiki}

\sect{Итоговая формулировка User Story}

User Story не согласовано. 
}
\UCsubsubsection{Привлечение пользователя в социальную сеть}{HBR.TS.011.08}{

\sect{Исходные данные с Wiki}

\begin{wiki}
при регистрации или иной активности на сайте, в результате которой
клиент оставил свой e-mail, он получает письма с призывом присоединиться к компании в социальной сети
на сайте в футере доступен функционал присоединиться к группам компании в социальных сетях
присоединившись к социальной сети, клиент получает возможность
комментировать все активности компании, быть в курсе новинок, акций
компания при помощи дополнительных настроек систем аналитики получает возможность отслеживать поведение и интересы клиента в сети, тем самым показывать ему "персональный" продукт, настраивать персональные рассылки.
\end{wiki}

\sect{Итоговая формулировка User Story}

User Story не согласовано.

}
\UCsubsubsection{Поведенческий и look-alike таргетинг}{HBR.TS.011.09}{

\sect{Исходные данные с Wiki}

\begin{wiki}
\newline настройка системы google analytics на сайт
\newline сегментация клиентов по признаку: пол, возраст, интересы, доходы
\newline привлечение релевантной по поведению аудитории на сайт при помощи таргетинга
\end{wiki}

\sect{Итоговая формулировка User Story}

User Story не согласовано.
}
\UCsubsubsection{Смс-оповещение}{HBR.TS.011.10}{

\sect{Исходные данные с Wiki}

\begin{wiki}
клиент указал контактный телефон при регистрации
использовать номер телефона для оповещения о различного рода акциях 
\end{wiki}

\sect{Итоговая формулировка User Story}

User Story не согласовано.

}
\UCsubsubsection{Рассылка письма при брошенной регистрации}{HBR.TS.011.11}{

\sect{Исходные данные с Wiki}

\begin{wiki}
пользователь регистрируется на сайте
строка ввода e-mail адреса идет третьей по счету, что бы повысить потенциальность ее заполнения
клиент прекратил регистрацию, но успел ввести адрес
система отправляет клиенту на почту письмо с предложением, продолжить регистрацию
\end{wiki}

\sect{Итоговая формулировка User Story}

User Story не согласовано.
}
\UCsubsubsection{Захват e-mail адресов}{HBR.TS.011.12}{

\sect{Исходные данные с Wiki}

\begin{wiki}
формирование на сайте раздела полезной информации
показ роликов, презентаций, выгод клиента
предложение поделиться понравившимся роликом/материалом с другом
форма для ввода адреса : одновременно получаем два e-mail для дальнейшей работы
\end{wiki}

\sect{Итоговая формулировка User Story}

User Story не согласовано.

}
\UCsubsubsection{Захват e-mail адресов, из адресной книги клиентов}{HBR.TS.011.13}{

\sect{Исходные данные с Wiki}

\begin{wiki}
после оформления заказа, всплывающее окно на сайте, предлагает рассказать контактам из электронной почты клиента, о нашем проекте(магазине), поделиться через социальные сети
клиент получает свою "плюшку"
контакты, откликнувшиеся на предложение пройти регистрацию, так же получают бонус.
\end{wiki}

\sect{Итоговая формулировка User Story}

User Story не согласовано.

}
\UCsubsubsection{Получение адреса клиента, если он ничего не купил и не оставил данных}{HBR.TS.011.14}{

\sect{Исходные данные с Wiki}

\begin{wiki}
если клиент нажимает "закрыть страницу сайта", не оформив заказ, клиенту прежде высвечивается форма, предлагающая оставить контактный мейл
содержание всплывающего окна-идентично
\end{wiki}

\sect{Итоговая формулировка User Story}

User Story не согласовано.
}
\fi

\ifcand
\section{Услуги}
\subsection{Кандидаты на последующие этапы}
\UCsubsubsection{Персональный менеджер}{HBR.TS.012.01}{}
\UCsubsubsection{Резервирование товара}{HBR.TS.012.02}{}
\UCsubsubsection{Экспресс-доставка (для регионов).}{HBR.TS.012.03}{}
\UCsubsubsection{Услуга по страхованию перевозки (для регионов).}{HBR.TS.012.04}{}
\UCsubsubsection{Техническое сопровождение }{HBR.TS.012.05}{}
\UCsubsubsection{Обследование сооружений}{HBR.TS.012.06}{}
\UCsubsubsection{Разработка технических предложений и обоснований }{HBR.TS.012.07}{}
\UCsubsubsection{Проектирование : расчет архитектурной и градостроительной концепции;}{HBR.TS.012.08}{}
\UCsubsubsection{Гарантия онлайн на товары собственного производства }{HBR.TS.012.09}{}
\UCsubsubsection{Калькуляторы расчета кровли}{HBR.TS.012.10}{}
\UCsubsubsection{КОНСУЛЬТАЦИИ НА ОБЪЕКТЕ по особенностям монтажа материалов и комплектующих кровельной системы}{HBR.TS.012.11}{}
\UCsubsubsection{Inventory management}{HBR.TS.012.12}{}
\UCsubsubsection{Пост продажный сервис, ремонт оборудования}{HBR.TS.012.13}{}
\UCsubsubsection{Преимущества, доступные для пользователей e-commerce}{HBR.TS.012.14}{}
\UCsubsubsection{Поддержка технических специалистов/Поддержка по продукту}{HBR.TS.012.15}{}
\UCsubsubsection{Национальный аккаунт (могла не так понять, просьба утвердить или опровергнуть)}{HBR.TS.012.16}{}
\UCsubsubsection{Бытовые услуги}{HBR.TS.012.17}{}
\UCsubsubsection{Авто-заказ}{HBR.TS.012.18}{}
\UCsubsubsection{Дополнительная гарантия на товар}{HBR.TS.012.19}{}
\fi