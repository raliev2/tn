Сценарии использования (см. Приложение \No 1 к настоящему документу) реализуются только в рамках требований, изложенных в п. \ref{funcreqs} \nameref{funcreqs}.

\section{Авторизация пользователя}
\ifcand
\subsection{Согласованные на 1 релиз}
\fi
\UCsubsubsection{Клиент хочет получать информацию о неудавшемся входе на сайт и причину неудачи}{HBR.TS.001.02}
{
\textit{<<Если клиент ввел не корректный логин/пароль, ему должна отобразиться всплывающее окно с информацией об ошибке>>}

Заказчик предоставляет информационное сообщение и дизайн-макеты для реализации этого требования до начала работ по этому блоку. 
}
\UCsubsubsection{Клиент хочет видеть правила ввода данных при авторизации или регистрации}{HBR.TS.001.05}
{
<<Если система определила некорректно введенные данные, или определила что пароль, выбранный при регистрации ненадежный, она должна отобразить эту информацию, и автоматически сбросить некорректные параметры при формировании учетной записи, подсветив поле с некорректной записью красным окном или отобразить эту информацию красным шрифтом над регистрационной формой>> 
}
\UCsubsubsection{Клиент хочет видеть подсказки по ходу процесса регистрации}{HBR.TS.001.06}
{
<<Напротив полей, предназначенных для ввода показывать знаки (например знак вопроса) и при нажатии на него выводить всплывающую подсказку>>

Заказчик предоставляет информационное сообщение и дизайн-макеты для реализации этого требования до начала работ по этому блоку. 
}
\UCsubsubsection{Клиент хочет online связаться с call-центром и получить помощь в регистрации}{HBR.TS.001.07}
{
<<При прохождении регистрации, клиенту необходимо видеть сервис <<связаться с оператором>> или телефон номера 8-800, для оперативной поддержки>>

Заказчик предоставляет информационное сообщение и дизайн-макеты для реализации этого требования до начала работ по этому блоку. 
}
\UCsubsubsection{Клиент хочет разлогиниться и выполнить вход под другими учетными данными}{HBR.TS.001.09}
{
<<Клиент выходит из системы при помощи кнопки <<выйти>>. При входе вводит необходимый для него логин/пароль.  
}



\UCsubsubsection{Клиент хочет разместить заказ по телефону}{HBR.TS.001.10}
{
<<Клиент может оставтиь заявку по телефону, если у клиента уже создан аккаунт, этот заказ также должен отобразиться в его личном кабинете в разделе заказов, оставленных по телефону>> 

Заказчик предоставляет информацию о заказах по телефону через загрузку данных в hybris, дизайн-макеты для реализации этого требования до начала работ по этому блоку.
}
\UCsubsubsection{Клиент хочет зарегистрироваться в процессе оформления заказа}{HBR.TS.001.11}
{
<<После того, как клиент положил товар в корзину и нажал checkout, ему будет предложено или войти как авторизованному пользователю или зарегистрироваться. Если же он решил пройти регистрацию до момента checkout, товары, положенные в корзину, должны остаться в ней>>
}
\UCsubsubsection{Клиент хочет видеть информацию о том, что он не может завершить заказ, пока не авторизуется}{HBR.TS.001.13}
{
<<При совершении заказа клиент должен не просто видеть вариант авторизации или регистрации, он должен понимать, что заказ в один клик, без авторизации/регистрации -невозможен. Это необходимо отобразить при помощи дополнительного сообщения>>

Заказчик предоставляет информационное сообщение и дизайн-макеты для реализации этого требования до начала работ по этому блоку.
}
\UCsubsubsection{Клиент хочет иметь возможность автоматической авторизации}{HBR.TS.001.14}
{
<<При входе в личный кабинет, необходимо предоставить клиенту запомнить его учетные данные, и при повторном входе на сайт, не запрашивать их, в том числе и при оформлении заказа. После сохранения данных, входа на сайт и добавления товара в корзину, после checkout система пропустила шаг с авторизацией, и сразу выдала шаг-выбор способа доставки и оплаты>>. 
}

\ifcand
\subsection{Кандидаты на последующие этапы}
\UCsubsubsection{Клиент, в зависимости от роли,хочет пройти авторизацию или регистрацию на сайте}{HBR.TS.001.01}
\UCsubsubsection{Клиент хочет изменить пароль или логин}{HBR.TS.001.03}
\UCsubsubsection{Клиент забыл пароль и хочет получить новый/старый пароль на свой e-mail}{HBR.TS.001.04}
\UCsubsubsection{Клиент хочет получить на свой e-mail ссылку для подтверждения регистрации}{HBR.TS.001.08}
\UCsubsubsection{Клиент хочет видеть статус своего запроса на регистрацию}{HBR.TS.001.12}
\fi

\section{Личный кабинет пользователя}

\ifcand
\subsection{Согласованные на 1 релиз}
\fi
\UCsubsubsection{Клиент зашел как покупатель}{HBR.TS.002.01}
{
<<Клиент в роли покупателя не может использовать функции управления аккаунтами и может смотреть историю только по своим заказам>>
}
\UCsubsubsection{Клиент зашел как утверждающий заказы}{HBR.TS.002.02}
{
<<Клиент в роли утверждающего заказы может иметь права покупателя. Может видеть в своем личном кабинете уведомления о наличии заказов, которые необходимо утвердить>>
}
\UCsubsubsection{Клиент зашел как менеджер}{HBR.TS.002.03}
{
<<Клиент в роли менеджера может иметь роль покупателя, и может видеть все покупки и все заказы сотрудников своей организации>>
}
\UCsubsubsection{Клиент зашел как администратор}{HBR.TS.002.04}
{
<<Клиент в роли администратора может использовать все функции ЛК>>
}
\UCsubsubsection{Клиент хочет отредактировать свои персональные данные}{HBR.TS.002.08}
{
<<Клиент, в роли администратора, переходит на вкладку "Мои персональные данные" и может редактировать персональные данные:
\begin{itemize}
	\item User ID
	\item Страну
	\item ФИО
	\item Почтовые адреса
	\item телефон
	\item Так же клиент может добавлять:
		\begin{itemize}
		\item несколько вариантов телефонов
		\item несколько вариантов адресов
		\item несколько вариантов e-mail, без изменения текущего>>
		\end{itemize}
\end{itemize}
}
\UCsubsubsection{Клиент, как администратор, хочет настраивать права всех пользователей, которых он привязывает к учетной записи}{HBR.TS.002.15}
{
<<Клиент Клиент переходит на вкладку <<Управение пользоват елями>> и создает новых пользователей, заполняет их данные и назначает права. просматривает права всех пользователей, привязанных к этой учетной записи, а так же задает права созданным пользователям и задает условия привязки для новых пользователей>>
}
\UCsubsubsection{Клиент, как администратор, хочет создавать пользователей и назначать им права}{HBR.TS.002.16}
{
<<Клиент переходит на вкладку <<Сопровождение пользователей>> и создает новых пользователей, заполняет их данные и назначает права>>
}
\ifcand
\subsection{Кандидаты на последующие этапы}
\UCsubsubsection{Клиент хочет отредактировать свои данные для входа на сайт}{HBR.TS.002.09}{}
\UCsubsubsection{Клиент хочет управлять своими маркетинговыми настройками}{HBR.TS.002.10}{}
\UCsubsubsection{Клиент хочет посмотреть статус по заказам}{HBR.TS.002.11}{}
\UCsubsubsection{Клиент, как администратор, хочет проводить персонализацию сайта для каждого пользователя}{HBR.TS.002.17}{}
\UCsubsubsection{Клиент хочет управлять настройками метода отгрузок}{HBR.TS.002.18}{}
\UCsubsubsection{Клиент хочет управлять ТК и экспедиторскими службами, с которыми работает.}{HBR.TS.002.19}{}
\UCsubsubsection{Клиент хочет настраивать для себя точки выдачи и отгрузки товара}{HBR.TS.002.20}{}
\UCsubsubsection{Клиент хочет скачать копии счет-фактур и других отгрузочных документов, привязанных к выбранному заказу}{HBR.TS.002.21}{}
\UCsubsubsection{Клиент хочет видеть счет-фактуры всех заказов своей организации.}{HBR.TS.002.22}{}
\UCsubsubsection{Клиент хочет видеть акты-сверки в своем личном кабинете}{HBR.TS.002.23}{}
\UCsubsubsection{Клиент хочет распечатывать список своих корзин, покупок и заказов}{HBR.TS.002.24}{}
\fi

\section{Поиск товара}
\ifcand
\subsection{Согласованные на 1 релиз}
\fi
\UCsubsubsection{Клиент воспользовался быстрым поиском на сайте}{HBR.TS.003.01}
{
<<Окно быстрого поиска товаров должно быть доступно на любой странице сайта.

Поиск возможно осуществлять по параметрам:
\begin{itemize}
	\item Артикул товара (ЕКН);
	\item Полное/частичное наименование товара;
	\item Ключевые слова;
	\item Категория>>
\end{itemize}

<<Предусмотрен поиск по нескольким значениям артикулов (указываются через запятую).

В поисковой строке должна содержаться подсказка о том, что требуется ввести (Введите артикул или слово для поиска).

Описание действий:
\begin{itemize}

\item Поиск по артикулу:
	\begin{enumerate} 
		\item Клиент вводит артикул в окно поиска;
		\item Нажимает Поиск (Search);
		\item Окрывается конкретная карточка товара.
	\end{enumerate}	
	
\item Поиск по ключевым словам:
 	\begin{enumerate} 
	 	\item Клиент дает запрос по слову/словосочетанию в окно поиска;
		\item Появляется всплывающий список, подходящий под параметры (категории, бренды) с возможностью выбора и перехода на любой пункт. При этом, если выбор сделан, производится автоматический переход к списку товаров категории либо бренда;
		\item Нажать на Поиск (Search);
		\item Отображаются товары, содержащие ключевые слова в полях, по которым условились выполнять поиск по ключевым словам.
	\end{enumerate}
	
\item Поиск по названию категории (полное совпадение названия):
 	\begin{enumerate} 
 		\item Клиент дает запрос названия категории в окно поиска;
		\item	Появляется всплывающий список, подходящий под параметры с возможностью перехода. При этом, если выбор сделан, производится автоматический переход к списку товаров категории;
		\item	Нажать на Поиск (Search);
		\item	Ниже появляются товары, принадлежащие категории.
	\end{enumerate}	
		
\item	Поиск по имени бренда (полное совпадение):
 	\begin{enumerate} 
		\item	Клиент дает запрос имени бренда в окно поиска;
		\item	Появляется всплывающий список, подходящий под параметры с возможностью перехода. При этом, если выбор сделан, производится автоматический переход к списку товаров бренда;
		\item	Нажать на Поиск (Search);
		\item	Ниже появляются товары, принадлежащие бренду.
	\end{enumerate}	
\end{itemize}>>
}
\UCsubsubsection{Клиент хочет ввести ограничения к поиску}{HBR.TS.003.02}
{
<<Уточнение поиска возможно при <<неконкретном>> запросе товара (когда поиск осуществляется по ключевым словам, категориям, брендам). Либо при изначальном поиске (после выбора категории в каталоге товаров).

Дополнительные фильтры появляются слева.

Фильтры делятся по типам (категория, производитель, ценовой диапазон, технические характеристики, поиск по словам в найденном, вариант каталога).

Выводятся только те значения, которые возможны при изначальной фильтрации. Т.е., если изначальные выбор сделан по категории Батареи, то категория Абразивы для уточнения появиться не может.

Рядом с каждым возможным элементом фильтра указано количество sku.

Предусмотрена возможность просмотра полного списка вариантов по выбранному типу уточнения.

Предусмотрена возможность скрыть (свернуть) тип уточнения.

Описание действий:
\begin{itemize}
\item Поиск по словам в найденном:
	\begin{itemize}
		\item Клиент вводит слово/словосочетание в поле слева;
		\item Нажимает Перейти (Go);
		\item Выводится список товаров, которые подходят под изначальный этап фильтрации + введенное слово/словосочетание.
	\end{itemize}

\item Уточнение по категории:
	\begin{itemize}
		\item Слева имеется возможность просмотра более полного перечня предлагаемых категорий;
		\item Производится выбор интересующей категории;
		\item Выводится список товаров;
		\item Происходит автоматическое уменьшение возможных фильтров исходя из отобранных товаров.
	\end{itemize}

\item Уточнение по бренду:

	\begin{itemize}
		\item Слева имеется возможность просмотра более полного перечня предлагаемых брендов;
		\item Производится выбор интересующего бренда;
		\item Выводится список товаров;
		\item Происходит автоматическое уменьшение возможных фильтров исходя из отобранных товаров.
	\end{itemize}

\item Уточнение по ценовому диапазону:

	\begin{itemize}
		\item Слева имеется возможность просмотра более полного перечня предлагаемых диапазонов;
		\item Производится выбор интересующего диапазона;
		\item Выводится список товаров;
		\item Происходит автоматическое уменьшение возможных фильтров исходя из отобранных товаров.
	\end{itemize}

\end{itemize}
}
\UCsubsubsection{Клиент хочет сортировать результат поиска}{HBR.TS.003.03}
{
Возможность сортировки появляется только в том случае, если выведен список товаров.

Результатом поиска является всегда список товаров. 

Варианты сортировки:
\begin{itemize}
\item Лидеры продаж;
\item По бренду А-Я;
\item По бренду Я-А;
\item По артикулу (ЕКН): возрастание;
\item По артикулу (ЕКН): убывание;
\item По цене: возрастание;
\item По цене: убывание;
\item По рейтингу (отзывам): возрастание;
\item По рейтингу (отзывам): убывание.
\end{itemize}

Сортировка производится по цене, хранящейся в индексе (может отличаться от цены для конкретного покупателя и от цены, изменившейся после последней индексации).

Все варианты сортировки находятся в одном выпадающем списке.

После осуществления сортировки по выбранному параметру возможность повторной сортировки должна оставаться.

При выборе нового способа сортировки предыдущий выбор должен сбрасываться.

Сортировка так же доступна в самом списке товаров - при нажатии на название соответствующего столбика в списке товаров.

Необходимые действия:

Необходимо сделать выбор типа сортировки, после чего будет выведен список товаров в соответствующем порядке.

Сортировка в списке товаров:

Нажать на название колонки, по которой должна быть сделана сортировка. При этом, колонки, по которым возможна сортировка, меняют тип курсора (со стрелки на руку). После - формируется список в выбранном порядке.
}
\UCsubsubsection{Клиент хочет видеть информацию о количестве найденных товаров}{HBR.TS.003.04}
{
<<Количество отобранных товаров всегда отображается над списком товаров.
 
Также отображается количество товаров при потенциальном отборе>>
}
\UCsubsubsection{Клиент хочет иметь возможность накладывать дополнительные условия поиска на результат поиска}{HBR.TS.003.05}
{
<<Данная возможность появляется после того, как установлено ограничение на список товаров.

После того, как устанавливается каждое из дополнительных условий, список таковых должен корректироваться в зависимости от перечня товаров.

Рядом с каждым дополнительным условием отображается количество доступных sku.

Предусмотрена возможность накладывания нескольких дополнительных условий.

Типы дополнительных условий (такие же, как и ограничения):

\begin{itemize}
\item категория, 
\item производитель, 
\item ценовой диапазон\footnote{Будет использована базовая цена, актуальная на момент индексации}, 
\item технические характеристики, 
\item поиск по словам в найденном, 
\item вариант каталога
\end{itemize}

Возможные действия:

\begin{itemize}
\item Установка дополнительного условия
\item Очистить форму дополнительного условия
\item Просмотреть более широкий список условий по предпочитаемому типу
\item Скрыть тип дополнительного условия, при этом установленных выбор не <<слетает>>
\end{itemize}
}
\UCsubsubsection{Клиент хочет выбирать товары из результата поиска и переходить в карточку товара}{HBR.TS.003.06}
{
<<Клиент кликает на на название либо артикул (ЕКН) товара из сформированного списка и "проваливается" в карточку товара>>
}
\UCsubsubsection{Клиент хочет позвонить менеджеру и проконсультироваться по товару}{HBR.TS.003.07}
{
<<Клиент в футере сайта обращается к блоку "Есть вопрос? Позвоните:..." и совершает звонок по указанному номеру телефона.
Заполняет форму обращения и отправляет заявку: цель обращения, контактные данные, вопрос>>
}
\UCsubsubsection{Клиент хочет позвонить менеджеру, если поиск не выдал ни одного результата}{HBR.TS.003.08}
{
<<Клиент обращается к блоку "Есть вопрос? Позвоните:..." и совершает звонок по указанному номеру телефона.
Заполняет форму обращения и отправляет заявку: цель обращения, контактные данные, вопрос>>
}
\UCsubsubsection{Клиент хочет видеть полное или краткое описание товара в результате поиска}{HBR.TS.003.10}{}

\UCsubsubsection{Клиент хочет увеличить изображение товара}{HBR.TS.003.11}{}

\ifcand
\subsection{Кандидаты на последующие этапы}
\UCsubsubsection{Клиент хочет видеть основные ТХ товаров в результате поиска, чтобы удобно сравнивать их}{HBR.TS.003.09}{}
\fi


\section{Карточка товара}
\ifcand
\subsection{Согласованные на 1 релиз}
\fi
\UCsubsubsection{Клиент перешел к карточке товара}{HBR.TS.005.01}{}
\UCsubsubsection{Клиент хочет увидеть увеличенное изображение товара}{HBR.TS.005.02}{}
\UCsubsubsection{Клиент хочет увидеть другие изображения товара}{HBR.TS.005.03}{}
\UCsubsubsection{Клиент хочет получить подробное описание товара}{HBR.TS.005.04}{}
\UCsubsubsection{Клиент хочет получить информацию о возможности заказать товар}{HBR.TS.005.05}{}
\UCsubsubsection{Клиент хочет увидеть похожие товары}{HBR.TS.005.06}{}
\UCsubsubsection{Клиент хочет увидеть сопутствующие товары}{HBR.TS.005.07}{}
\UCsubsubsection{Клиент хочет увидеть специальные предложения по товару}{HBR.TS.005.08}{}
\UCsubsubsection{Клиент хочет добавить товар в корзину}{HBR.TS.005.09}{}
\UCsubsubsection{Клиент хочет видеть стоимость товара в зависимости от своей роли и согласно с заключенным договором на поставку товаров}{HBR.TS.005.10}{}


\section{Корзина}
\ifcand
\subsection{Согласованные на 1 релиз}
\fi
\UCsubsubsection{Клиент хочет добавить в корзину выбранные им товары}{HBR.TS.006.01}
{
\begin{itemize}
\item Шаг 1. Находим нужный товар в каталоге Продуктов
\item Шаг 2. Вносим в окне "Кол-во", кол-во товара к заказу
\item Шаг 3. Нажимаем кнопку "В корзину"
\end{itemize}

Находясь в продуктовом каталоге, клиент указывает количество товара для заказа, нажимает на кнопку <<Добавить в корзину>>. 
Клиент попадает на форму в которой ему предлагается добавить в корзину товары, рекомендуемые к закупке вместе с тем товаром, который добавляется в корзину, а также он может выбрать дальнейшие действия, и, либо продолжить выбор товаров (ссылка <<Продолжить добавление>> или кнопка <<Оформить заказ>>)
}
\UCsubsubsection{Заказ рекомендованного товара в корзине}{HBR.TS.006.02}
{
\begin{itemize}
\item Шаг 1. Нажимаем ссылку "Моя корзина"
\item Шаг 2. Клиент вводит кол-во товара к заказу в окне "Кол-во" выбранного товара
\item Шаг 3. Клиент нажимает кнопку "В корзину" и товар попадает в список товаров в корзине
\end{itemize}
}
\UCsubsubsection{Редактирование кол-ва товара добавленного в корзину}{HBR.TS.006.04}{}
\UCsubsubsection{Удаление товара добавленного в корзину}{HBR.TS.006.05}{}
\UCsubsubsection{Клиент хочет, чтобы не подтвержденная корзина сохранялась и при следующем входе на сайт}{HBR.TS.006.13}{}
\UCsubsubsection{Клиент хочет начать оформление заказа воспользовавшись функцией корзины Checkout }{HBR.TS.006.19}{}
\ifcand
\subsection{Кандидаты на последующие этапы}
\UCsubsubsection{Быстрое добавление товара в корзину}{HBR.TS.006.06}{}
\UCsubsubsection{Проверка доступности товара на складе ближайшем к месту доставки}{HBR.TS.006.07}{}
\UCsubsubsection{Клиент может проверить доступность товара набранного в корзину на ближайшем складе после ввода индекса}{HBR.TS.006.09}{}
\fi

\section{Заказ}
\ifcand
\subsection{Согласованные на 1 релиз}
\fi
\UCsubsubsection{Клиент хочет подтвердить корзину и перейти к оформлению заказа}{HBR.TS.007.01}{}
\UCsubsubsection{Клиент хочет видеть пошаговый механизм оформления заказа}{HBR.TS.007.03}{}
\UCsubsubsection{Клиент хочет получать подсказки о назначении элементов интерфейса в контекстном меню}{HBR.TS.007.04}{}
\UCsubsubsection{Клиент хочет выбирать тип оплаты}{HBR.TS.007.05}{}
\UCsubsubsection{Клиент хочет выбирать юридическое лицо для конкретного заказа}{HBR.TS.007.06}{}
\UCsubsubsection{Клиент хочет выбирать способ отгрузки из предлагаемого списка способов доставки}{HBR.TS.007.08}{}
\UCsubsubsection{Клиент хочет изменить дату доставки на последнем шаге оформления заказа (финальная страница order review)}{HBR.TS.007.13}{}
\UCsubsubsection{Клиент хочет распечатать спецификацию на этапе корзины (текущей или сохранённой)}{HBR.TS.007.15}{}
\UCsubsubsection{Клиент хочет видеть подсказки по оформлению заказа}{HBR.TS.007.16}{}
\UCsubsubsection{Клиент хочет видеть сообщения о неудачной обработке заказа и причину}{HBR.TS.007.17}{}
\UCsubsubsection{Клиент хочет связаться со специалистом call-центра для получения помощи}{HBR.TS.007.18}{}
\UCsubsubsection{Клиент хочет поменять тип оплаты}{HBR.TS.007.19}{}
\UCsubsubsection{Клиент хочет иметь возможность разместить свой заказ}{HBR.TS.007.22}{}
\UCsubsubsection{Клиент хочет получать уведомление на свой e-mail о том, что он успешно разместил заказ}{HBR.TS.007.25}{}
\UCsubsubsection{Клиент хочет получить информацию о том, что заказ не удалось разместить и причину неудачи}{HBR.TS.007.26}{}
\UCsubsubsection{Клиент хочет, чтобы с ними связался менеджер, если заказ не удалось разместить}{HBR.TS.007.27}{}
\UCsubsubsection{Клиент хочет связаться с менеджером, если у него возникли вопросы по размещению заказа}{HBR.TS.007.28}{}
\UCsubsubsection{Клиент хочет регулярно получать одни и те же товары, без дополнительной процедуры оформления заказа}{HBR.TS.007.29}{}
\UCsubsubsection{Клиент хочет видеть, что товар в наличии на складе}{HBR.TS.007.30}{}
\UCsubsubsection{Клиент хочет видеть, когда на складе может появиться товар, который он хочет заказать}{HBR.TS.007.31}{}
\UCsubsubsection{Клиент хочет иметь возможность выбрать любой тип отгрузки не зависимо от доступности (статуса) товара}{HBR.TS.007.38}{}
\UCsubsubsection{Клиенту доступен бесплатный способ доставки}{HBR.TS.007.39}{}

\ifcand
\subsection{Кандидаты на последующие этапы}
\UCsubsubsection{Клиент хочет сохранить корзину и перейти к списку сохранённых корзин}{HBR.TS.007.02}{}
\UCsubsubsection{Клиент хочет сделать запрос на коммерческое предложение}{HBR.TS.007.07}{}
\UCsubsubsection{Клиент хочет указать желаемую дату доставки при оформлении заказа}{HBR.TS.007.09}{}
\UCsubsubsection{Клиент хочет видеть прогнозируемую дату доставки}{HBR.TS.007.10}{}
\UCsubsubsection{Клиент хочет, чтобы стоимость доставки добавилась к стоимости заказа}{HBR.TS.007.11}{}
\UCsubsubsection{Клиент хочет оставить комментарий для курьера при оформлении заказа}{HBR.TS.007.12}{}
\UCsubsubsection{Клиент хочет иметь возможность оплатить заказ картой на этапе его формирования}{HBR.TS.007.20}{}
\UCsubsubsection{Клиент хочет иметь возможность использовать раннее введенные данные кредитной карты}{HBR.TS.007.21}{}
\UCsubsubsection{Клиент хочет видеть стоимость заказа с НДС/без НДС}{HBR.TS.007.23}{}
\UCsubsubsection{Клиент хочет видеть, что на складе имеется ограниченное кол-во товара }{HBR.TS.007.32}{}
\UCsubsubsection{Клиент хочет видеть, что некоторое кол-во заказанного товара не доступно к отгрузке }{HBR.TS.007.33}{}
\UCsubsubsection{Клиент хочет видеть доступность акционного товара}{HBR.TS.007.34}{}
\UCsubsubsection{Клиент хочет видеть, когда товар снимается с производства }{HBR.TS.007.35}{}
\UCsubsubsection{Клиент хочет видеть, когда товар временно недоступен}{HBR.TS.007.36}{}
\UCsubsubsection{Клиент хочет видеть возможные способы оплаты, в зависимости от выбранного способа отгрузки}{HBR.TS.007.37}{}
\fi

\section{Управление заказами}
\ifcand
\subsection{Согласованные на 1 релиз}
\fi
\UCsubsubsection{Клиент хочет отменить заказ}{HBR.TS.008.02}{}
\UCsubsubsection{Клиент хочет создать заказ на основании старой заказа}{HBR.TS.008.04}{}
\UCsubsubsection{Клиент хочет утвердить заказ}{HBR.TS.008.05}{}
\UCsubsubsection{Клиент хочет посмотреть статусы заказа построчно}{HBR.TS.008.06}{}
\ifcand
\subsection{Кандидаты на последующие этапы}
\UCsubsubsection{Клиент хочет изменить заказ}{HBR.TS.008.03}{}
\fi

\ifcand
\section{Каталог версии Staged}
\subsection{Кандидаты на последующие этапы}
\UCsubsubsection{Создание каталога}{HBR.TS.009.1}{}
\UCsubsubsection{Создание категорий каталога Master}{HBR.TS.009.2}{}
\UCsubsubsection{Создание категорий каталога Web версии Staged}{HBR.TS.009.3}{}
\UCsubsubsection{Создание категорий в каталоге Web, версии поставщика}{HBR.TS.009.4}{}
\UCsubsubsection{Создание товаров поставщиком }{HBR.TS.009.5}{}
\UCsubsubsection{Редактирование категорий в каталоге Web версии поставщика}{HBR.TS.009.6}{}
\UCsubsubsection{Редактирование товаров в каталоге Web версии поставщика}{HBR.TS.009.7}{}
\fi

\ifcand
\section{Конвертация посетителей в покупателей  (захват)}
\subsection{Кандидаты на последующие этапы}
\UCsubsubsection{Первый визит клиента на сайт}{HBR.TS.011.01}{}
\UCsubsubsection{регистрация клиента на сайте}{HBR.TS.011.02}{}
\UCsubsubsection{Клиент зарегистрировался, но не совершил покупки}{HBR.TS.011.03}{}
\UCsubsubsection{Клиент зарегистрировался, и подключил к аккаунту других сотрудников компании}{HBR.TS.011.04}{}
\UCsubsubsection{Клиент зарегистрировался и совершил покупку}{HBR.TS.011.05}{}
\UCsubsubsection{Клиент зарегистрировался, настроил информацию о себе}{HBR.TS.011.06}{}
\UCsubsubsection{На сайте доступна подписка на e-mail рассылки}{HBR.TS.011.07}{}
\UCsubsubsection{Привлечение пользователя в социальную сеть}{HBR.TS.011.08}{}
\UCsubsubsection{Поведенческий и look-alike таргетинг}{HBR.TS.011.09}{}
\UCsubsubsection{Смс-оповещение}{HBR.TS.011.10}{}
\UCsubsubsection{Рассылка письма при брошенной регистрации}{HBR.TS.011.11}{}
\UCsubsubsection{Захват e-mail адресов}{HBR.TS.011.12}{}
\UCsubsubsection{Захват e-mail адресов, из адресной книги клиентов}{HBR.TS.011.13}{}
\UCsubsubsection{Получение адреса клиента, если он ничего не купил и не оставил данных}{HBR.TS.011.14}{}
\fi

\ifcand
\section{Услуги}
\subsection{Кандидаты на последующие этапы}
\UCsubsubsection{Персональный менеджер}{HBR.TS.012.01}{}
\UCsubsubsection{Резервирование товара}{HBR.TS.012.02}{}
\UCsubsubsection{Экспресс-доставка (для регионов).}{HBR.TS.012.03}{}
\UCsubsubsection{Услуга по страхованию перевозки (для регионов).}{HBR.TS.012.04}{}
\UCsubsubsection{Техническое сопровождение }{HBR.TS.012.05}{}
\UCsubsubsection{Обследование сооружений}{HBR.TS.012.06}{}
\UCsubsubsection{Разработка технических предложений и обоснований }{HBR.TS.012.07}{}
\UCsubsubsection{Проектирование : расчет архитектурной и градостроительной концепции;}{HBR.TS.012.08}{}
\UCsubsubsection{Гарантия онлайн на товары собственного производства }{HBR.TS.012.09}{}
\UCsubsubsection{Калькуляторы расчета кровли}{HBR.TS.012.10}{}
\UCsubsubsection{КОНСУЛЬТАЦИИ НА ОБЪЕКТЕ по особенностям монтажа материалов и комплектующих кровельной системы}{HBR.TS.012.11}{}
\UCsubsubsection{Inventory management}{HBR.TS.012.12}{}
\UCsubsubsection{Пост продажный сервис, ремонт оборудования}{HBR.TS.012.13}{}
\UCsubsubsection{Преимущества, доступные для пользователей e-commerce}{HBR.TS.012.14}{}
\UCsubsubsection{Поддержка технических специалистов/Поддержка по продукту}{HBR.TS.012.15}{}
\UCsubsubsection{Национальный аккаунт (могла не так понять, просьба утвердить или опровергнуть)}{HBR.TS.012.16}{}
\UCsubsubsection{Бытовые услуги}{HBR.TS.012.17}{}
\UCsubsubsection{Авто-заказ}{HBR.TS.012.18}{}
\UCsubsubsection{Дополнительная гарантия на товар}{HBR.TS.012.19}{}
\fi