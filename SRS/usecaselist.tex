Сценарии использования (см. Приложение \No 1 к настоящему документу) реализуются только в рамках требований, изложенных в п. \ref{funcreqs} \nameref{funcreqs}.

\section{Авторизация пользователя}
\ifcand
\subsection{Согласованные на 1 релиз}
\fi

\UCsubsubsection{Клиент хочет получать информацию о неудавшемся входе на сайт и причину неудачи}{HBR.TS.001.02}
{

\sect{Исходные данные c Wiki}

\begin{wiki}
Если клиент ввел не корректный логин/пароль, ему должна отобразиться страница с информацией, example (все поля можно  оставить как есть в примере)
\end{wiki}

\begin{teamidea}
Нужно определиться, будет ли это всплывающее окно или отдельная страница на сайте. Если отдельная - над или под ней будет форма.  В идеале нужен прототип. Но требование простое, можно просто ответить словами (ответ TN: <<Это будет всплывающее окно>>)
\end{teamidea}


\sect{Итоговая формулировка User Story}


\begin{itogo}
Пользователь вводит некорректный логин или пароль; При отправке формы пользователь получает сообщение "Указанные вами реквизиты доступа к системе некорректные. Воспользуйтесь сервисом восстановления пароля";
\end{itogo}

}

\UCsubsubsection{Клиент хочет видеть правила ввода данных при авторизации или регистрации}{HBR.TS.001.05}
{

\sect{Исходные данные c Wiki}

\begin{wiki}
Если система определила некорректно введенные данные, или определила, что пароль, выбранный при регистрации не является надежным, она должна отобразить эту информацию, и автоматически сбросить некорректные параметры при формировании учетной записи, подсветив поле с некорректной записью красным окном или отобразить эту информацию красным шрифтом над регистрационной формой
\end{wiki} 

\sect{Пояснения}

\begin{teamidea}
На первом этапе нет формы регистрации. Для формы авторизации оба поля обязательны. В случае ошибочно заполненной формы указывать на поле, в котором была сделана ошибка, нельзя по соображениям безопасности. Поэтому этот User Story для первого этапа целиком повторяет \textbf{HBR.TS.001.02} 
\end{teamidea}

\sect{Итоговая формулировка User Story}

\begin{itogo}
Пользователь вводит некорректный логин или пароль;
При отправке формы пользователь получает сообщение "Указанные вами реквизиты доступа к системе некорректные. Воспользуйтесь сервисом восстановления пароля";
\end{itogo}

}


\UCsubsubsection{Клиент хочет видеть подсказки по ходу процесса регистрации}{HBR.TS.001.06}
{
\sect{Исходные данные c Wiki}

\begin{wiki}
Напротив полей, предназначенных для ввода показывать знаки (например знак вопроса) и при нажатии на него выводить всплывающую подсказку, в grainger информация отображается в отдельном окне браузера, происходит переход на другое окно , но клиент не покидает форму регистрации.
\end{wiki}

\sect{Итоговая формулировка User Story}

\begin{itogo}
При авторизации пользователь может получить подсказку, что означает каждое поле, нажав на иконку рядом (если это предусмотрено дизайном);
\end{itogo}

}

\UCsubsubsection{Клиент хочет online связаться с call-центром и получить помощь в регистрации}{HBR.TS.001.07}
{

\sect{Исходные данные c Wiki}

\begin{wiki}
При прохождении регистрации, клиенту необходимо видеть сервис "связаться с оператором" или телефон номера 8-800, для оперативной поддержки, а так же необходима информация клиенту, призывающая связаться с оператором, если регистрация на сайте не доступна/не получается и т.д., что бы клиент при многоразовых попытках ввода своих данных, не ушел со страницы регистрации. Либо эта подсказка может появляться как всплывающее окно, после трех-кратно неудачных попыток регистрации. Здесь стоит понимать, что в хайбрис при многократных попытках происходить ничего не будет, данный модуль будет доработкой. вопрос-принимаем или нет. 
\end{wiki}

\begin{teamidea}
Требование сформулированно некорректно. "Клиент хочет online связаться с call-центром" - это что-то про онлайн-чат и онлайн-поддержку. В описании же указано совсем иное - это просто информационное окно.
Логика вида "После N неудачных попыток показывать информацию", думаю, избыточна. Достаточно показывать эту строчку просто всегда.
(ответ Технониколь: стоит обсудить потребность в данном функционале /Сорокин Андрей Романовский Виктор/)
\end{teamidea}

\sect{Итоговая формулировка User Story}

\begin{itogo}
 При прохождении регистрации, клиенту необходимо видеть ссылку на сервис <<связаться с оператором>>\footnote{Сервис не входит в объем проекта} или телефон номера 8-800, для оперативной поддержки.
\end{itogo}
}

\UCsubsubsection{Клиент хочет разлогиниться и выполнить вход под другими учетными данными}{HBR.TS.001.09}
{

\sect{Исходные данные c Wiki}

\begin{wiki}
Клиент просто выходит из системы при помощи кнопки "выйти" и при входе вводит необходимый для него логин/пароль. Это не должно быть сложно. Да же если клиент запомнил свои данные, и при новом входе на сайт, вверху сразу отобразилась последняя используемая учетная запись, он прjсто рядом с этим же меню выходит из аккаунта, и при помощи кнопки "войти" может войти под другим аккаунтом. 
\end{wiki}

\sect{Итоговая формулировка User Story}

\begin{itemize}
\item Клиент выходит из системы при помощи кнопки <<выйти>>. При входе вводит необходимый для него логин/пароль.  
\end{itemize}
}



\UCsubsubsection{Клиент хочет разместить заказ по телефону}{HBR.TS.001.10}
{
\sect{Исходные данные c Wiki}

\begin{wiki}
Клиент звонит по телефону сам, или оставляет заявку, что бы с ним связались -оставив заявку по телефону, если у клиента уже создан аккаунт, этот заказ так же должен отобразиться в его личном кабинете в разделе заказов, оставленных по телефону. см. табличку п. 2 статус заказа.
\end{wiki}

\begin{teamidea}
Потребуется доработка "Заявка на обратный звонок" (1-2 человеко-дня)
(ответ ТехноНиколь: здесь ключевое требование не столько заявка на обратный звонок, сколько отображения заказов, сделанных через телефон, в личном кабинете-предотвратить потерю истории Алиев Рауф)
\end{teamidea}

\sect{Итоговая формулировка User Story}

\begin{itogo}
Клиент может оставтиь заявку по телефону, если у клиента уже создан аккаунт. Этот заказ также должен отобразиться в его личном кабинете в разделе заказов, оставленных по телефону.
\end{itogo}

}

\UCsubsubsection{Клиент хочет зарегистрироваться в процессе оформления заказа}{HBR.TS.001.11}
{

\sect{Исходные данные c Wiki}

\begin{wiki}
После того, как клиент положил товар в корзину и нажал checkout, ему будет предложено или войти как авторизованному пользователю или зарегистрироваться. Если же он решил пройти регистрацию до момента checkout, он должен пройти регистрацию, при этом товары, накиданные в корзину должны остаться в ней. 
\end{wiki}

\begin{teamidea}
Это в определенной степени не согласуется с желанием таскать с собой корзину между устройствами. Типа положил что-то на мобильном (или дома), а на десктопе (или на работе) после авторизации это что-то увидел уже положенным в корзину. Тем не менее, требуемый порядок может реализовать на первом этапе
(Ответ ТехноНИКОЛЬ: Рауф, ты говоришь про кроссдевайсность, это хорошая вещь, но вряд ли на 1 этапе)
\end{teamidea}

\sect{Итоговая формулировка User Story}

\begin{itogo}
После того, как клиент положил товар в корзину и нажал checkout, ему будет предложено или войти как авторизованному пользователю или зарегистрироваться. Если же он решил пройти регистрацию до момента checkout, товары, положенные в корзину, должны остаться в ней
\end{itogo}
}

\UCsubsubsection{Клиент хочет видеть информацию о том, что он не может завершить заказ, пока не авторизуется}{HBR.TS.001.13}
{
\sect{Исходные данные c Wiki}

\begin{wiki}
При совершении заказа, клиент должен не просто видеть вариант авторизации или регистрации, он должен понимать, что заказ в один клик, без авторизации/регистрации -невозможен. Это необходимо отобразить при помощи дополнительного сообщения.
\end{wiki}

\sect{Итоговая формулировка User Story}

\begin{itogo}
Необходимо отобразить информационное сообщение, что заказв один клик без авторизации невозможен.
\end{itogo}

Заказчик предоставляет информационное сообщение и дизайн-макеты для реализации этого требования до начала работ по этому блоку.
}

\UCsubsubsection{Клиент хочет иметь возможность автоматической авторизации}{HBR.TS.001.14}
{
\sect{Исходные данные c Wiki}

\begin{wiki}
При входе в личный кабинет, необходимо предоставить клиенту запомнить его учетные данные, и при повторном входе на сайт, не запрашивать их., в том числе и при оформлении заказа. После сохранения данных, входа на сайт и добавления товара в корзину, после checkout система пропустила шаг с авторизацией, и сразу выдала шаг-выбор способа доставки и оплаты. 
\end{wiki}

\sect{Итоговая формулировка User Story}

\begin{itogo}
При входе в личный кабинет, необходимо предоставить клиенту запомнить его учетные данные, и при повторном входе на сайт, не запрашивать их, в том числе и при оформлении заказа (с учетом ограничений, накладываемых браузером). 
Для авторизованного пользователя заказ проходит в обход формы авторизации, так как авторизация уже совершена ранее.
\end{itogo}

}

\ifcand
\subsection{Кандидаты на последующие этапы}
\UCsubsubsection{Клиент, в зависимости от роли,хочет пройти авторизацию или регистрацию на сайте}{HBR.TS.001.01}
\UCsubsubsection{Клиент хочет изменить пароль или логин}{HBR.TS.001.03}
\UCsubsubsection{Клиент забыл пароль и хочет получить новый/старый пароль на свой e-mail}{HBR.TS.001.04}
\UCsubsubsection{Клиент хочет получить на свой e-mail ссылку для подтверждения регистрации}{HBR.TS.001.08}
\UCsubsubsection{Клиент хочет видеть статус своего запроса на регистрацию}{HBR.TS.001.12}
\fi

\section{Личный кабинет пользователя}

\ifcand
\subsection{Согласованные на 1 релиз}
\fi
\UCsubsubsection{Клиент зашел как покупатель}{HBR.TS.002.01}
{
\sect{Исходные данные c Wiki}

\begin{wiki}
Клиент в роли покупателя не может использовать функции управления аккаунтами и может смотреть историю только по своим \sout{покупкам} и заказам
\end{wiki}

\begin{hybris}
История по покупкам и заказам - это история заказа и его содержимого?
\end{hybris}

\sect{Итоговая формулировка User Story}

\begin{itogo}
Клиент в роли покупателя не может использовать функции управления аккаунтами и может смотреть историю только по своим заказам
\end{itogo}
}
\UCsubsubsection{Клиент зашел как утверждающий заказы}{HBR.TS.002.02}
{

\sect{Исходные данные c Wiki}

\begin{wiki}
Клиент в роли утверждающего заказы наследует права покупателя, и может видеть в своем ЛК уведомления о наличии заказов, которые необходимо утвердить
\end{wiki}

\begin{teamidea}
В логике станлартного акселлератора он не обязательно может наследовать.
\end{teamidea}

\begin{hybris}
Нет наследования, есть группы с соответствующим функционалом. B2BCustomer роль может покупать. B2B Approver может утверждать. Соответственно пользователь может получить обе роли
\end{hybris}

\sect{Итоговая формулировка User Story}

\begin{itogo}
Клиент в роли утверждающего заказы может иметь права покупателя. Может видеть в своем личном кабинете уведомления о наличии заказов, которые необходимо утвердить
\end{itogo}
}
\UCsubsubsection{Клиент зашел как менеджер}{HBR.TS.002.03}
{

\sect{Исходные данные c Wiki}

\begin{wiki}
Клиент в роли менеджера наследует роль покупателя, и может видеть все покупки и все заказы сотрудников своей организации
\end{wiki}

\begin{teamidea}
В логике станлартного акселлератора он не обязательно может наследовать.
\end{teamidea}

\sect{Итоговая формулировка User Story}

\begin{itogo}
Клиент в роли менеджера может иметь роль покупателя, и может видеть все покупки и все заказы сотрудников своей организации
\end{itogo}
}



\UCsubsubsection{Клиент зашел как администратор}{HBR.TS.002.04}
{
\sect{Исходные данные c Wiki}

\begin{wiki}
Клиент в роли администратора может использовать все функции ЛК
\end{wiki}

\sect{Итоговая формулировка User Story}

\begin{itogo}
Клиент в роли администратора может использовать все функции ЛК
\end{itogo}
}


\UCsubsubsection{Клиент хочет отредактировать свои персональные данные}{HBR.TS.002.08}
{
\sect{Исходные данные c Wiki}

\begin{wiki}
Клиент, в роли администратора, переходит на вкладку "Мои персональные данные" и может редактировать персональные данные:
\begin{itemize}
\item User ID
\item Страну
\item ФИО
\item Почтовые адреса
\item телефон
\item Так же клиент может добавлять:
	\begin{itemize}
	\item несколько вариантов телефонов
	\item несколько вариантов адресов
	\item несколько вариантов e-mail, без изменения текущего
	\end{itemize}
\end{itemize}
\end{wiki}

\begin{hybris}
То есть пользователь с другой ролью не может делать эти изменения?
\end{hybris}

\sect{Итоговая формулировка User Story}

\begin{itogo}
Клиент, в роли администратора, переходит на вкладку "Мои персональные данные" и может редактировать персональные данные:
\begin{itemize}
	\item User ID
	\item Страну
	\item ФИО
	\item Почтовые адреса
	\item телефон
	\item Так же клиент может добавлять:
		\begin{itemize}
		\item несколько вариантов телефонов
		\item несколько вариантов адресов
		\item несколько вариантов e-mail, без изменения текущего>>
		\end{itemize}
\end{itemize}
\end{itogo}
}



\UCsubsubsection{Клиент, как администратор, хочет настраивать права всех пользователей, которых он привязывает к учетной записи}{HBR.TS.002.15}
{

\sect{Исходные данные c Wiki}

\begin{wiki}
Клиент переходит на вкладку "Управление аккаунтом" и просматривает права всех пользователей, привязанных к этой учетной записи, а так же задает права созданным пользователям и задает условия привязки для новых пользователей.
\end{wiki}

\sect{Пояснения}

\begin{teamidea}
Непонятно, что такое <<задает условия привязки для новых пользователей>>
\end{teamidea}

\sect{Итоговая формулировка User Story}

\begin{itogo}
Клиент переходит в раздел <<Управение пользователями>> личного кабинета и создает новых пользователей, заполняет их данные и назначает права. просматривает права всех пользователей, привязанных к этой учетной записи, а так же задает права созданным пользователям.
\end{itogo}
}


\UCsubsubsection{Клиент, как администратор, хочет создавать пользователей и назначать им права}{HBR.TS.002.16}
{

\sect{Исходные данные c Wiki}

\begin{wiki}
Клиент переходит на вкладку "Сопровождение пользователей" и создает новых пользователей, заполняет их данные и назначает права.
\end{wiki}

\begin{teamidea}
Непонятно, почему не обойтись одной вкладкой? Зачем "Сопровождение..." ограничивать только созданием
\end{teamidea}

\sect{Итоговая формулировка User Story}

\begin{itogo}
Клиент переходит на вкладку <<Сопровождение пользователей>> и создает новых пользователей, заполняет их данные и назначает права
\end{itogo}

}
\ifcand
\subsection{Кандидаты на последующие этапы}
\UCsubsubsection{Клиент хочет отредактировать свои данные для входа на сайт}{HBR.TS.002.09}{}
\UCsubsubsection{Клиент хочет управлять своими маркетинговыми настройками}{HBR.TS.002.10}{}
\UCsubsubsection{Клиент хочет посмотреть статус по заказам}{HBR.TS.002.11}{}
\UCsubsubsection{Клиент, как администратор, хочет проводить персонализацию сайта для каждого пользователя}{HBR.TS.002.17}{}
\UCsubsubsection{Клиент хочет управлять настройками метода отгрузок}{HBR.TS.002.18}{}
\UCsubsubsection{Клиент хочет управлять ТК и экспедиторскими службами, с которыми работает.}{HBR.TS.002.19}{}
\UCsubsubsection{Клиент хочет настраивать для себя точки выдачи и отгрузки товара}{HBR.TS.002.20}{}
\UCsubsubsection{Клиент хочет скачать копии счет-фактур и других отгрузочных документов, привязанных к выбранному заказу}{HBR.TS.002.21}{}
\UCsubsubsection{Клиент хочет видеть счет-фактуры всех заказов своей организации.}{HBR.TS.002.22}{}
\UCsubsubsection{Клиент хочет видеть акты-сверки в своем личном кабинете}{HBR.TS.002.23}{}
\UCsubsubsection{Клиент хочет распечатывать список своих корзин, покупок и заказов}{HBR.TS.002.24}{}
\fi

\section{Поиск товара}
\ifcand
\subsection{Согласованные на 1 релиз}
\fi
\UCsubsubsection{Клиент воспользовался быстрым поиском на сайте}{HBR.TS.003.01}
{

\sect{Исходные данные c Wiki}

\begin{wikilong}
Окно быстрого поиска товаров должно быть доступно на любой странице сайта.

Поиск возможно осуществлять по параметрам:
\begin{itemize}
	\item Артикул товара (ЕКН);
	\item Полное/частичное наименование товара;
	\item Ключевые слова;
	\item Категория>>
\end{itemize}

Предусмотрен поиск по нескольким значениям артикулов (указываются через запятую).

В поисковой строке должна содержаться подсказка о том, что требуется ввести (Введите артикул или слово для поиска).

Описание действий:
\begin{itemize}

\item Поиск по артикулу:
	\begin{enumerate} 
		\item Клиент вводит артикул в окно поиска;
		\item Нажимает Поиск (Search);
		\item Окрывается конкретная карточка товара.
	\end{enumerate}	
	
\item Поиск по ключевым словам:
 	\begin{enumerate} 
	 	\item Клиент дает запрос по слову/словосочетанию в окно поиска;
		\item Появляется всплывающий список, подходящий под параметры (категории, бренды) с возможностью выбора и перехода на любой пункт. При этом, если выбор сделан, производится автоматический переход к списку товаров категории либо бренда;
		\item Нажать на Поиск (Search);
		\item Отображаются товары, содержащие ключевые слова в полях, по которым условились выполнять поиск по ключевым словам.
	\end{enumerate}
	
\item Поиск по названию категории (полное совпадение названия):
 	\begin{enumerate} 
 		\item Клиент дает запрос названия категории в окно поиска;
		\item	Появляется всплывающий список, подходящий под параметры с возможностью перехода. При этом, если выбор сделан, производится автоматический переход к списку товаров категории;
		\item	Нажать на Поиск (Search);
		\item	Ниже появляются товары, принадлежащие категории.
	\end{enumerate}	
		
\item	Поиск по имени бренда (полное совпадение):
 	\begin{enumerate} 
		\item	Клиент дает запрос имени бренда в окно поиска;
		\item	Появляется всплывающий список, подходящий под параметры с возможностью перехода. При этом, если выбор сделан, производится автоматический переход к списку товаров бренда;
		\item	Нажать на Поиск (Search);
		\item	Ниже появляются товары, принадлежащие бренду.
	\end{enumerate}	
\end{itemize}
\end{wikilong}

\begin{teamidea}
"Появляется всплывающий список, подходящий под параметры (категории, бренды)" - стандартное поведение акселлератора предполагает отображение в всплывающем списке товаров, а не брендов или категорий. Да и пользоавтели привыкли видеть там дополнение их начатой фразы, а не совсем другие слова. Подтвердите, что верно понимаем задачу.

\textbf{Ответ заказчика: "Если в поиске мы введем Шинглас, то должна отобразиться категория Шинглас, куда клиент сможет "провалиться" и увидеть полный перечень товаров. На Грейнджере реализовано именно так."}

Поиск по категориям является дополнительным функционалом, которого нет в базовом акселлераторе hybris. В качестве результатов поиска будут выдаваться всегда только товары. Другое поведение - на второй этап. Для оценки нужно дополнительное время на изучение (около 1 недели)
\end{teamidea}

\sect{Итоговая формулировка User Story}

\begin{itogolong}
Окно быстрого поиска товаров должно быть доступно на любой странице сайта.

Поиск возможно осуществлять по параметрам:
\begin{itemize}
	\item Артикул товара (ЕКН);
	\item Полное/частичное наименование товара;
	\item Ключевые слова;
	\item Категория>>
\end{itemize}

Предусмотрен поиск по нескольким значениям артикулов (указываются через запятую).

В поисковой строке должна содержаться подсказка о том, что требуется ввести (Введите артикул или слово для поиска).

Описание действий:
\begin{itemize}

\item Поиск по артикулу:
	\begin{enumerate} 
		\item Клиент вводит артикул в окно поиска;
		\item Нажимает Поиск (Search);
		\item Окрывается конкретная карточка товара.
	\end{enumerate}	
	
\item Поиск по ключевым словам:
 	\begin{enumerate} 
	 	\item Клиент дает запрос по слову/словосочетанию в окно поиска;
		\item Появляется всплывающий список, подходящий под параметры (категории, бренды) с возможностью выбора и перехода на любой пункт. При этом, если выбор сделан, производится автоматический переход к списку товаров категории либо бренда;
		\item Нажать на Поиск (Search);
		\item Отображаются товары, содержащие ключевые слова в полях, по которым условились выполнять поиск по ключевым словам.
	\end{enumerate}
	
\item Поиск по названию категории (полное совпадение названия):
 	\begin{enumerate} 
 		\item Клиент дает запрос названия категории в окно поиска;
		\item	Появляется всплывающий список, подходящий под параметры с возможностью перехода. При этом, если выбор сделан, производится автоматический переход к списку товаров категории;
		\item	Нажать на Поиск (Search);
		\item	Ниже появляются товары, принадлежащие категории.
	\end{enumerate}	
		
\item	Поиск по имени бренда (полное совпадение):
 	\begin{enumerate} 
		\item	Клиент дает запрос имени бренда в окно поиска;
		\item	Появляется всплывающий список, подходящий под параметры с возможностью перехода. При этом, если выбор сделан, производится автоматический переход к списку товаров бренда;
		\item	Нажать на Поиск (Search);
		\item	Ниже появляются товары, принадлежащие бренду.
	\end{enumerate}	
\end{itemize}
\end{itogolong}
}



\UCsubsubsection{Клиент хочет ввести ограничения к поиску}{HBR.TS.003.02}
{

\sect{Исходные данные c Wiki}

\begin{wikilong}
Уточнение поиска возможно при "неконкретном" запросе товара (когда поиск осуществляется по ключевым словам, категориям, брендам). Либо при изначальном поиске (после выбора категории в каталоге товаров).

Дополнительные фильтры появляются слева.

Фильтры делятся по типам (категория, производитель, ценовой диапазон, технические характеристики, поиск по словам в найденном, вариант каталога).

Выводятся только те значения, которые возможны при изначальной фильтрации. Т.е., если изначальные выбор сделан по категории Батареи, то категория 

Абразивы для уточнения появиться не может.

Рядом с каждым возможным выбором указано количество sku.

Предусмотрена возможность просмотра полного списка вариантов по выбранному типу уточнения.

Предусмотрена возможность скрыть (свернуть) тип уточнения.

Описание действий:
\begin{itemize}
\item Поиск по словам в найденном:
	\begin{itemize}
		\item Клиент вводит слово/словосочетание в поле слева;
		\item Нажимает Перейти (Go);
		\item Выводится список товаров, которые подходят под изначальный этап фильтрации + введенное слово/словосочетание.
	\end{itemize}

\item Уточнение по категории:
	\begin{itemize}
		\item Слева имеется возможность просмотра более полного перечня предлагаемых категорий;
		\item Производится выбор интересующей категории;
		\item Выводится список товаров;
		\item Происходит автоматическое уменьшение возможных фильтров исходя из отобранных товаров.
	\end{itemize}

\item Уточнение по бренду:

	\begin{itemize}
		\item Слева имеется возможность просмотра более полного перечня предлагаемых брендов;
		\item Производится выбор интересующего бренда;
		\item Выводится список товаров;
		\item Происходит автоматическое уменьшение возможных фильтров исходя из отобранных товаров.
	\end{itemize}

\item Уточнение по ценовому диапазону:

	\begin{itemize}
		\item Слева имеется возможность просмотра более полного перечня предлагаемых диапазонов;
		\item Производится выбор интересующего диапазона;
		\item Выводится список товаров;
		\item Происходит автоматическое уменьшение возможных фильтров исходя из отобранных товаров.
	\end{itemize}

\end{itemize}
\end{wikilong}

\begin{teamidea}
"Слева имеется возможность просмотра более полного перечня предлагаемых категорий;" - уточнение по категории показывает категории, в которых находятся найденные товары, а не "более полный перечень". Следовательно, должны отображаться категории, сужающие поисковые результаты, список этих категорий является списком категорий, в которые входят найденные товары. Просим подтвердить, правильно ли понимаем задачу.

Ответ ТехноНиколь: Речь об опции View All Categories
\end{teamidea}


\sect{Итоговая формулировка User Story}

\begin{itogolong}
Уточнение поиска возможно при <<неконкретном>> запросе товара (когда поиск осуществляется по ключевым словам, категориям, брендам). Либо при изначальном поиске (после выбора категории в каталоге товаров).

Дополнительные фильтры появляются слева.

Фильтры делятся по типам (категория, производитель, ценовой диапазон, технические характеристики, поиск по словам в найденном, вариант каталога).

Выводятся только те значения, которые возможны при изначальной фильтрации. Т.е., если изначальные выбор сделан по категории Батареи, то категория Абразивы для уточнения появиться не может.

Рядом с каждым возможным элементом фильтра указано количество sku.

Предусмотрена возможность просмотра полного списка вариантов по выбранному типу уточнения.

Предусмотрена возможность скрыть (свернуть) тип уточнения.

Описание действий:
\begin{itemize}
\item Поиск по словам в найденном:
	\begin{itemize}
		\item Клиент вводит слово/словосочетание в поле слева;
		\item Нажимает Перейти (Go);
		\item Выводится список товаров, которые подходят под изначальный этап фильтрации + введенное слово/словосочетание.
	\end{itemize}

\item Уточнение по категории:
	\begin{itemize}
		\item Слева имеется возможность просмотра более полного перечня предлагаемых категорий;
		\item Производится выбор интересующей категории;
		\item Выводится список товаров;
		\item Происходит автоматическое уменьшение возможных фильтров исходя из отобранных товаров.
	\end{itemize}

\item Уточнение по бренду:

	\begin{itemize}
		\item Слева имеется возможность просмотра более полного перечня предлагаемых брендов;
		\item Производится выбор интересующего бренда;
		\item Выводится список товаров;
		\item Происходит автоматическое уменьшение возможных фильтров исходя из отобранных товаров.
	\end{itemize}

\item Уточнение по ценовому диапазону:

	\begin{itemize}
		\item Слева имеется возможность просмотра более полного перечня предлагаемых диапазонов;
		\item Производится выбор интересующего диапазона;
		\item Выводится список товаров;
		\item Происходит автоматическое уменьшение возможных фильтров исходя из отобранных товаров.
	\end{itemize}

\end{itemize}
\end{itogolong}
}



\UCsubsubsection{Клиент хочет сортировать результат поиска}{HBR.TS.003.03}
{
\sect{Исходные данные c Wiki}

\begin{wikilong}
Возможность сортировки появляется только в том случае, если выведен список товаров.
\begin{itemize}
\item Варианты сортировки:
\item Лидеры продаж;
\item По бренду А-Я;
\item По бренду Я-А;
\item По артикулу (ЕКН): возрастание;
\item По артикулу (ЕКН): убывание;
\item По остаткам на складе: возрастание;
\item По остаткам на складе: убывание;
\item По цене: возрастание;
\item По цене: убывание;
\item По рейтингу (отзывам): возрастание;
\item По рейтингу (отзывам): убывание.
\end{itemize}
Все варианты сортировки находятся в одном выпадающем списке.
После осуществления сортировки по выбранному параметру возможность повторной сортировки должна оставаться.
При выборе нового способа сортировки предыдущий выбор должен сбрасываться.
Сортировка так же доступна в самом списке товаров - при нажатии на название соответствующего столбика в списке товаров.
Необходимые действия:
\begin{enumerate}
\item Сортировка через окно:
Необходимо сделать выбор типа сортировки, после чего будет выведен список товаров в соответствующем порядке.
\item Сортировка в списке товаров:
Нажать на название колонки, по которой должна быть сделана сортировка. При этом, колонки, по которым возможна сортировка, меняют тип курсора (со стрелки на руку). После - формируется список в выбранном порядке.
\end{enumerate}
\end{wikilong}

\begin{teamidea}

"Возможность сортировки появляется только в том случае, если выведен список товаров." - Результатом поиска является всегда список товаров. Просим подтвердить, правильно ли понимаем задачу.

Ответ Технониколь: На Грейнджере не всегда список товаров является результатом поиска.

По остаткам на складе сортировка возможна только, если эти остатки будут индексироваться. В случае, если остатки изменяются часто извне, и по большому набору товаров сразу, (пере)индексирование большинства товаров может занять 1-2 часа (точные оценки после прототипирования). За это время может прийти обновленная информация по остаткам. Будет расхождение. Рекомендуется убрать из индекса эту сортировку.
С ценой замечания те же, но цены меняются реже наличия.
Вопрос - зачем нужна сортировка по артикулам?
"При выборе нового способа фильтрации предыдущий выбор должен сбрасываться." - непонятное требование

Ответ Технониколь: Речь о том, что сортировка должна выстраиваться заново. Вероятно, это логично. Но на всякий случай указала.

"Сортировка так же доступна в самом списке товаров - при нажатии на название соответствующего столбика в списке товаров." - непонятное требование

Ответ Технониколь: Ниже описано в Необходимые действия

В блоке "Необходимые действия" есть ощущение, что автор путает сортировку с фильтрацией. Фильтрация есть ограничение списка по какому-то критерию. Сортировка - изменение списка с сохранением его длины и состава элементов.

Ответ Технониколь: внесены коррективы в кейс
\end{teamidea}

\begin{hybris}
Согласен, большая тема и нужно на этапе анализа требований понять, что мы будем индексировать и хранить в индексе.

Вопрос Технониколь: Anton Gavazyuk - а какие данные по умолчанию индексируются
\end{hybris}

\sect{Итоговая формулировка User Story}

\begin{itogolong}
Возможность сортировки появляется только в том случае, если выведен список товаров.

Результатом поиска является всегда список товаров. 

Варианты сортировки:
\begin{itemize}
\item Лидеры продаж;
\item По бренду А-Я;
\item По бренду Я-А;
\item По артикулу (ЕКН): возрастание;
\item По артикулу (ЕКН): убывание;
\item По цене: возрастание;
\item По цене: убывание;
\item По рейтингу (отзывам): возрастание;
\item По рейтингу (отзывам): убывание.
\end{itemize}

Сортировка производится по цене, хранящейся в индексе (может отличаться от цены для конкретного покупателя и от цены, изменившейся после последней индексации).

Все варианты сортировки находятся в одном выпадающем списке.

После осуществления сортировки по выбранному параметру возможность повторной сортировки должна оставаться.

При выборе нового способа сортировки предыдущий выбор должен сбрасываться.

Сортировка так же доступна в самом списке товаров - при нажатии на название соответствующего столбика в списке товаров.

Необходимые действия:

Необходимо сделать выбор типа сортировки, после чего будет выведен список товаров в соответствующем порядке.

Сортировка в списке товаров:

Нажать на название колонки, по которой должна быть сделана сортировка. При этом, колонки, по которым возможна сортировка, меняют тип курсора (со стрелки на руку). После - формируется список в выбранном порядке.

\end{itogolong}

}




\UCsubsubsection{Клиент хочет видеть информацию о количестве найденных товаров}{HBR.TS.003.04}
{

\sect{Исходные данные c Wiki}

\begin{wiki}
Количество отобранных товаров всегда отображается над списком товаров.
Так же отображается количество товаров при потенциальном отборе (указано в описании кейса на уточнении поиска).
\end{wiki}

\begin{teamidea}
Количество товаров показывается поисковой системой SOLR, демонстрируем "как есть". Теоретически может отличаться от действительного количества товаров, удовлетворяющих критериям.
\end{teamidea}

\sect{Итоговая формулировка User Story}

\begin{itogo}
Количество отобранных товаров всегда отображается над списком товаров. Также отображается количество товаров при потенциальном отборе
\end{itogo}
}



\UCsubsubsection{Клиент хочет иметь возможность накладывать дополнительные условия поиска на результат поиска}{HBR.TS.003.05}
{

\sect{Исходные данные c Wiki}

\begin{wiki}
Количество отобранных товаров всегда отображается над списком товаров.
Так же отображается количество товаров при потенциальном отборе (указано в описании кейса на уточнении поиска).
\end{wiki}

\begin{hybris}
Большая тема, если facet search должен учитывать цены завязанные для клиентов, так как по умолчанию в поисковом индексе хранится только одна цена, не специальная для клиента. То есть сейчас возможна ситуация применить фильтр 50-100 рублей, но цена товара для текущего клиента будет не в этом диапазоне.
\end{hybris}

\sect{Итоговая формулировка User Story}

\begin{itogolong}
Данная возможность появляется после того, как установлено ограничение на список товаров.

После того, как устанавливается каждое из дополнительных условий, список таковых должен корректироваться в зависимости от перечня товаров.

Рядом с каждым дополнительным условием отображается количество доступных sku.

Предусмотрена возможность накладывания нескольких дополнительных условий.

Типы дополнительных условий (такие же, как и ограничения):

\begin{itemize}
\item категория, 
\item производитель, 
\item ценовой диапазон\footnote{Будет использована базовая цена, актуальная на момент индексации}, 
\item технические характеристики, 
\item поиск по словам в найденном, 
\item вариант каталога
\end{itemize}

Возможные действия:

\begin{itemize}
\item Установка дополнительного условия
\item Очистить форму дополнительного условия
\item Просмотреть более широкий список условий по предпочитаемому типу
\item Скрыть тип дополнительного условия, при этом установленных выбор не <<слетает>>
\end{itemize}
\end{itogolong}


}



\UCsubsubsection{Клиент хочет выбирать товары из результата поиска и переходить в карточку товара}{HBR.TS.003.06}
{

\sect{Исходные данные c Wiki}

\begin{wiki}
Клиент кликает на на название либо артикул (ЕКН) товара из сформированного списка и "проваливается" в карточку товара
\end{wiki}

\sect{Итоговая формулировка User Story}

\begin{itogo}
Клиент кликает на на название либо артикул (ЕКН) товара из сформированного списка и "проваливается" в карточку товара
\end{itogo}
}


\UCsubsubsection{Клиент хочет позвонить менеджеру и проконсультироваться по товару}{HBR.TS.003.07}
{
\sect{Исходные данные c Wiki}

\begin{wiki}
Вариант 1: Клиент в футере сайта обращается к блоку "Есть вопрос? Позвоните:..." и совершает звонок по указанному номеру телефона.
 
Вариант 2: В футере сайта клиент обращается к блоку Поддержка/Заказ отложенного звонка.
 
Заполняет форму обращения и отправляет заявку: цель обращения, контактные данные, вопрос.
\end{wiki}

\begin{teamidea}
<<Форма обращения>> -- новый функционал. Имеет ли смысл делать эту US или лучше сконцетрироваться на том, чтоб у клиента не было вопросов при просмотре товара.
\end{teamidea}

\sect{Итоговая формулировка User Story}

\begin{itogo}
Клиент в футере сайта обращается к блоку "Есть вопрос? Позвоните:..." и совершает звонок по указанному номеру телефона.
\end{itogo}

}
\UCsubsubsection{Клиент хочет позвонить менеджеру, если поиск не выдал ни одного результата}{HBR.TS.003.08}
{
Дублирует HBR.TS.003.07.
}
\UCsubsubsection{Клиент хочет видеть полное или краткое описание товара в результате поиска}{HBR.TS.003.10}{

\sect{Исходные данные c Wiki}

\begin{wiki}
В отфильтрованном списке товаров отображается краткое описание:
 
Полное описание товара клиент может увидеть, перейдя в карточку товара (описано в Карточка товара).
\end{wiki}

\sect{Итоговая формулировка User Story}

\begin{itogo}
В отфильтрованном списке товаров отображается краткое описание товара.  
Полное описание товара клиент может увидеть, перейдя в карточку товара.
\end{itogo}
}

\UCsubsubsection{Клиент хочет увеличить изображение товара}{HBR.TS.003.11}{

\sect{Исходные данные c Wiki}

\begin{wiki}
Доступно только в Карточка товара
\end{wiki}

\sect{Итоговая формулировка User Story}

\begin{itogo}
Дублирует User Story для Карточки товара
\end{itogo}
}

\ifcand
\subsection{Кандидаты на последующие этапы}
\UCsubsubsection{Клиент хочет видеть основные ТХ товаров в результате поиска, чтобы удобно сравнивать их}{HBR.TS.003.09}{}
\fi


\section{Карточка товара}
\ifcand
\subsection{Согласованные на 1 релиз}
\fi
\UCsubsubsection{Клиент перешел к карточке товара}{HBR.TS.005.01}{

\sect{Исходные данные c Wiki}

\begin{wikilong}

Карточка товара, должна отвечать определенным критериям и содержать в себе информацию о: 

Перед информацией о товаре включается текст, с описанием продукта -по своей сути это  SEO текст, с основными ключевиками. 
\begin{enumerate}
\item наименование товара
\item цена авторизованного пользователя. так же непосредственно в карточке должна содержаться информация, что если клиент не авторизовался, ему доступна, только розничная цена товара
\item бренд товара
\item идентификатор производителя
\item ID товара на сайте 
\item Минимальное количество, от которого производится отгрузка
\item Минимальное количество, которое доступно для заказа
\item отгрузочный вес
\item количество доступное для заказа
\item страница с товаром в печатном каталоге
\item страна производитель
\item количество, требуемое для заказа 
\end{enumerate}

Далее система должна предложить клиенту оформить заказ, как: 

\begin{itemize}
\item единичный заказ
\item добавить товар, в список для системы авто-заказа
\end{itemize}

(продолжение списка)
\begin{enumerate}
\item добавить в корзину
\item добавить в персональный или общий лист
\item проверить наличие необходимого количества товара на складе
\begin{itemize}
\item   поле для ввода необходимого количества
\item   поле для ввода идентификатора (индекса)
\end{itemize}
\item фото контент 
\item  автоматически отобразить,доступные для данного товара услуги, и позволять добавлять их в заказ 
\item  возможность работы с отзывами о товаре
\begin{itemize}
\item оставить отзыв
\item прочитать отзывы других пользователей
\item воспользоваться сервисом вопрос-ответ
\end{itemize}
\end{enumerate}
\end{wikilong}

\begin{teamidea}
Автозаказ возможен только на весь заказ целиком. Поэтому кнопку "Автозаказ" на карточке товара без изменения логики сделать нельзя.

Не предполагаем реализацию на первом этапе страницы товара в печатном каталоге (непонятно)

Решили не показывать количество - не учитываем

Решили не делать мультикорзинность на первом этапе - не учитывали на первом этапе эти требования

Не было требований по услугам - если услуги будут частью каталога и будут связанными товарами - то ок, если нет - то на второй этап

Непонятно, что такое сервис "вопрос-ответ" Не включен в оценку на первый этап

Со всеми дополнениями - на первый этап
\end{teamidea}

\begin{hybris}
Минимальное количество, которое доступно для заказа (Ответ ТехноНИКОЛЬ: это привязано к остатками или к продукту?)

количество доступное для заказа (Ответ ТехноНИКОЛЬ: решили же не показывать)

страница с товаром в печатном каталоге (Ответ ТехноНИКОЛЬ: это что?)

В первом релизе возможно только добавления в текущую корзину (Ответ ТехноНИКОЛЬ: верно?)
\end{hybris}

\sect{Итоговая формулировка User Story}

\begin{itogo}
Карточка товара, должна содержать в себе информацию о: 
\begin{itemize}
\item наименование товара,
\item описание товара,
\item цена,
\item (для неаутентифицированных) информация о том, что цена - розничная,
\item бренд товара,
\item идентификатор производителя,
\item ID товара на сайте,
\item Минимальное количество, от которого производится отгрузка,
\item Минимальное количество, которое доступно для заказа,
\item отгрузочный вес,
\item страна-производитель,
\item количество, требуемое для заказа.
\item добавить в корзину
\item фотография товара
\item отзывах (с возможностью оставить свой отзыв)
\end{itemize}
\end{itogo}
}


\UCsubsubsection{Клиент хочет увидеть увеличенное изображение товара}{HBR.TS.005.02}{

	\sect{Исходные данные c Wiki}
	
	\begin{wiki}
	В карточке товара есть фото-контент. Изображение можно увеличить, с помощью специальной кнопки "увеличить изображение"
	картинка, в увеличенном формате, всплывает в дополнительно окне. 
	\end{wiki}
	
	\sect{Итоговая формулировка User Story}
	
	\begin{itogo}
	Изображение товара можно увеличить, с помощью специальной кнопки "увеличить изображение". Картинка, в увеличенном формате, всплывает в дополнительном окне. 
	\end{itogo}

}

\UCsubsubsection{Клиент хочет увидеть другие изображения товара}{HBR.TS.005.03}{

\sect{Исходные данные c Wiki}

\begin{wiki}
Если изначально системой залито несколько фотографий товара, клиент может пролистать список, при помощи стрелки передвижения вперед-назад
Если товар имеет вариантные товары, клиент может менять изображение по вариантам.
\end{wiki}

\sect{Итоговая формулировка User Story}

\begin{itogo}
Если изначально загружено несколько фотографий товара, клиент может просматривать другие изображения товара. 
Если товар имеет вариантные товары, клиент может менять изображение по вариантам.
\end{itogo}

}
\UCsubsubsection{Клиент хочет получить подробное описание товара}{HBR.TS.005.05}{

\sect{Исходные данные c Wiki}

\begin{wikilong}
Для того, чтобы получить более подробную информацию о продукте, после ключевых характеристик и данных,предусмотрены закладки с дополнительной информацией

Закладки содержат информацию о:

\begin{itemize}
\item технические характеристики 
\item дополнительная информация
\item меры предосторожности
\item паспорт безопасности( технический паспорт)
\item необходимые аксессуары ( товары без которых использование основного товара не возможно-расходники)
\item дополнительные аксессуары ( товары, которые рекомендуется использовать вместе с основным товаром) 
\item альтернативные товары (товары, которыми можно заменить основной товар, без потери в характеристиках и качестве). 
система предлагает сравнить выбранные товары в режиме on-line  по всем ключевым характеристикам, и предоставляет возможность самостоятельной настройки отображения таблицы
\item запасные части ( см. отображение вложение №4) 
\end{itemize}

Всю необходимую информацию можно так же выводить списком, вслед за основной информацией, по стандарту 
\begin{itemize}
\item запасные части
\item необходимые аксессуары
\item дополнительные аксессуары
\item альтернативные товары
\end{itemize}
\end{wikilong}

\begin{teamidea}
Откуда меры предосторожности и паспорт безопасности( технический паспорт)?
Этого не было в структуре продукта. За исключением этого - на первый этап
\end{teamidea}

\begin{hybris}
Где будут администрироваться связи между продуктами - кто мастер?
Ответ Технониколь: hybris
\end{hybris}

\sect{Итоговая формулировка User Story}

\begin{itogolong}
Для того, чтобы получить более подробную информацию о продукте, после ключевых характеристик и данных, предусмотрены закладки с дополнительной информацией

Закладки содержат информацию о:
\begin{itemize}
\item технические характеристиках 
\item дополнительной информации
\item необходимых аксессуарах (товары без которых использование основного товара не возможно-расходники)
\item дополнительных аксессуарах ( товары, которые рекомендуется использовать вместе с основным товаром) 
\item альтернативных товарах (товары, которыми можно заменить основной товар, без потери в характеристиках и качестве). 
\item запасных частях
\end{itemize}

Всю необходимую информацию можно также выводить списком, вслед за основной информацией, по стандарту 
\begin{itemize}
\item запасные части
\item необходимые аксессуары
\item дополнительные аксессуары
\item альтернативные товары
\end{itemize}

\end{itogolong}

}
\UCsubsubsection{Клиент хочет получить информацию о возможности заказать товар}{HBR.TS.005.04}{

\sect{Исходные данные c Wiki}

\begin{wiki}
Данная информация отображается в карточке товара,  в виде отдельной строки для проверки наличия товара на складе, привязанного к идентификатору-индексу.
\end{wiki}

\begin{teamidea}
Зачем необходима кнопка, если ее функции выполняет "Добавить в корзину"? Проще попробовать добавить и получить отказ (уже нет на складе), чем перед каждым нажатием на "Добавить в корзину" нажимать на "Проверить".

Поскольку интерфейс проверки все равно будет (для корзины), функционал даже в описанном виде можно на первый этап.
\end{teamidea}

\begin{hybris}
Функциональность проверки заказа на складе? Это в каком релизе и с помощью чего?

Ответ Технониколь: В первом через web service
\end{hybris}

\sect{Итоговая формулировка User Story}

\begin{itogo}
Информация о возможности заказать товар отображается в карточке товара,  в виде отдельной строки для проверки наличия товара на складе, привязанного к идентификатору-индексу.
\end{itogo}


}
\UCsubsubsection{Клиент хочет увидеть похожие товары}{HBR.TS.005.06}{

\sect{Исходные данные c Wiki}

\begin{wiki}
Для реализации этого сервиса служит закладка в карточке товара, которая называется "альтернативные товары". , а так же при наличии таковых товаров, в теле карточки отображается возможность подобрать аналоги 

Таких товаров может быть неограниченное множество.  (один аналог) или   (два и более аналогов) 

Товары могут быть сравнены в виде таблицы, по ключевым характеристикам.Отображение полей в сравнительной таблице настраивается самим пользователем.
так же доступен для сравнения выбор и с другим товаром, на усмотрение клиента
\end{wiki}

\begin{teamidea}
Без включения функционала сравнения
\end{teamidea}

\begin{hybris}
Где будут администрироваться связи между продуктами - кто мастер?

Ответ ТехноНиколь: На пилоте Hybris, на втором этапе рекомендательный сервис
\end{hybris}

\sect{Итоговая формулировка User Story}

\begin{itogo}
Для реализации этого сервиса служит закладка в карточке товара, которая называется "альтернативные товары", а так же при наличии таковых товаров, в теле карточки отображается возможность подобрать аналоги.
\end{itogo}

}
\UCsubsubsection{Клиент хочет увидеть сопутствующие товары}{HBR.TS.005.07}{

\sect{Исходные данные c Wiki}

\begin{wikilong}
\begin{enumerate}
\item Аксессуары в карточке товара должны делиться на 
обязательные (те, без которых использование продукта не возможно)
желательные (те, использование которых улучшает эффект от использования основного продукта)
\item  Аксессуары отображаются "закладками" в карточке товара 
\item  Аксессуары выводятся списком внизу и могут быть добавлены в корзину, в необходимом количестве 
\item Добавив необходимый или желательный аксессуар в коризну, клиент получает возможность увидеть подборку товаров, которые система так же рекомендует приобрести
из корзины предоставлена возможность продолжить покупки или перейти к оформлению заказа. Если выбрать один из рекомендованных товаров, система оставляет тебя в корзине,добавив выбранный товар,система перенастраивает список рекомендованных продуктов и выводит аксессуары, к последнему добавленному в корзину товару. 
\end{enumerate}
\end{wikilong}

\begin{hybris}
Где будут администрироваться связи между продуктами - кто мастер?

Ответ Технониколь: На пилоте Hybris, на втором этапе рекомендательный сервис
\end{hybris}

\sect{Итоговая формулировка User Story}

\begin{itogo}

Необходимо иметь возможность выводить товары, связанные с данным.
Предусмотреть разные типы связей товаров (желательные аксессуары, обязательные аксессуары)

Добавив связанный товар в корзину, клиент получает возможность увидеть в корзине рекомендации, составленные на основе связей товаров из корзины.  
\end{itogo}

}
\UCsubsubsection{Клиент хочет увидеть специальные предложения по товару}{HBR.TS.005.08}{

\sect{Исходные данные c Wiki}

\begin{wiki}
После основной информации, клиенту должна быть отображена информация о доступных к этому товару акциях. 

После того, как клиент положил товар в корзину, и на товар распространяется акция, она должна отобразиться в корзине, как подсказка.

Например, услуга дополнительной гарантии. Клиент может добавить ее в корзину, выбрать ее стоимость (пакет), и ознакомиться с условиями предоставления.
\end{wiki}

\sect{Итоговая формулировка User Story}

\begin{itogo}
Если на товар распространяется акция, посетитель должен иметь возможность получить информацию об этом с карточки товара и из корзины.
\end{itogo}

}
\UCsubsubsection{Клиент хочет добавить товар в корзину}{HBR.TS.005.09}{

\sect{Исходные данные c Wiki}

\begin{wiki}
Клиент может из карточки товара добавить товар в корзину.

Товар отображается как добавленный, а клиенту предлагается продолжить покупки или оформить заказ.

При этом клиенту отображается список, рекомендованных к покупке товаров, к тому, что добавлен в корзину. 

При переходе в корзину, клиент видит все товары, которые он добавлял ранее, а если он авторизуется, то увидит и товары прошлых сессий (неделю назад), которые не были доведены до checkout 
\end{wiki}

\begin{teamidea}
Описывается объединение товаров неавторизованного покупателя и корзины авторизованного после прохождения авторизации. hybris работает не так. Проговаривали вариант, когда в случае, если корзина заполнена, после авторизации она остается с тем же набором товаров. В оценку включено именно такое поведение системы на первый этап, так как это стандартное поведение акселлератора
\end{teamidea}

\begin{hybris}
Где будут администрироваться связи между продуктами - кто мастер?

Ответ ТехноНиколь: На пилоте Hybris, на втором этапе рекомендательный сервис
\end{hybris}

\sect{Итоговая формулировка User Story}

\begin{itogo}
Клиент может из карточки товара добавить товар в корзину.

Товар отображается как добавленный, а клиенту предлагается продолжить покупки или оформить заказ.

При этом клиенту в корзине отображается список, рекомендованных к покупке товаров, к тому, что добавлен в корзину. 

\end{itogo}

}
\UCsubsubsection{Клиент хочет видеть стоимость товара в зависимости от своей роли и согласно с заключенным договором на поставку товаров}{HBR.TS.005.10}{

\sect{Исходные данные c Wiki}

\begin{wiki}
Что бы клиенту отображались цены, согласно его условиям, в каждой карточке отображается информация, с просьбой авторизоваться, что бы это стало возможно. 
\end{wiki}

\sect{Итоговая формулировка User Story}

\begin{itogo}
Разные пользователи могут иметь разные цены.  На карточке товара для неаутентифицированных пользователей должна отображаться информация о том, что нужно пройти аутентификацию. 
\end{itogo}

}


\section{Корзина}
\ifcand
\subsection{Согласованные на 1 релиз}
\fi
\UCsubsubsection{Клиент хочет добавить в корзину выбранные им товары}{HBR.TS.006.01}
{

\sect{Исходные данные с Wiki}

\begin{wiki}
\begin{itemize}
\item Шаг 1. Находим нужный товар в каталоге Продуктов
\item Шаг 2. Вносим в окне "Кол-во", кол-во товара к заказу
\item Шаг 3. Нажимаем кнопку "В корзину"
\end{itemize}

Находясь в продуктовом каталоге клиент указывает количество товара для заказа и нажимает на кнопку "Добавить в корзину". 

Клиент попадает на форму в которой ему предлагается добавить в корзину товары рекомендуемые к закупке вместе с тем товаром который добавляется в корзину, а также он может выбрать дальнейшие действия и либо: продолжить выбор товаров (ссылка "Продолжить добавление" или кнопка "Оформить заказ")
\end{wiki}

\begin{hybris}
Процесс Отличается от стандартной процедура добавления товара в корзину, рекомендую исключить форму с рекомендованными товарами из процесса покупки - не думаете что будет раздражать пользователя?

Ответ Технониколь: Кейс работает в e-commerce.
\end{hybris}

\sect{Итоговая формулировка User Story}

\begin{itogo}
Шаг 1. Находим нужный товар в каталоге Продуктов
Шаг 2. Вносим в окне "Кол-во", кол-во товара к заказу
Шаг 3. Нажимаем кнопку "В корзину"
\end{itogo}


}
\UCsubsubsection{Заказ рекомендованного товара в корзине}{HBR.TS.006.02}
{
\sect{Исходные данные с Wiki}

\begin{wiki}
Шаг 1. Нажимаем ссылку "Моя корзина".
Шаг 2. Клиент вводит кол-во товара к заказу в окне "Кол-во" выбранного товара.
Шаг 3. Клиент нажимает кнопку "В корзину" и товар попадает в список товаров в корзине.

Находясь в корзине клиент видит в нижней ее части список товаров которые доступны к заказу в качестве Дополнительных к выбранным им товарам.
\end{wiki}

\begin{hybris}
Где мастер ассоциаций между товарами?

Ответ Технониколь: Excel. На пилоте hybris, на втором этапе рекомендательный сервис
\end{hybris}

\sect{Итоговая формулировка User Story}

\begin{itogo}
Находясь в корзине клиент видит в нижней ее части список товаров которые доступны к заказу в качестве рекомендованных к выбранным им товарам.
\end{itogo}

}
\UCsubsubsection{Редактирование кол-ва товара добавленного в корзину}{HBR.TS.006.04}{
\sect{Исходные данные с Wiki}

\begin{wiki}
Шаг 1. Нажимаем ссылку "Моя корзина".
Шаг 2. Меняем количество товара в окне "Кол-во".
Шаг 3. Нажимаем кнопку "Редактировать".
\end{wiki}

На форме корзины клиент может изменить кол-во заказанного товара

\sect{Итоговая формулировка User Story}

\begin{itogo}
Шаг 1. Нажимаем ссылку "Моя корзина".
Шаг 2. Меняем количество товара в окне "Кол-во" напротив товара в корзине.
Шаг 3. Нажимаем кнопку "Редактировать".

На форме корзины клиент может изменить кол-во заказанного товара
\end{itogo}

}



\UCsubsubsection{Удаление товара добавленного в корзину}{HBR.TS.006.05}{

\sect{Исходные данные с Wiki}

\begin{wiki}
Шаг 1. Нажимаем ссылку "Моя корзина"
Шаг 2. Меняем количество товара в окне "Кол-во" на "0"
Шаг 3. Нажимаем кнопку "Редактировать"

В форме корзины клиент может удалить строку добавленного товара
\end{wiki}

\sect{Итоговая формулировка User Story}

\begin{itogo}
В форме корзины клиент может удалить строку добавленного товара через кнопку "Удалить" или установкой количества товара в ноль.
\end{itogo}

}
\UCsubsubsection{Клиент хочет, чтобы неподтвержденная корзина сохранялась и при следующем входе на сайт}{HBR.TS.006.13}{

\sect{Исходные данные с Wiki}

\begin{wiki}
Шаг 1. Клиент входит на любую страницу сайта. 
Шаг 2. Клиент видит в правом верхнем углу ссылку "Моя корзина" с кол-вом строк в ней

при входе на сайт клиент должен видеть количество товаров в корзине в верхней правой части экрана.
\end{wiki}




\begin{teamidea}
Будет конфликт с требованием сохранять корзину при авторизации.

Если аноним положил 3 товара в корзину, потом авторизовался под Ивановым, который вчера бросил корзину с другими 2 товарами, то есть три варианта действий:
\begin{enumerate}
\item очистить корзину анонима (3) и заменить ее корзиной Иванова (2). 
\item оставить корзину анонима (3), про корзину Иванова забыть (Вариант "как в Хайбрисе")
\item объединить корзины (3+2). При разлогинивании оставить только анонимную (3). Сложный вариант.
\end{enumerate}
\end{teamidea}

\begin{hybris}
в стандартном B2B accelerator'e если у анонима есть товар в корзине, потом он входит на сайт под своим именем - корзина авторизированного пользователя заменяется корзиной анонима. Надо определиться с логикой этих замен корзин или их слиянием

Ответ Технониколь: Меня вот это тоже взволновало, есть best practice по слиянию? 
\end{hybris}

\sect{Итоговая формулировка User Story}

\begin{itogo}
Шаг 1. Клиент входит на любую страницу сайта. 
Шаг 2. Клиент видит в области "Корзина" ссылку "Моя корзина" с количеством товаров, отложенных в корзину.

При открытии любой страницы сайта клиент должен видеть количество товаров в корзине в соответствующей области.
\end{itogo}

}


\UCsubsubsection{Клиент хочет начать оформление заказа воспользовавшись функцией корзины Checkout }{HBR.TS.006.19}
{

\sect{Исходные данные с Wiki}

\begin{wikilong}
\begin{itemize}
\item Шаг 1. после нажатия кнопки Checkout клиент попадает в новую форму в которой ему предлагается выбор "Типа доставки" и "Способа оплаты"
\item Шаг 2. После выбора типа "Доставка" появляется окно для ввода дополнительной информации по доставке
\item Шаг 3. у клиента есть возможность выставить галку и сделать такой метод доставки методом по умолчанию
\item Шаг 4. у клиента есть возможность установить галкой необходимость ожидания полной комплектации заказа перед его доставкой
\item Шаг 5. у клиента есть возможность выбрать один из сохраненных адресов или ввести новый
\item Шаг 6. у клиента есть возможность внести (и или откоректировать персональные данные, страну, адрес, индекс, город, контактный телефон, контактный адрес электронной почты, и название компании в текстовых полях или оставить в них значения по умолчанию.
\item Шаг 7. у клиента есть возможность сохранить введенную информацию как один из своих адресов доставки
\item Шаг 8. клиент может выбрать из списка оператора по доставке
\item Шаг 9. клиент может указать будет ли он платить за доставку отдельно оператору или хочет чтобы поставщик включил стоимость доставки в счет
\item Шаг 10. клиент может ввести данные для наклейке к грузу (упаковочного листа) указав детальную информацию по доставке, свою должность, номер почтового ящика, номер проекта, дополнительные номера для идентификации груза
\item Шаг 11. после ввода данных клиент может продолжить чекаут корзхины нажав кнопку Сохранить, данные по доставке сохраняются
\item Шаг 12. клиент имеет возможность выбрать один из доступных способов оплаты (Master, Visa, Amex)
\item Шаг 13. клиент может указать свои платежные реквизиты, имя фамилию, номер карты, дата ее окончания, комментарий, сохранить данные о карте в качестве одного из методов оплаты по умолчанию
\item Шаг 14. клиент может выбрать опцию и указать свои координаты доставки как адрес доставки документов или ввести новые (Фамилию, Имя, Компанию, Адрес 1, Адрес 2, Город, Штат, Индекс, телефон и факс)
\item Шаг 15. после нажатия на кнопку "Сохранить" информация о платежных реквизитах сохранится
\item Шаг 16. клиент имеет возможность завершить оформление заказа нажав кнопку "Продолжить"
\end{itemize}

Клиент после завершения подборра товара в корзину может ввести адрес доставки, указать способ оплаты и подтвердить заказ к исполнению.
\end{wikilong}

\sect{Пояснения}

\begin{teamidea}
Описанный процесс отличается от стандартного B2B Checkout. 
Описанный процесс требует сертификации по PCI DSS.
Описанный процесс требует интеграции с процессинговым центром.
Многое в описанном непонятно.
Слишком много шагов. Настолько длинный механизм заказа не встречается ни на одном сайте - редко кто доберется до конца.
Оператор доставки = метод доставки? Другого способа не предусмотрено.
\end{teamidea}

\sect{Итоговая формулировка User Story}

\begin{itogolong}
 Из корзины неаутентифицированные пользователи по нажатию на "Оформление заказа" попадают на форму аутентификации (т.к. требуется быть аутентифицированным для совершения заказа). После аутентификации пользователь попадает на страницу оформления заказа.
\begin{itemize}
\item Выбор метода платежа (по карте, выписать счет)
\item В случае выбора "выписать счет" доступен блок "Выбор Cost Center".
\item В случае выбора Cost Center и выбора метода платежа доступен блок "Адрес доставки"
\item В случае отсутствия адреса доставки предлагается ввести адрес доставки и привязать в дальнейшем к пользователю
\item В случае наличия нескольких адресов доставки предлагается выбрать нужный
\item В случае выбора адреса доставки, предлагается выбрать метод доставки
\item После указания адреса доставки, метода доставки, Cost Center пользователь может
\begin{itemize}
\item указать, что заказ является заказом по расписанию, после подтвердить заказ
\item сделать запрос на предложение и подтвердить заказ
\item подтвердить заказ
\end{itemize}
\item В случае выбора заказа по расписанию, пользователь может ввести
\begin{itemize}
\item Дату начала заказов по расписанию - дату первого заказа
\item Расписание регулярности заказов, а именно одно из
\begin{enumerate}
\item Через указанное число дней
\item В указанный день недели через указанное число недель
\item В указанный день месяца
\end{enumerate}
\end{itemize}
\end{itemize}
Страница оформления заказа. Включает:
В случае выбора запроса на предложение - ввести в свободной форме комментарий и выполнить запрос
\end{itogolong}
}
\ifcand
\subsection{Кандидаты на последующие этапы}
\UCsubsubsection{Быстрое добавление товара в корзину}{HBR.TS.006.06}{}
\UCsubsubsection{Проверка доступности товара на складе ближайшем к месту доставки}{HBR.TS.006.07}{}
\UCsubsubsection{Клиент может проверить доступность товара набранного в корзину на ближайшем складе после ввода индекса}{HBR.TS.006.09}{}
\fi

\section{Заказ}
\ifcand
\subsection{Согласованные на 1 релиз}
\fi
\UCsubsubsection{Клиент хочет подтвердить корзину и перейти к оформлению заказа}{HBR.TS.007.01}{}
\UCsubsubsection{Клиент хочет видеть пошаговый механизм оформления заказа}{HBR.TS.007.03}{}
\UCsubsubsection{Клиент хочет получать подсказки о назначении элементов интерфейса в контекстном меню}{HBR.TS.007.04}{}
\UCsubsubsection{Клиент хочет выбирать тип оплаты}{HBR.TS.007.05}{}
\UCsubsubsection{Клиент хочет выбирать юридическое лицо для конкретного заказа}{HBR.TS.007.06}{}
\UCsubsubsection{Клиент хочет выбирать способ отгрузки из предлагаемого списка способов доставки}{HBR.TS.007.08}{}
\UCsubsubsection{Клиент хочет изменить дату доставки на последнем шаге оформления заказа (финальная страница order review)}{HBR.TS.007.13}{}
\UCsubsubsection{Клиент хочет распечатать спецификацию на этапе корзины (текущей или сохранённой)}{HBR.TS.007.15}{}
\UCsubsubsection{Клиент хочет видеть подсказки по оформлению заказа}{HBR.TS.007.16}{}
\UCsubsubsection{Клиент хочет видеть сообщения о неудачной обработке заказа и причину}{HBR.TS.007.17}{}
\UCsubsubsection{Клиент хочет связаться со специалистом call-центра для получения помощи}{HBR.TS.007.18}{}
\UCsubsubsection{Клиент хочет поменять тип оплаты}{HBR.TS.007.19}{}
\UCsubsubsection{Клиент хочет иметь возможность разместить свой заказ}{HBR.TS.007.22}{}
\UCsubsubsection{Клиент хочет получать уведомление на свой e-mail о том, что он успешно разместил заказ}{HBR.TS.007.25}{}
\UCsubsubsection{Клиент хочет получить информацию о том, что заказ не удалось разместить и причину неудачи}{HBR.TS.007.26}{}
\UCsubsubsection{Клиент хочет, чтобы с ними связался менеджер, если заказ не удалось разместить}{HBR.TS.007.27}{}
\UCsubsubsection{Клиент хочет связаться с менеджером, если у него возникли вопросы по размещению заказа}{HBR.TS.007.28}{}
\UCsubsubsection{Клиент хочет регулярно получать одни и те же товары, без дополнительной процедуры оформления заказа}{HBR.TS.007.29}{}
\UCsubsubsection{Клиент хочет видеть, что товар в наличии на складе}{HBR.TS.007.30}{}
\UCsubsubsection{Клиент хочет видеть, когда на складе может появиться товар, который он хочет заказать}{HBR.TS.007.31}{}
\UCsubsubsection{Клиент хочет иметь возможность выбрать любой тип отгрузки не зависимо от доступности (статуса) товара}{HBR.TS.007.38}{}
\UCsubsubsection{Клиенту доступен бесплатный способ доставки}{HBR.TS.007.39}{}

\ifcand
\subsection{Кандидаты на последующие этапы}
\UCsubsubsection{Клиент хочет сохранить корзину и перейти к списку сохранённых корзин}{HBR.TS.007.02}{}
\UCsubsubsection{Клиент хочет сделать запрос на коммерческое предложение}{HBR.TS.007.07}{}
\UCsubsubsection{Клиент хочет указать желаемую дату доставки при оформлении заказа}{HBR.TS.007.09}{}
\UCsubsubsection{Клиент хочет видеть прогнозируемую дату доставки}{HBR.TS.007.10}{}
\UCsubsubsection{Клиент хочет, чтобы стоимость доставки добавилась к стоимости заказа}{HBR.TS.007.11}{}
\UCsubsubsection{Клиент хочет оставить комментарий для курьера при оформлении заказа}{HBR.TS.007.12}{}
\UCsubsubsection{Клиент хочет иметь возможность оплатить заказ картой на этапе его формирования}{HBR.TS.007.20}{}
\UCsubsubsection{Клиент хочет иметь возможность использовать раннее введенные данные кредитной карты}{HBR.TS.007.21}{}
\UCsubsubsection{Клиент хочет видеть стоимость заказа с НДС/без НДС}{HBR.TS.007.23}{}
\UCsubsubsection{Клиент хочет видеть, что на складе имеется ограниченное кол-во товара }{HBR.TS.007.32}{}
\UCsubsubsection{Клиент хочет видеть, что некоторое кол-во заказанного товара не доступно к отгрузке }{HBR.TS.007.33}{}
\UCsubsubsection{Клиент хочет видеть доступность акционного товара}{HBR.TS.007.34}{}
\UCsubsubsection{Клиент хочет видеть, когда товар снимается с производства }{HBR.TS.007.35}{}
\UCsubsubsection{Клиент хочет видеть, когда товар временно недоступен}{HBR.TS.007.36}{}
\UCsubsubsection{Клиент хочет видеть возможные способы оплаты, в зависимости от выбранного способа отгрузки}{HBR.TS.007.37}{}
\fi

\section{Управление заказами}
\ifcand
\subsection{Согласованные на 1 релиз}
\fi
\UCsubsubsection{Клиент хочет отменить заказ}{HBR.TS.008.02}{}
\UCsubsubsection{Клиент хочет создать заказ на основании старой заказа}{HBR.TS.008.04}{}
\UCsubsubsection{Клиент хочет утвердить заказ}{HBR.TS.008.05}{}
\UCsubsubsection{Клиент хочет посмотреть статусы заказа построчно}{HBR.TS.008.06}{}
\ifcand
\subsection{Кандидаты на последующие этапы}
\UCsubsubsection{Клиент хочет изменить заказ}{HBR.TS.008.03}{}
\fi

\ifcand
\section{Каталог версии Staged}
\subsection{Кандидаты на последующие этапы}
\UCsubsubsection{Создание каталога}{HBR.TS.009.1}{}
\UCsubsubsection{Создание категорий каталога Master}{HBR.TS.009.2}{}
\UCsubsubsection{Создание категорий каталога Web версии Staged}{HBR.TS.009.3}{}
\UCsubsubsection{Создание категорий в каталоге Web, версии поставщика}{HBR.TS.009.4}{}
\UCsubsubsection{Создание товаров поставщиком }{HBR.TS.009.5}{}
\UCsubsubsection{Редактирование категорий в каталоге Web версии поставщика}{HBR.TS.009.6}{}
\UCsubsubsection{Редактирование товаров в каталоге Web версии поставщика}{HBR.TS.009.7}{}
\fi

\ifcand
\section{Конвертация посетителей в покупателей  (захват)}
\subsection{Кандидаты на последующие этапы}
\UCsubsubsection{Первый визит клиента на сайт}{HBR.TS.011.01}{}
\UCsubsubsection{регистрация клиента на сайте}{HBR.TS.011.02}{}
\UCsubsubsection{Клиент зарегистрировался, но не совершил покупки}{HBR.TS.011.03}{}
\UCsubsubsection{Клиент зарегистрировался, и подключил к аккаунту других сотрудников компании}{HBR.TS.011.04}{}
\UCsubsubsection{Клиент зарегистрировался и совершил покупку}{HBR.TS.011.05}{}
\UCsubsubsection{Клиент зарегистрировался, настроил информацию о себе}{HBR.TS.011.06}{}
\UCsubsubsection{На сайте доступна подписка на e-mail рассылки}{HBR.TS.011.07}{}
\UCsubsubsection{Привлечение пользователя в социальную сеть}{HBR.TS.011.08}{}
\UCsubsubsection{Поведенческий и look-alike таргетинг}{HBR.TS.011.09}{}
\UCsubsubsection{Смс-оповещение}{HBR.TS.011.10}{}
\UCsubsubsection{Рассылка письма при брошенной регистрации}{HBR.TS.011.11}{}
\UCsubsubsection{Захват e-mail адресов}{HBR.TS.011.12}{}
\UCsubsubsection{Захват e-mail адресов, из адресной книги клиентов}{HBR.TS.011.13}{}
\UCsubsubsection{Получение адреса клиента, если он ничего не купил и не оставил данных}{HBR.TS.011.14}{}
\fi

\ifcand
\section{Услуги}
\subsection{Кандидаты на последующие этапы}
\UCsubsubsection{Персональный менеджер}{HBR.TS.012.01}{}
\UCsubsubsection{Резервирование товара}{HBR.TS.012.02}{}
\UCsubsubsection{Экспресс-доставка (для регионов).}{HBR.TS.012.03}{}
\UCsubsubsection{Услуга по страхованию перевозки (для регионов).}{HBR.TS.012.04}{}
\UCsubsubsection{Техническое сопровождение }{HBR.TS.012.05}{}
\UCsubsubsection{Обследование сооружений}{HBR.TS.012.06}{}
\UCsubsubsection{Разработка технических предложений и обоснований }{HBR.TS.012.07}{}
\UCsubsubsection{Проектирование : расчет архитектурной и градостроительной концепции;}{HBR.TS.012.08}{}
\UCsubsubsection{Гарантия онлайн на товары собственного производства }{HBR.TS.012.09}{}
\UCsubsubsection{Калькуляторы расчета кровли}{HBR.TS.012.10}{}
\UCsubsubsection{КОНСУЛЬТАЦИИ НА ОБЪЕКТЕ по особенностям монтажа материалов и комплектующих кровельной системы}{HBR.TS.012.11}{}
\UCsubsubsection{Inventory management}{HBR.TS.012.12}{}
\UCsubsubsection{Пост продажный сервис, ремонт оборудования}{HBR.TS.012.13}{}
\UCsubsubsection{Преимущества, доступные для пользователей e-commerce}{HBR.TS.012.14}{}
\UCsubsubsection{Поддержка технических специалистов/Поддержка по продукту}{HBR.TS.012.15}{}
\UCsubsubsection{Национальный аккаунт (могла не так понять, просьба утвердить или опровергнуть)}{HBR.TS.012.16}{}
\UCsubsubsection{Бытовые услуги}{HBR.TS.012.17}{}
\UCsubsubsection{Авто-заказ}{HBR.TS.012.18}{}
\UCsubsubsection{Дополнительная гарантия на товар}{HBR.TS.012.19}{}
\fi