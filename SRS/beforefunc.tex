



\secl{Цели разработки системы}

	\subsecl{Цель проекта}
	
	Создать и внедрить платформу электронного каталога на базе системы SAP Hybris, как инструмент мультиканальной модели продаж, включая разработку и внедрение, связанных с ним бизнес-процессов.
	
	\subsecl{Цель разработки системы}
	
	Автоматизировать процессы продажи покупателям Технониколь.  
	
	Обеспечить возможность покупателям Технониколь организовывать процессы согласования заявок внутри своих групп сотрудников. 
	
	\subsecl{Ключевая задача настоящего Проекта}

	Адаптация и внедрение платформы Hybris на базе B2B Accelerator в рамках утвержденного функционального объема. 

\secl{Соглашения о терминах}

\textbf{Спецификация требований} -- согласованное описание предполагаемого поведения программного обеспечения.

\textbf{Требование} -– это условие, которому должно удовлетворять программное обеспечение, или свойство, которым оно должно обладать, чтобы 
\begin{itemize}
\item удовлетворить потребность пользователя в решении некоторой задачи; 
\item удовлетворить требования контракта, спецификации или стандарта. 
\end{itemize}

\textbf{Бизнес-требования} — определяют назначение ПО, описываются в договоре, уставе проекта.

\textbf{Функциональные требования} -- требования, определяющие действия, которые должна выполнять система. Описываются в документе Спецификация требований.

\textbf{Нефункциональные требования} -- требования к характеру поведения системы.

\textbf{Проект} -- сокращенное название проекта. Полное название проекта -- <<Адаптация и внедрение SAP Hybris B2B Accelerator в <<ООО УТС ТехноНИКОЛЬ>>. 

\textbf{Система} -- сокращенное название системы <<Первая Платформа>>, техническая составляющая которой, является результатом работ по данной спецификации.

\textbf{Пользователи системы} -- человек или внешняя система, взаимодействующие с Системой через интерфейсы.


\secl{Назначение настоящего документа}

Документ <<Спецификация требований>> является систематизацией согласованных на рабочих встречах требований. По своему назначению документ <<Спецификация требований>>  не включает предложений TeamIdea по реализации, но фиксирует требования ТехноНИКОЛЬ к этой реализации. По своему назначению документ <<Спецификация требований>> не является обязательством TeamIdea выполнить указанные работы до тех пор, как не станет приложением к договору. Согласно своему назначению и в соответствии с договором (п.2) документ <<Спецификация требований>> включает только структурированные технические требования, сценарии пользовательских действий и не включает способы реализации бизнес-требований, структуры данных, методы решения задач, если они не являются требованиями заказчика. 

\secl{Цель настоящего документа}

Целью документа <<Спецификация требований>> является:

\begin{enumerate}
\item \textbf{Создание основы для соглашения между заказчиком и разработчиком по поводу функций, которые должен выполнять программный продукт}. Описание функций ПО, приведенное в Спецификации требований, поможет потенциальным пользователям определить, будет ли отвечать Система их требованиям и ожиданиям или как необходимо изменить ПО, чтобы удовлетворить эти потребности. 
\item \textbf{Уменьшение объема работ по разработке}. Сокращает последующее повторное проектирование, кодирование и тестирование, позволяет вскрыть упущения, неправильное понимание и противоречия на ранних циклах разработки, когда эти проблемы проще исправить.
\item \textbf{Обеспечение основы для оценки расходов и графика работ}. Описание Системы, разрабатываемой в соответствии со Спецификацией требований, является практической основой для оценки затрат на проект и будет использоваться для формирования контракта и бюджетных ограничений
\item \textbf{Обеспечение основы для аттестации и валидации}. При использовании Спецификации значительно увеличивается эффективность процесса приемки ПО. Спецификация требований обеспечивает основу для проведения проверки полноты и качества реализации.
\item \textbf{Основой для расширения работ в дальнейшем}. При последующих изменениях продукта упрощается и ускоряется процесс проверки новых требований на совместимость с уже реализованными.  
\end{enumerate}

\secl{Ссылки на источники}

Источником общей информации и требований к результатом работ являются:
\begin{itemize}
\item Договор \No 0302/С-14 от 03.03.2014
\item Дополнительное соглашение к нему \No 2 от 14.04.2014. 
\end{itemize}

Источником функциональных требований большинства \footnote{На 08.04.2014 -- всех} требований являются результаты встреч -- протоколы интервью, предусмотренные договором. Требования из протоколов интервью сопровождаются соответствующей пометкой. 

\begin{itemize}
\item Протокол интервью от 19 марта (в т.ч. с изменениями от 04.04.13)
\item Протокол интервью от 21 марта (в т.ч. с изменениями от 27.03.13)
\item Протокол интервью от 04 апреля (в т.ч. с изменениями от 08.04.13)
\end{itemize}

\secl{Рамки функциональности системы}

Рамки функциональности системы определяются базовым перечнем требований к системе SAP hybris. Данные рамки могут быть пересмотрены при подписании договора на разработку. 

\begin{longtable}{|p{0.5\textwidth}|p{0.5\textwidth}|}
\input{restrictions}
\end{longtable}

%\secl{Ограничения и предположения}

\secl{Правила и стандарты}
\subsecl{Используемые при разработке спецификации}

При разработке спецификации Исполнитель руководствовался рекомендациями IEEE 830 к структуре и составу документа.

\secl{Общее описание системы}

Система <<Первая платформа>> предназначена для автоматизации взаимодействия <<ТехноНИКОЛЬ>> и покупателей. С позиции интерфейса система представляет собой следующие крупные компоненты 
\begin{itemize}
\item электронный каталог товаров, с удобным поиском и навигацией по нему 
\item авторизацию покупателей и личный кабинет
\item оформление заказанного товара
\item набор компонентов SAP hybris для управления данными (кокпиты)
\end{itemize}

\secl{Классы и характеристики пользователей}

Пользователями системы являются 
\begin{itemize}
\item клиенты ТехноНИКОЛЬ (<<покупатели>>), 
\item сотрудники ТехноНИКОЛЬ (для управления данными),
\item SAP PI.
\end{itemize}

\secl{Этапность требований}

В ходе спецификации отдельные требования были отнесены к первому или второму релизу. 